% Document-Author: KaleidosCode
% Document-Date: 27/08/2017
% Document-Description: Lettera di presentazione
\author{KaleidosCode}
\date{27/08/2017}

\documentclass[a4paper,12pt]{article}
\usepackage{../../../Templates/kaleidos}
\usepackage{titletoc}
\newcommand\VRule[1][\arrayrulewidth]{\vrule width #1}
\pagestyle{empty}
\intestazioni{yht}

\begin{document}
	\begin{titlepage}
		\includegraphics[scale=0.2]{../../../Immagini/KaleidosCodeLogo.png}
		\hrule
		\vspace{1.2cm}
		\flushright 28 agosto 2017\\
		\vspace{0.4cm}
		Alla gentile attenzione del Committente:\\
		\vardanega\\
		\cardin\\
		Università degli Studi di Padova\\
		Dipartimento di Matematica\\
		Via Trieste, 63\\
		35121, Padova (PD)\\
		\vspace{1.2cm}
		\flushleft Il \responsabilediprogetto, Pezzuto Francesco - \kaleidoscode\\
		\vspace{0.4cm}
		Oggetto: \textbf{partecipazione alla Revisione di Accettazione}.\\
		\vspace{1cm}
		Egregio \vardanega,\\
		\vspace{0.4cm}
		Con la presente, il gruppo \kaleidoscode\ intende comunicarLe ufficialmente l'intenzione di
		partecipare alla Revisione di Accettazione per la realizzazione del prodotto da Lei commissionato:
		\begin{center}
			\textbf{SWEDesigner: editor di diagrammi UML con generazione di codice} 
		\end{center}
		proposto da \proponente.\\
		Alla presente lettera sono allegati i seguenti documenti:
		\begin{itemize}
			\item \analisideirequisitiRA\ (Esterni/AnalisiDeiRequisiti\char`_v4.0.0.pdf);
			\item \definizionediprodottoRA\ (Esterni/DefinizioneDiProdotto\char`_v2.0.0.pdf);
			\item \glossarioRA\ (Esterni/Glossario\char`_v2.0.0.pdf);
			\item \manualeutenteRA\ (Esterni/ManualeUtente\char`_v2.0.0.pdf);
			\item \normediprogettoRA\ (Interni/NormeDiProgetto\char`_v4.0.0.pdf);
			\item \pianodiprogettoRA\ (Esterni/PianoDiProgetto\char`_v4.0.0.pdf);
			\item \pianodiqualificaRA\ (Esterni/PianoDiQualifica\char`_v4.0.0.pdf);	
			\item \specificatecnicaRA\ (Interni/SpecificaTecnica\char`_v3.0.0.pdf);
			\item \studiodifattibilitaRA\ (Interni/SpecificaTecnica\char`_v1.0.0.pdf);
			\item \textit{Verbale\char`_2017-07-30} (Verbali/Interni/Verbale\char`_2017-07-30.pdf).
		\end{itemize}
		Con questa consegna, il gruppo \kaleidoscode\ fornisce il prodotto nella sua forma finale al
		costo di \hbox{\euro\ 12 229,00}.\\
		I file di codice sorgente ed il manuale sviluppatore (in formato ipertestuale) sono situati
		all'interno della cartella ``SWEDesigner-source'' del CD-ROM consegnato.\\
		I dettagli di pianificazione adottata, progettazione e qualità del prodotto sono trattati approfonditamente
		nei documenti allegati.
		Di seguito, il link al repository del codice nel quale ha lavorato il team:
		\begin{center}
			\url{https://github.com/KaleidosCodeTeam/SWEDesigner-source}
		\end{center}
		\vspace{0.8cm}
		\begin{center}
			\textbf{Organigramma del gruppo}
			\begin{table}[H]
				\center
				\begin{tabular}{!{\VRule[1.4pt]}l!{\VRule}c!{\VRule}l!{\VRule[1.4pt]}}
					\noalign{\hrule height 1.4pt}
					\rowcolor{blue!30}\textbf{Nome} & \textbf{Matricola} & \textbf{Posta elettronica} \\ \hline
					Bonato Enrico & 1096071 & enrico.bonato.5@studenti.unipd.it \\ \hline
					Bonolo Marco & 1102360 & marco.bonolo@studenti.unipd.it \\ \hline
					Pace Giulio & 1102974 & giulio.pace@studenti.unipd.it \\ \hline
					Pezzuto Francesco & 1116523 & francesco.pezzuto@studenti.unipd.it \\ \hline
					Sanna Giovanni & 1029744 & giovannibruno.sanna@studenti.unipd.it \\ \hline
					Sovilla Matteo & 1124500 & matteo.sovilla@studenti.unipd.it \\
					\noalign{\hrule height 1.4pt}
				\end{tabular}
			\end{table}
		\end{center}
		\vspace{1.4cm}
		Rimango a Sua completa disposizione per ogni ulteriore chiarimento e ringrazio per l'attenzione.\\
		\vspace{0.8cm}
		Distinti saluti,
		\flushright Pezzuto Francesco\\
		\vspace{0.4cm}
		\includegraphics[scale=0.5]{../../../Immagini/Firme/FrancescoPezzuto.png}
	\end{titlepage}
\end{document}
