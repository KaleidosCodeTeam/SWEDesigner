\documentclass[../AnalisiDeiRequisiti.tex]{subfiles}
\begin{document}
	\section*{Ordine del giorno}
		\begin{itemize}
			\item Rinegoziazione di alcuni requisiti secondo quanto scritto nel verbale interno del 27/06/2017;
			\item Presentazione del prodotto e panoramica completa delle funzionalità già implementate;
			\item Accordo con il Proponente sul da farsi per la consegna in fase di \revisionediaccettazione.
		\end{itemize}
		Di seguito, il riassunto delle discussioni fatte.\\
		
		Come prima cosa il gruppo ha esposto le motivazioni secondo le quali si ritenesse opportuno eliminare
		alcuni requisiti. Il Proponente è stato completamente d'accordo sia per quanto riguarda la fusione tra
		activity diagram e bubble diagram che per l'eliminazione della possibilità di creare classi parametriche.\\
		Il gruppo ha mostrato al Proponente quanto del prodotto realizzato finora. Il Proponente è rimasto
		soddisfatto dell'applicazione ma ha comunque fatto alcune osservazioni:
		\begin{itemize}
			\item La grafica potrebbe essere migliorabile adeguandola ad uno stile più moderno.;
			\item Nell'interfaccia utente ci sono delle porzioni di schermo inutilmente occupate da margini o
			bottoni troppo grandi che sottraggono spazio all'area di lavoro;
			\item Non è presente un'indicazione visiva immediata (come il cambio del tipo di puntatore-mouse)
			durante alcune azioni come l'inserimento di una relazione tra due elementi di un diagramma.
		\end{itemize}
		Tutto ciò andrebbe ad incidere sull'usabilità del prodotto.\\
		Il gruppo ha quindi ordinato per priorità assieme al Proponente, le cose ancora da fare per il periodo di
		\revisionediaccettazione, visto il ridotto numero di ore (produttive e non) da dedicare ulteriormente al
		progetto:
		\begin{enumerate}
			\item Ottimizzare e rifinire il prodotto fin'ora realizzato, seguendo anche i suggerimenti forniti;
			\item Implementazione della base dati per le custom bubble predefinite lato server;
			\item Implementazione della funzionalità di Undo-Redo (ritenuta dal proponente, complessa e soprattutto
			dispendiosa in termini di tempo).
		\end{enumerate}
		Per la realizzazione della base dati, il Proponente ha suggerito di valutare l'utilizzo di PostgreSQL o
		ArangoDB.\\
		
		Lista di cose da ottimizzare/rifinire nel prodotto, emerse durante l'incontro:
		\begin{itemize}
			\item Cambiare il tipo di cursore in base all'azione che si sta compiendo;
			\item Specificare il nome del package, classe, operazione nel percorso attuale;
			\item Evidenziare l'elemento selezionato nell'area di lavoro;
			\item Ridimensionare i contenuti su schermo per massimizzare lo spazio occupato dall'area di lavoro;
			\item Se possibile, inserire la lista di attributi e metodi della classe corrente, all'interno di un
			diagramma delle bubble;
			\item Sostituire le icone dei bottoni attuali, facenti parte del pannello strumenti, con altre più
			significative per tipo di strumento;
			\item Implementare un cambio di colore per l'elemento nell'area di lavoro al quale viene cambiato il
			grado di importanza, al posto di avere un filtro di visualizzazione ``a layer'';
			\item Alla generazione del codice sorgente, compilarlo lato server ed inviare eventualmente il report
			degli errori emersi;
			\item Quando si sta per creare un nuovo progetto, chiedere se si vuole il lavoro corrente.
		\end{itemize}
		Documenti di riferimento alle decisioni prese:
		\begin{itemize}
			\item \analisideirequisitiv;
			\item \pianodiprogettov.
		\end{itemize}
\end{document}
