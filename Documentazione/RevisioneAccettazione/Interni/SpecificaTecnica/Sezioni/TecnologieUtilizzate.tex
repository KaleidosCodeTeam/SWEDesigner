\documentclass[../PianoDiQualifica.tex]{subfiles}
\begin{document}
	\section{Tecnologie utilizzate}
		\subsection{HTML5}
			Linguaggio per la strutturazione delle pagine web, come richiesto dal Proponente.
			\paragraph{Principali vantaggi}
				\begin{itemize}
					\item Possibilità di gestire immagini, canvas, e audio direttamente
					attraverso Javascript;
					\item Linguaggio ben documentato;
					\item Il suo corretto utilizzo permette di separare struttura e contenuti delle
					pagine web	dalla loro presentazione e comportamento. Questi ultimi vengono realizzati con
					altri linguaggi, aumentando quindi la manutenibilità del prodotto.
				\end{itemize}
			\paragraph{Principali svantaggi}
				\begin{itemize}
					\item Linguaggio non ancora	pienamente supportato da tutti i \gl{browser}.
				\end{itemize}
		\subsection{CSS}
			Linguaggio per la formattazione e presentazione delle pagine \gl{HTML}, come richiesto dal
			Proponente.
			\paragraph{Principali vantaggi}
			\begin{itemize}
					\item Il suo corretto utilizzo permette di separare totalmente la presentazione
					della struttura delle pagine HTML;
					\item Diminuisce i tempi di sviluppo e restyling di un sito, aumentandone quindi
					la manutenibilità;
					\item Consente di produrre pagine più leggere, riducendo i tempi di attesa per
					gli utenti;
					\item Il suo corretto utilizzo consente di aumentare l'accessibilità di un sito
					a \gl{screen-reader}, browser testuali e dispositivi alternativi.
				\end{itemize}
			\paragraph{Principali svantaggi}
				\begin{itemize}
					\item La versione 3 del linguaggio non è ancora pienamente supportata da tutti
					i browser. 
				\end{itemize}
		\subsection{Javascript}
			Linguaggio utilizzato per la realizzazione del comportamento delle pagine HTML, come
			richiesto dal Proponente.
			\paragraph{Principali vantaggi}
			\begin{itemize}
					\item Il codice Javascript viene eseguito dal browser; di conseguenza
					il server non è sfruttato più del dovuto;
					\item Permette di gestire le azioni dell'utente attraverso la gestione degli eventi;
					\item Rende possibile interagire con il \gl{DOM}.
				\end{itemize}
			\paragraph{Principali svantaggi}
				\begin{itemize}
					\item Il linguaggio non è fortemente tipizzato.
				\end{itemize}
		\subsection{\gl{JointJS}}
			Libreria Javascript scelta per la creazione dell'editor dei diagrammi UML.\\
			(\url{https://www.jointjs.com/opensource} - 02/05/2017)
			\paragraph{Principali vantaggi}
			\begin{itemize}
					\item Fornisce elementi grafici di diagrammi UML;
					\item Elementi e collegamenti interattivi;
					\item Serializzazione/de-serializzazione da/a formato \gl{JSON};
					\item Supporto a \gl{Node.js}.
				\end{itemize}
			\paragraph{Principali svantaggi}
				\begin{itemize}
					\item La versione open-source utilizzata è dipendente da altre librerie
					(jQuery, Lodash, Backbone), rendendo le ulteriori scelte tecnologiche obbligate
					all'utilizzo di	queste ultime.
				\end{itemize}
		\subsection{jQuery}
			Libreria Javascript utilizzata da JointJS utile allo sviluppo di \gl{script} lato client.\\
			(\url{https://jquery.com/} - 02/05/2017)
			\paragraph{Principali vantaggi}
			\begin{itemize}
					\item Facilita la manipolazione del DOM;
					\item Facilita lo sviluppo di comunicazioni asincrone tra client e server
					utilizzanti AJAX;
					\item Facilita la realizzazione di animazioni a livello base.
				\end{itemize}
			\paragraph{Principali svantaggi}
				\begin{itemize}
					\item Non tutta la libreria è sviluppata rispettando uno standard comune, rendendo
					eventualmente necessario manipolarne il codice per i propri bisogni;
					%\item Può rallentare un sito nel caso di manipolazioni multiple simultanee del DOM.
				\end{itemize}
		\subsection{Lodash}
			Libreria Javascript utilizzata da JointJS utile per svolgere operazioni di base all'interno
			di script.\\
			(\url{https://lodash.com/} - 02/05/2017)
			\paragraph{Principali vantaggi}
			\begin{itemize}
					\item Fornisce molte funzionalità utili per la manipolazione degli oggetti;
					\item Rendono più semplice la valutazione e la manipolazione dei test.
				\end{itemize}
		\subsection{Backbone.js}
			\gl{Framework} Javascript utilizzato da JointJS utile per fornire una struttura ad applicazioni
			web rendendo disponibili modelli con binding chiave-valore ed eventi personalizzati,
			collezioni con una \gl{API} contenente funzioni enumerabili, viste con gestione degli eventi
			dichiarativa ed un'interfaccia JSON RESTful.\\
			(\url{http://backbonejs.org/} - 02/05/2017)
			\paragraph{Principali vantaggi}
			\begin{itemize}
					\item Compatto e versatile;
					\item Contiene solo le componenti base necessarie a strutturare una web app
					secondo il pattern MVC;   %CHIEDERE
					\item Ha una buona documentazione; inoltre è presente una versione commentata
					del codice sorgente dove è quindi spiegato come lavora nel dettaglio;
					\item Supporta \gl{plugins} di terze parti.
				\end{itemize}
			\paragraph{Principali svantaggi}
				\begin{itemize}
					\item Non supporta il \gl{data binding} bidirezionale;
					\item È difficile eseguire test di unità sulle views scrivendo poco codice per \gl{mocking}.
				\end{itemize}
		\subsection{Node.js}
			Runtime Javascript open-source scelto per sviluppare la parte server di \progetto\,
			come richiesto dal Proponente (Requisito R0V1); utilizza un modello I/O non bloccante
			ad eventi asincroni ed è costruito sul motore Javascript v8; non è \gl{multi-threaded}, ma
			funziona in un singolo thread con il concetto di callback, inoltre esegue loop basato
			su eventi a singolo thread così da rendere non bloccanti tutte le esecuzioni.\\
			(\url{https://nodejs.org/it/} - 02/05/2017)
			\paragraph{Principali vantaggi}
			\begin{itemize}
					\item I/O ad eventi asincroni aiutano la gestione di richieste simultanee;
					\item Condivide la stessa porzione di codice con entrambi i lati client e server;
					\item Community molto attiva con molto codice condiviso via \gl{GitHub}, ecc.
				\end{itemize}
			\paragraph{Principali svantaggi}
				\begin{itemize}
					\item Rende complessa la gestione di \gl{database} relazionali.
				\end{itemize}
			\subsection{Bootstrap}
				Framework HTML, CSS e javaScript che facilita lo sviluppo di front-end.\\
				(\url{http://getbootstrap.com/} - 30/05/2017) 
				\paragraph{Principali vantaggi}
				\begin{itemize}
					\item Librerie semplici e complete;
					\item Ben documentato;
					\item Utilizza la tecnologia Less, che permette a bootstrap di essere al passo con i tempi più velocemente rispetto ad altre librerie.
				\end{itemize}
				\paragraph{Principali svantaggi}
				\begin{itemize}
					\item le pagine create con Bootstrap hanno un aspetto simile tra loro
					\item va inclusa tutta la libreria anche se ne viene usata una parte ridotta
				\end{itemize}
		\subsection{JSON}
			JavaScript Object Notation, è il formato scelto per l'interscambio di dati tra client e
			server; è basato su Javascript, inoltre viene utilizzato in AJAX come alternativa a \gl{XML}. JSON è basato su due strutture:
			\begin{itemize}
				\item Un insieme di coppie nome/valore; in diversi linguaggi questo è realizzato come
				un oggetto, uno struct, una tabella hash, un array associativo, ecc.;
				\item Un elenco ordinato di valori; nella maggior parte dei linguaggi questo è
				realizzato con un array, un vettore, un elenco, ecc.;
			\end{itemize}
			\paragraph{Principali vantaggi}
			\begin{itemize}
					\item Leggero e supportato da tutti i browser;
					\item Formato conciso grazie al suo approccio basato su coppia nome/valore;
					\item Modo completamente automatizzato di serializzare/de-serializzare gli oggetti
					Javascript che richiede poco sviluppo di codice;
					\item API semplice;
					\item Supportato da molti toolkit di AJAX e librerie Javascript.
				\end{itemize}
			\paragraph{Principali svantaggi}
				\begin{itemize}
					\item Nessun supporto a namespace che porta ad una scarsa estensibilità;
					\item Nessun sostegno per la definizione della grammatica formale; di conseguenza
					i contratti di interfaccia sono difficili da comunicare e far rispettare;
					\item Supporta strumenti di sviluppo limitati.
				\end{itemize}
		\subsection{AJAX}
			Asynchronous Javascript And Xml, è la tecnica scelta per lo sviluppo della comunicazione
			dei client verso il server.
			\paragraph{Principali vantaggi}
			\begin{itemize}
					\item Migliora l'esperienza utente, "nascondendo" l'aggiornamento della
					pagina web;
					\item Riduce l'uso di banda e velocizza i caricamenti;
					\item È compatibile con molti linguaggi ed è supportato da molti browser.
				\end{itemize}
			\paragraph{Principali svantaggi}
				\begin{itemize}
					\item Può rendere difficile il \gl{debug} della pagina web sulla quale è utilizzato
					poiché aumenta la dimensione del suo codice sorgente;
					\item Può incrementare il carico sul server web nel caso in cui si utilizzi per
					aggiornare troppo frequentemente una pagina.
				\end{itemize}
		\subsection{RequireJS}
			Loader di moduli e file Javascript ottimizzato per l'uso in-browser ma anche per altri
			ambienti Javascript come Node.js.\\
			(\url{http://requirejs.org/} - 02/05/2017)
			\paragraph{Principali vantaggi}
			\begin{itemize}
					\item Buon supporto alla separazione del codice;
					\item Supporto a plugins;
					\item Può caricare moduli in modo asincrono su richiesta.
				\end{itemize}
			\paragraph{Principali svantaggi}
				\begin{itemize}
					\item Strumento che richiede uno studio molto approfondito in quanto non di apprendimento immediato.
				\end{itemize}
%		\subsection{SQLite}
%			Libreria che implementa un DBMS SQL scelta per creare la base dati del sistema.\\
%			(\url{https://sqlite.org/} - 02/05/2017)
%			\paragraph{Principali vantaggi}
%			\begin{itemize}
%					\item Veloce, leggera e semplice da usare;
%					\item Ha transazioni che godono delle proprietà ACID;
%					\item Supporta basi dati che possono essere anche molto grandi (attualmente fino
%					a 2TB);
%					\item Non ha dipendenze esterne.
%				\end{itemize}
%			\paragraph{Principali svantaggi}
%				\begin{itemize}
%					\item Non supporta alcuni costrutti SQL ed alcune tipologie di trigger;
%					\item Non ha una vera gestione della concorrenza nell'accesso; di conseguenza,
%					le applicazioni che la utilizzano, se necessaria, devono implementarla.
%				\end{itemize}
%		\subsection{MySQL}
%			\gl{DBMS} SQL relazionale scelta per lo sviluppo della base dati del sistema.\\
%			(\url{https://www.mysql.com/} - 02/05/2017) 
%			\paragraph{Principali vantaggi}
%			\begin{itemize}
%					\item Tanto famoso quanto solido per sviluppare basi di dati;
%					\item È progettato con in mente il web, il cloud e \gl{big data};
%					\item Supporta moli di dati che possono essere anche molto grandi senza compromettere le prestazioni.
%				\end{itemize}
%			\paragraph{Principali svantaggi}
%				\begin{itemize}
%					\item Non supporta alcune tipologie di join;
%					\item Non supporta la possibilità di fare sub-query senza rieseguirle ogni volta.
%				\end{itemize}
%		
%		\subsection{Typescript}
%		\subsection{Spring boot}
%		\subsection{StringTemplate}
\end{document}
