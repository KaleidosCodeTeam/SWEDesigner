\documentclass[../AnalisiDeiRequisiti.tex]{subfiles}
\begin{document}
	\section*{Ordine del giorno}
		\begin{enumerate}
			\item Studio e discussione degli errori emersi dalla valutazione della \revisionediprogettazione;
			\item Decidere incontro con il committente;
			\item Stesura delle possibili domande da sottoporre al committente riguardo incomprensioni sulla valutazione della \revisionediprogettazione;
			\item Assegnazione delle componenti software da sviluppare, ai componenti del gruppo incaricati a svolgere le attività di codifica, 
		\end{enumerate}
		Di seguito, il riassunto delle discussioni fatte.
		\begin{enumerate}
		\item Si è discusso ogni errore segnalato nella dalla valutazione della \revisionediprogettazione; per ogni errore è stata studiata una correzione seguendo le valutazioni riportate nel documento; in caso tali valutazioni non siano state comprese appieno, vengono aggiunte alla lista di domande da presentare al committente al fine di chiarimenti;
		\item Si è deciso che è necessario fissare un incontro con il committente allo scopo di chiarire i dubbi del gruppo \kaleidoscode, riguardo le valutazioni non comprese della \revisionediprogettazione. Il \responsabilediprogetto\ provvederà a fissare un incontro entro e non oltre la fine della prossima settimana;
		\item Sono state stilate le domande da sottoporre al committente, riguardanti gli errori emersi dalla \revisionediprogettazione; in particolare: le sezioni di pianificazione e preventivo del \pianodiprogetto; decisioni riguardo le tecnologie adottate; classificazione di UC2.2.8.1 e UC2.2.14; specifica dei test e stile di presentazione delle tendenze per verifiche ripetute, in \pianodiqualifica;
		\item I componenti del gruppo sono stati suddivisi in due gruppi di lavoro: ad uno è stata assegnata la codifica della parte di front-end, mentre al secondo è stata assegnata la codifica della parte di back-end. 
		
		\end{enumerate}

\end{document}