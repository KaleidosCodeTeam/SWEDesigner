\newglossaryentry{UML} {
	name=UML,
	description={Acronimo di Unified Modeling Language (linguaggio di modellazione
	unificato), è un linguaggio di modellazione e specifica basato sul paradigma
	orientato agli oggetti}
}
\newglossaryentry{Java} {
	name=Java,
	description={Linguaggio di programmazione ad alto livello orientato agli oggetti}
}
\newglossaryentry{Javascript} {
	name=Javascript,
	description={Linguaggio di scripting orientato agli oggetti e agli eventi, 
	comunemente utilizzato nella programmazione web lato client}
}
\newglossaryentry{Android} {
	name=Android,
	description={Sistema operativo per dispositivi mobili sviluppato da Google Inc. e basato su
	kernel Linux}
}
\newglossaryentry{HTML} {
	name=HTML,
	description={Acronimo di HyperText Markup Language (linguaggio a marcatori per
	ipertesti), è un linguaggio di markup usato principalmente per creare la
	struttura di documenti ipertestuali}
}
\newglossaryentry{CSS} {
	name=CSS,
	description={Acronimo di Cascading Style Sheets (fogli di stile a cascata), è un linguaggio
	usato per definire la formattazione di documenti HTML}
}
\newglossaryentry{ISO} {
	name=ISO,
	description={Abbreviazione di International Organization for Standardization (organizzazione
	internazionale per la normalizzazione), è la più importante organizzazione a livello mondiale
	per la definizione di norme tecniche}
}
\newglossaryentry{UTF-8} {
	name=UTF-8,
	description={Acronimo di Unicode Transformation Format 8 bit, è una codifica di caratteri
	Unicode in sequenze di lunghezza variabile di byte}
}
\newglossaryentry{Mailing list} {
	name=Mailing list,
	description={Lista di distribuzione o diffusione, consiste in un elenco di indirizzi e-mail per
	l'invio di materiale informativo o pubblicitario}
}
\newglossaryentry{Internet} {
	name=Internet,
	description={Rete ad accesso pubblico che connette vari dispositivi in tutto il mondo.
	Dalla sua nascita rappresenta il principale mezzo di comunicazione di massa, che offre
	all'utente una vasta serie di contenuti potenzialmente informativi e di servizi}
}
\newglossaryentry{Browser} {
	name=Browser,
	description={Applicazione per recupero, presentazione e navigazione di risorse sul web}
}
\newglossaryentry{Git} {
	name=Git,
	description={Software di controllo versione distribuito utilizzabile da interfaccia
	a riga di comando, creato da Linus Torvalds nel 2005}
}
\newglossaryentry{Repository} {
	name=Repository,
	description={Archivio in cui sono racchiusi dati ed informazioni in formato digitale,
	valorizzati e archiviati sulla base di metadati che ne permettano la rapida individuazione.
	Nel caso specifico di repository Git, si tratta dell'archivio contenente tutte le versioni
	del codice caricato sul server}
}
\newglossaryentry{Google Drive} {
	name=Google Drive,
	description={Servizio, in ambiente cloud computing, di memorizzazione e sincronizzazione
	online introdotto da Google il 24 aprile 2012. Il servizio può essere usato via Web, caricando
	e visualizzando i file tramite il web browser, oppure tramite l'applicazione installata su
	computer, che sincronizza automaticamente una cartella locale del file system con quella
	condivisa. Su Google Drive sono presenti anche i documenti creati con Google Documenti}
}
\newglossaryentry{Diagramma di Gantt} {
	name=Diagramma di Gantt,
	description={Strumento di supporto alla gestione dei progetti che permette la rappresentazione
	grafica di un calendario di attività, utile al fine di pianificare, coordinare e tracciare
	specifiche attività in un progetto dando una chiara illustrazione dello stato d'avanzamento
	del progetto rappresentato}
}
\newglossaryentry{Diagramma di PERT} {
	name=Diagramma di PERT,
	description={Diagramma reticolare di PERT (Program Evaluation and Review Technique); descrive
	la sequenza cronologica delle azioni pianificate per il completamento di un progetto nel suo
	complesso. Rappresenta graficamente il piano d'azione}
}
\newglossaryentry{Work Breakdown Structure} {
	name=Work Breakdown Structure,
	description={Work breakdown structure (WBS), detta anche struttura di scomposizione del lavoro
	(traduzione letterale) o struttura analitica di progetto; è l'elenco di tutte le attività di un
	progetto organizzate attraverso un albero gerarchico}
}
\newglossaryentry{Schedule Variance} {
	name=Schedule Variance,
	description={Metrica di progetto standard. Indica se si è in linea, in anticipo o in ritardo
	rispetto alla schedulazione delle attività di progetto pianificate}
}
\newglossaryentry{Budget Variance} {
	name=Budget Variance,
	description={È una metrica di progetto standard. Indica se alla data corrente si è speso di
	più o di meno rispetto a quanto previsto alla data corrente}
}
\newglossaryentry{XML} {
	name=XML,
	description={Metalinguaggio per la definizione di linguaggi di markup, ovvero un linguaggio
	marcatore basato su un meccanismo sintattico che consente di definire e controllare il
	significato degli elementi contenuti in un documento o in un testo}
}
\newglossaryentry{W3C} {
	name=W3C,
	description={World Wide Web Consortium, è un'organizzazione non governativa internazionale che
	ha come scopo quello di sviluppare tutte le potenzialità del World Wide Web}
}
\newglossaryentry{LaTeX} {
	name=LaTeX,
	description={Linguaggio di markup usato per la preparazione di testi basato sul programma di
	composizione tipografica TeX}
}
\newglossaryentry{Range} {
	name=Range,
	description={Campo di variazione; intervallo continuo in cui si può trovare il valore
	ricercato}
}
\newglossaryentry{Stack} {
	name=Stack,
	description={Struttura dati astratta, utilizzata in diversi linguaggi di programmazione,
	in cui i dati possono essere inseriti e acceduti seguendo regole ben definite}
}
\newglossaryentry{Statement} {
	name=Statement,
	description={Il più piccolo elemento indipendente di un linguaggio di programmazione imperativo
	che esprime una qualche azione da effettuare; blocco di istruzioni;
	singolo enunciato o comando}
}
\newglossaryentry{Package} {
	name=Package,
	description={Collezione di classi logicamente correlate.\\
	Indica un unico spazio dei nomi per le classi che contiene. In questo contesto coincide
	con il termine "componente"}
}
\newglossaryentry{Asana} {
	name=Asana,
	description={Applicazione web e mobile progettata per aiutare teams a tenere traccia del
	proprio lavoro}
}
\newglossaryentry{Gulpease} {
	name=Gulpease,
	description={Indice di leggibilità di un testo tarato sulla lingua italiana. Utilizza la
	lunghezza delle parole in lettere anziché in sillabe, semplificandone il calcolo automatico}
}
\newglossaryentry{PDCA} {
	name=PDCA,
	description={Acronimo di Plan-Do-Check-Act (Pianificare-Fare-Verificare-Agire), indica il
	ciclo di Deming: un metodo di gestione iterativo in quattro fasi utilizzato in attività per
	il controllo e il miglioramento continuo dei processi e	dei prodotti}
}
\newglossaryentry{CMM} {
	name=CMM,
	description={Capability Maturity Model (Modello di Maturità delle Capacità Software), è
	un approccio al miglioramento dei processi il cui obiettivo è di aiutare un'organizzazione a
	migliorare le sue prestazioni}
}
\newglossaryentry{Capitolato} {
	name=Capitolato,
	description={Documento tecnico che viene allegato al contratto e che definisce le specifiche
	tecniche delle opere che devono essere portate a termine sulla base di quel che è previsto
	nel contratto stesso}
}
\newglossaryentry{Tomcat} {
	name=Tomcat,
	description={Applicazione server nella forma di contenitore servlet open-source sviluppato
	da Apache Software Foundation. Fornisce una piattaforma software per l'esecuzione di
	applicazioni Web sviluppate in linguaggio Java}
}
\newglossaryentry{GitHub} {
	name=GitHub,
	description={Servizio di hosting per progetti software. Il nome deriva dal fatto che GitHub è
	un servizio sostitutivo del software dell'omonimo strumento di controllo
	versione distribuito, Git}
}
\newglossaryentry{Node.js} {
	name=Node.js,
	description={Piattaforma event-driven per il motore Javascript V8, su piattaforme UNIX-like.
	Sebbene Node.js non sia un framework Javascript, molti dei suoi moduli base sono scritti in
	Javascript, e gli sviluppatori possono scrivere nuovi moduli nello stesso linguaggio}
}
\newglossaryentry{Stack tecnologico} {
	name=Stack tecnologico,
	description={Insieme di strumenti e componenti software necessari e/o richiesti nello sviluppo
	di un prodotto software}
}
\newglossaryentry{Amazon Web Services} {
	name=Amazon Web Services,
	description={Collezione di servizi di cloud computing che compongono la piattaforma
	"on demand" (su richiesta) offerta dall'azienda Amazon. Il suo acronimo è AWS}
}
\newglossaryentry{Bootstrap} {
	name=Bootstrap,
	description={Raccolta di strumenti liberi per la creazione di siti e applicazioni per il Web.
	Essa contiene modelli di progettazione basati su HTML e CSS, sia per la tipografia che per le
	varie componenti dell'interfaccia come moduli, pulsanti e navigazione, così come alcune
	estensioni opzionali di Javascript}
}
\newglossaryentry{Slack} {
	name=Slack,
	description={Piattaforma di messaggistica per team che integra insieme diversi canali di
	comunicazione in un unico servizio}
}
\newglossaryentry{NoSQL} {
	name=NoSQL,
	description={Movimento che promuove sistemi software dove la persistenza dei dati è
	caratterizzata dal fatto di non utilizzare il modello relazionale, di solito usato dai
	database tradizionali. L'espressione NoSQL fa riferimento al linguaggio SQL, che è il più
	comune linguaggio di interrogazione dei dati nei database relazionali, qui preso a simbolo
	dell'intero paradigma relazionale}
}
\newglossaryentry{DynamoDB} {
	name=DynamoDB,
	description={Servizio di database completamente gestito dal proprietario NoSQL offerto da
	Amazon. Componente di Amazon Web Services}
}
\newglossaryentry{MongoDB} {
	name=MongoDB,
	description={Sistema di gestione di basi di dati non relazionale orientato ai documenti}
}
\newglossaryentry{Swift} {
	name=Swift,
	description={Linguaggio di programmazione object-oriented (orientato agli oggetti) per
	sistemi macOS, iOS, watchOS, tvOS e Linux. Concepito per coesistere con il linguaggio
	Objective-C tipico degli sviluppi per i sistemi operativi Apple semplificando
	la scrittura del codice}
}
\newglossaryentry{Alexa} {
	name=Alexa,
	description={Assistente digitale intelligente sviluppato da Amazon}
}
\newglossaryentry{Siri} {
	name=Siri,
	description={Assistente digitale intelligente sviluppato da Apple presente nei dispositivi
	iOS, macOS, watchOS e tvOS, quali iPhone, iPad, Mac, Apple Watch e Apple TV. Permette
	l'esecuzione di comandi vocali sul dispositivo o sulle sue applicazioni, utilizzando il
	linguaggio naturale}
}
\newglossaryentry{Microservizio} {
	name=Microservizio,
	description={Componente software di dimensioni ridotte che hanno un ciclo di sviluppo e di
	rilascio indipendente; contiene tutto quello che serve a garantire una funzionalità
	autocontenuta del sistema}
}
\newglossaryentry{Jolie} {
	name=Jolie,
	description={Linguaggio di programmazione open-source orientato ai servizi sviluppato da
	ItalianaSoftware}
}
\newglossaryentry{Framework} {
	name=Framework,
	description={Intelaiatura o struttura; in informatica e specificatamente nello sviluppo software,
	è un'architettura logica di supporto (spesso un'implementazione logica di un particolare design
	pattern) su cui un software può essere progettato e realizzato, spesso facilitandone lo sviluppo
	da parte del programmatore}
}
\newglossaryentry{API} {
	name=API,
	description={Acronimo di Application Programming Interface (interfaccia di programmazione di
	un'applicazione), è un insieme di procedure disponibili al \programmatore, di solito
	raggruppate a formare un set di strumenti specifici per l'espletamento di un determinato
	compito all'interno di un certo programma. Spesso con tale termine si intendono le librerie
	software disponibili in un certo linguaggio di programmazione}
}
\newglossaryentry{Design pattern} {
	name=Design pattern,
	description={Schema progettuale (schema di progettazione, schema architetturale), è una
	descrizione o modello logico da applicare per la risoluzione di un problema che può presentarsi
	in diverse situazioni durante le fasi di progettazione e sviluppo del software: è un concetto
	che può essere definito "una soluzione progettuale generale ad un problema ricorrente"}
}
\newglossaryentry{CamelCase} {
	name=CamelCase,
	description={Notazione a Cammello, è la pratica di scrivere parole composte o frasi unendo tutte
	le parole tra loro, ma lasciando le loro iniziali maiuscole}
}
\newglossaryentry{K and R} {
	name=K and R,
	description={Stile d'indentazione del codice che prende il nome da Brian Kernighan e Dennis
	Ritchie, autori di ``The C Programming Language'', comunemente usato nella programmazione in
	C e derivati}
}
\newglossaryentry{Stakeholder} {
	name=Stakeholder,
	description={Portatore di interesse, indica un soggetto (o un gruppo di soggetti) influente nei
	confronti di un'iniziativa economica, che sia un'azienda o un progetto (ad esempio clienti,
	fornitori, finanziatori, ecc.)}
}
\newglossaryentry{Chat} {
	name=Chat,
	description={In inglese letteralmente "chiacchierata", viene usato per riferirsi a un'ampia gamma
	di servizi sia telefonici sia via Internet, ovvero quelli che i paesi di lingua inglese
	distinguono di solito con l'espressione on-line chat ("chat in linea") dove il dialogo avviene
	in tempo reale}
}
\newglossaryentry{Template} {
	name=Template,
	description={Traducibile in italiano come "modello", "semi-compilato", "schema", "struttura
	base", "scheletro", indica un documento o programma nel quale, su una struttura
	generica o standard esistono spazi temporaneamente "bianchi" da riempire successivamente}
}
\newglossaryentry{Task} {
	name=Task,
	description={Nel software Asana, è un compito (o una lista di compiti) da portare a termine,
	assegnabile a uno o più utenti membri del gruppo di lavoro}
}
\newglossaryentry{Branch} {
	name=Branch,
	description={Letteralmente "ramo" in italiano, indica, appunto, un ramo di sviluppo di una
	componente (ad esempio software) di un progetto all'interno di uno strumento per il controllo
	di versione}
}
\newglossaryentry{Merge} {
	name=Merge,
	description={All'interno di uno strumento per il controllo di versione, è l'operazione di
	riconciliazione di cambiamenti multipli fatti a una collezione di file (tipicamente è
	l'unione di due branch indipendenti)}
}
\newglossaryentry{Script} {
	name=Script,
	description={In informatica, è un tipo particolare di programma scritto in una particolare
	classe di linguaggi di programmazione, detti linguaggi di scripting}
}
\newglossaryentry{Heroku} {
	name=Heroku,
	description={Platform as a service (PaaS) sul cloud che supporta diversi linguaggi di
	programmazione}
}
\newglossaryentry{Bitbucket} {
	name=Bitbucket,
	description={Servizio di hosting web-based per progetti che usano i sistemi di controllo
	versione Mercurial o Git. È simile a GitHub}
}
\newglossaryentry{Rocket.Chat} {
	name=Rocket.Chat,
	description={Piattaforma di web chat costruita come un clone open-source della piattaforma
	Slack}
}
\newglossaryentry{Front-end} {
	name=Front-end,
	description={Parte di un sistema software che gestisce l'interazione con l'utente o con
	sistemi esterni che producono dati di ingresso (es. interfaccia utente con un form)}
}
\newglossaryentry{Screen-reader} {
	name=Screen-reader,
	description={Lettralmente \textit{lettore dello schermo}, è un'applicazione software che
	identifica ed interpreta il testo mostrato sullo schermo di un computer, presentandolo tramite
	sintesi vocale o attraverso un display braille}
}
\newglossaryentry{DOM} {
	name=DOM,
	description={Document Object Model (modello a oggetti del documento), è una forma di
	rappresentazione dei documenti strutturati come modello orientato agli oggetti}
}
\newglossaryentry{Data binding} {
	name=Data binding,
	description={Tecnica generale che lega le origini dei dati del provider (fornitore) e
	del consumatore e le sincronizza. Nelle interfacce utente (UI), ad esempio, i dati
	e gli oggetti informativi, sono collegati assieme}
}
\newglossaryentry{Mocking} {
	name=Mocking,
	description={Tecnica utilizzata nel test di unità. Consiste nel creare oggetti che
	simulano il comportamento di oggetti reali facenti parte del sistema in fase di test}
}
\newglossaryentry{Multithreading} {
	name=Multithreading,
	description={Diffuso modello di programmazione ed esecuzione che consente l'esistenza
	di più thread (sotto-processo eseguito concorrentemente) all'interno del singolo processo.
	Questi thread condividono le risorse del processo, ma sono in grado di essere eseguiti in
	modo indipendente}
}
\newglossaryentry{Database} {
	name=Database,
	description={Base dati; indica un insieme di dati, omogeneo per contenuti e formato,
	memorizzati in un elaboratore elettronico e interrogabili utilizzando le chiavi di
	accesso previste; collezione di dati strutturati}
}
\newglossaryentry{Debugging} {
	name=Debugging,
	description={Attività che consiste nell'individuazione da parte del programmatore di uno
	o più errori (bug) rilevati nel software a seguito dell'utilizzo di un programma}
}
\newglossaryentry{Plugin} {
	name=Plugin,
	description={Programma non autonomo che interagisce con un altro programma per ampliarne
	o estenderne le funzionalità originarie}
}
\newglossaryentry{DBMS} {
	name=DBMS,
	description={DataBase Management System (sistema di gestione di basi di dati), è un
	sistema software progettato per consentire la creazione, manipolazione e interrogazione
	efficiente di database}
}
\newglossaryentry{Big data} {
	name=Big data,
	description={Termine usato per descrivere una raccolta di dati così estesa in termini di
	volume, velocità e varietà da richiedere tecnologie e metodi analitici specifici per
	l'estrazione di valore}
}
\newglossaryentry{SPA} {
	name=SPA,
	description={Single Page Application (applicazione su singola pagina), è un'applicazione
	o un sito web che può essere usato o consultato su una singola pagina con l'obiettivo di
	fornire un'esperienza utente più fluida e simile alle applicazioni desktop}
}
\newglossaryentry{Back-end} {
	name=Back-end,
	description={Parte di programma con il quale l'utente interagisce indirettamente, di solito
	attraverso l'utilizzo di un'applicazione front-end}
}