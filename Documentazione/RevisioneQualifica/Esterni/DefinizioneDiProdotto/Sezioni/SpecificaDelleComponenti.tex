\documentclass[../DefinizioneDiProdotto.tex]{subfiles}
\begin{document}
		\section{Specifica delle componenti}
			\subsection{SWEDesigner}
				I package contenuti al suo interno sono:
				\begin{itemize}
					\item SWEDesigner::Client;
					\item SWEDesigner::Server.
				\end{itemize}
				Questo package non contiene delle classi.
			\subsection{SWEDesigner::Client}
				% IMMAGINE ARCHITETTURA CLIENT GENERALE
				\begin{figure}[H]\label{fig:ClientSubsystem}
					\centering
					\includegraphics[scale=0.46]{Immagini/DiagrammaArchitettura/ClientSubsystem.png}
					\caption{Architettura del client}
				\end{figure}
				I package contenuti al suo interno sono:
				\begin{itemize}
					\item SWEDesigner::Client::Model;
					\item SWEDesigner::Client::View.
				\end{itemize}
				Questo package non contiene delle classi.
			\subsection{SWEDesigner::Client::Model}
				\hypertarget{SWEDesigner::Client::Model}
				I package contenuti al suo interno sono:
				\begin{itemize}
					\item SWEDesigner::Client::Model::Items.
				\end{itemize}
				Le classi contenute al suo interno verranno elencate qui di seguito.

				\subsubsection{SWEDesigner::Client::Model::DataManager}
				\hypertarget{SWEDesigner::Client::Model::DataManager}
					\begin{itemize}
						\item \textbf{Tipo}: \emph{Classe statica};
						\item \textbf{Descrizione}: Si occupa della persistenza dei dati, in particolare del salvataggio su file system locale del progetto già esistente.\\;
						\item \textbf{Metodi}:
						\begin{itemize}
							\item \emph{newProject(): void} \\
							Dopo aver chiesto conferma all'utente, crea un nuovo progetto sovrascrivendo quello correntemente aperto; \\
							\item \emph{openProject(): void} \\
							Legge un file JSON e ne salva il contenuto in project e nel projectModel come progetto attualmente aperto; \\
							\item \emph{save(fileName: String): void} \\
							Salva i dati del progetto, li converte in formato JSON e avvia la procedura di download in locale del browser; \\
							\textbf{Parametri}:
							\begin{itemize}
								\item \emph{fileName: String}
								Nome del file generato da scaricare; \\
							\end{itemize}
							\item \emph{saveAs(): void}\\
							Estrae la stringa inserita dall'utente nella schermata per il salvataggio con nome e invoca la il metodo per il salvataggio del progetto corrente in un file con il nome desiderato; \\
						\end{itemize}
						\item \textbf{Relazioni con le altre classi}:
						\begin{itemize}
							\item OUT \hyperlink{SWEDesigner::Client::Model::ProjectModel}{\emph{SWEDesigner::Client::Model::ProjectModel}}: si occupa di gestire la parte logica dell'editor;
							\item OUT \hyperlink{SWEDesigner::Client::Model::Project}{\emph{SWEDesigner::Client::Model::Project}}: si occupa di gestire gli elementi contenuti nel diagramma.
						\end{itemize}
					\end{itemize}

				\subsubsection{SWEDesigner::Client::Model::ProjectModel}
				\hypertarget{SWEDesigner::Client::Model::ProjectModel}
					% IMMAGINE ARCHITETTURA MAINMODEL
					\begin{figure}[H]\label{fig:Model}
						\centering
						\includegraphics[scale=0.46]{Immagini/DiagrammaArchitettura/MainModel.png}
						\caption{Architettura di Model}
					\end{figure}

					\begin{itemize}
						\item \textbf{Tipo}: \emph{Classe};
						\item \textbf{Descrizione}: Model del progetto corrente. Si occupa di gestire il graph (joint.dia.Graph) e tutti gli eventi ad esso associati;
						\item \textbf{Padre}: \emph{Backbone.model};
						\item \textbf{Attributi}:
						\begin{itemize}
							\item \emph{currentDiagram: String} \\
							L'id del diagramma correntemente caricato nel graph (null se è il diagramma dei package); \\
							\item \emph{currentDiagramType: String} \\
							Il tipo del diagramma correntemente caricato nel graph ("packageDiagram", "classDiagram" o "bubbleDiagram"); \\
							\item \emph{graph: joint.dia.Graph} \\
							Il model dell'area di disegno associata al paper della \hyperlink{SWEDesigner::Client::View::ProjectView}{\emph{SWEDesigner::Client::View::ProjectView}}; \\
							\item \emph{itemToBeAdded: String} \\
							Salvataggio temporaneo dell'elemento da aggiungere al graph corrente; \\
						\end{itemize}
						\item \textbf{Metodi}:
						\begin{itemize}
							\item \emph{addItem(item: Object): void} \\
							Salva in itemToBeAdded l'elemento passato in input che è un oggetto di Swedesigner::Client::Model::Items; \\
								\textbf{Parametri}:
								\begin{itemize}
									\item \emph{item: Object}
									Elemento del diagramma; \\
								\end{itemize}
							\item \emph{addItemToGraph(): void} \\
							Aggiunge un elemento al grafo del diagramma corrente; \\
							\item \emph{changedPosition(graph: joint.dia.Graph, cell: joint.dia.Cell, newPosition: Object, opt: Object): void} \\
							Gestisce la traslazione di un elemento selezionato nel grafo; \\
								\textbf{Parametri}:
								\begin{itemize}
									\item \emph{graph: joint.dia.Graph}
									Grafo del diagramma corrente; \\
									\item \emph{cell: joint.dia.Cell}
									Elemento correntemente selezionato; \\
									\item \emph{newPosition: Object}
									Posizione attuale dell'oggetto nel grafo; \\
									\item \emph{opt: Object}
									Traslazione dell'oggetto dalla posizione iniziale alla posizione "newPosition"; \\
								\end{itemize}
							\item \emph{deleteCell(): void} \\
							Rimuove un elemento dal grafo eliminando anche gli eventuali diagrammi derivati (classi o bubble); \\
							\item \emph{deleteOperation(): void} \\
							Rimuove un'operazione ed eventualmente anche il diagramma delle bubble associato; \\
							\item \emph{getCellFromId(cellId: String): void} \\
							Ritorna l'elemento del graph avente l'id passato come parametro in input; \\
								\textbf{Parametri}:
								\begin{itemize}
									\item \emph{cellId: String}
									Identificativo dell'elemento nel graph; \\
								\end{itemize}
							\item \emph{graphSwitched(): void} \\
							Genera l'evento "switchgraph"; \\
							\item \emph{initialize(): void} \\
							Inizializzazione del ProjectModel: inizializzazione del graph, del currentDiagramType, degli eventi verificabili; \\
							\item \emph{resizeParent(parent: Object): void} \\
							Esegue il resize di un elemento del diagramma ingrandendolo; \\
								\textbf{Parametri}:
								\begin{itemize}
									\item \emph{parent: Object}
									Elemento del diagramma; \\
								\end{itemize}
							\item \emph{saveCurrentDiagram(): void} \\
							Salva il diagramma correntemente aperto all'interno della struttura definita nella classe Project; \\
							\item \emph{switchInGraph(): void} \\
							Esegue lo switch in profondità al diagramma selezionato svuotando il graph dagli elementi correntemente presenti e caricando gli eventuali nuovi elementi; \\
							\item \emph{switchOutGraph(): void} \\
							Esegue lo switch all'antistante tipo di diagramma selezionato svuotando il graph dagli elementi correntemente presenti e caricando gli eventuali nuovi elementi; \\
						\end{itemize}
						\item \textbf{Relazioni con le altre classi}:
						\begin{itemize}
							\item IN \hyperlink{SWEDesigner::Client::Model::DataManager}{\emph{SWEDesigner::Client::Model::DataManager}}: si occupa della persistenza dei dati, in particolare del salvataggio su file system locale del progetto e del caricamento di un progetto già esistente;
							\hyperlink{SWEDesigner::Client::Model::ToolbarModel}{\emph{SWEDesigner::Client::Model::ToolbarModel}}: È il componente del programma che si occupa di gestire la parte logica della toolbar;
							\item IN \hyperlink{SWEDesigner::Client::Model::RequestHandler}{\emph{SWEDesigner::Client::Model::RequestHandler}}: si occupa di gestire i dati ricevuti dal server;
							\item OUT \hyperlink{SWEDesigner::Client::Model::Project}{\emph{SWEDesigner::Client::Model::Project}}: si occupa di gestire gli elementi contenuti nel diagramma;
							\item OUT \hyperlink{SWEDesigner::Client::Model::Items::Swedesigner}{\emph{SWEDesigner::Client::Model::Items::Swedesigner}}: è il contenitore degli elementi che si possono inserire in un diagramma.
						\end{itemize}
					\end{itemize}

				\subsubsection{SWEDesigner::Client::Model::ToolbarModel}
				\hypertarget{SWEDesigner::Client::Model::ToolbarModel}
					\begin{itemize}
						\item \textbf{Tipo}: \emph{Classe};
						\item \textbf{Descrizione}: È il componente del programma che si occupa di gestire la parte logica della toolbar;
						\item \textbf{Padre}: \emph{Backbone.model};
						\item \textbf{Attributi}:
						\begin{itemize}
							\item \emph{items: Object} \\
							Contiene tutti gli elementi definibili nel diagramma corrente; \\
						\end{itemize}
						\item \textbf{Metodi}:
						\begin{itemize}
							\item \emph{addElement(id: String): void} \\
							Salva lo strumento selezionato interagendo con il \hyperlink{SWEDesigner::Client::Model::ProjectModel}{\emph{SWEDesigner::Client::Model::ProjectModel}}; \\
							\textbf{Parametri}:
							\begin{itemize}
								\item \emph{id: String}
								Identificativo del tipo di strumento/elemento da inserire; \\
							\end{itemize}
							\item \emph{createItems(): void} \\
							Assegna al campo dati "items" il set di strumenti utilizzabili nel diagramma corrente; \\
							\item \emph{currentDiagram(): String} \\
							Ritorna il tipo del diagramma corrente; \\
							\item \emph{initialize(): void} \\
							Inizializzazione del ToolbarModel: chiama il metodo ToolbarModel#createItems;
						\end{itemize}
						\item \textbf{Relazioni con le altre classi}:
						\begin{itemize}
							\item OUT \hyperlink{SWEDesigner::Client::Model::ProjectModel}{\emph{SWEDesigner::Client::Model::ProjectModel}}: si occupa di gestire la parte logica dell'editor;
							\item OUT \hyperlink{SWEDesigner::Client::Model::Items::Swedesigner}{\emph{SWEDesigner::Client::Model::Items::Swedesigner}}: è il contenitore degli elementi che si possono inserire in un diagramma.
						\end{itemize}
					\end{itemize}

				\subsubsection{SWEDesigner::Client::Model::Project}
				\hypertarget{SWEDesigner::Client::Model::Project}
					\begin{itemize}
						\item \textbf{Tipo}: \emph{Classe};
						\item \textbf{Descrizione}: Contenitore di tutti gli elementi del progetto correntemente aperto nella Single Page Application;
						\item \textbf{Padre}: \emph{Backbone.model};
						\item \textbf{Attributi}:
						\begin{itemize}
							\item \emph{classes: Object} \\
							Contiene: classesArray (array contentente diagrammi delle classi; in ogni indice è presente un oggetto {id: idPackagePadre, items: [arrayClassiDelDiagramma]}) e dependenciesArray (array contenente i link del corrispondente diagramma delle classi; in ogni indice è presente un oggetto {id: idPackagePadre, items: [arrayLinkDelDiagramma]}); \\
							\item \emph{operations: Array<Object>} \\
							Contiene un array di oggetti; in ogni indice è presente un oggetto {id: id dell'operazione, items: [arrayBubbleDelDiagramma]}); \\
							\item \emph{packages: Object} \\
							Contiene: packagesArray (array contenente i package item del diagramma dei package) e dependenciesArray (array contenente i link del diagramma dei package); \\
						\end{itemize}
						\item \textbf{Metodi}:
						\begin{itemize}
							\item \emph{deleteClassesDiagramOfPkg(id: String): void} \\
							Elimina il diagramma delle classi associato al package e tutti i diagrammi delle bubble associati alle operazioni delle relative classi; \\
							\textbf{Parametri}:
							\begin{itemize}
								\item \emph{id: String}
								Identificativo del package; \\
							\end{itemize}
							\item \emph{deleteOperationDiagram(id: String): void} \\
							Elimina il diagramma delle bubble associato all'operazione; \\
							\textbf{Parametri}:
							\begin{itemize}
								\item \emph{id: String}
								Identificativo dell'operazione; \\
							\end{itemize}
							\item \emph{getClassIndex(id: String): Number} \\
							Cerca ed eventualmente ritorna l'indice dell'array classesArray del diagramma delle classi associato al package; \\
							\textbf{Parametri}:
							\begin{itemize}
								\item \emph{id: String}
								Identificativo del package; \\
							\end{itemize}
							\item \emph{getOperationIndex(id: String): Number} \\
							Cerca ed eventualmente ritorna l'indice dell'array operations del diagramma delle bubble associato all'operazione; \\
							\textbf{Parametri}:
							\begin{itemize}
								\item \emph{id: String}
								Identificativo dell'operazione; \\
							\end{itemize}
						\end{itemize}
						\item \textbf{Relazioni con le altre classi}:
						\begin{itemize}
							\item IN \hyperlink{SWEDesigner::Client::Model::ProjectModel}{\emph{SWEDesigner::Client::Model::ProjectModel}}: si occupa di gestire la parte logica dell'editor;
							\item IN \hyperlink{SWEDesigner::Client::Model::DataManager}{\emph{SWEDesigner::Client::Model::DataManager}}: si occupa della persistenza dei dati, in particolare del salvataggio su file system locale del progetto e del caricamento di un progetto già esistente;
							\item IN \hyperlink{SWEDesigner::Client::Model::RequestHandler}{\emph{SWEDesigner::Client::Model::RequestHandler}}: si occupa di gestire i dati ricevuti dal server;
						\end{itemize}
					\end{itemize}
				
				\subsubsection{SWEDesigner::Client::Model::RequestHandler}
				\hypertarget{SWEDesigner::Client::Model::RequestHandler}
					\begin{itemize}
						\item \textbf{Tipo}: \emph{Classe};
						\item \textbf{Descrizione}: Si occupa della gestione delle comunicazioni tra client e server (lato client);
						\item \textbf{Padre}: \emph{Backbone.model};
						\item \textbf{Metodi}:
						\begin{itemize}
							\item \emph{caricaJa(): void} \\
							Carica il file json nel server e ne genera il codice Java restituendo il nome della cartella compressa; \\
							\item \emph{caricaJs(): void} \\
							Carica il file json nel server e ne genera il codice Javascript restituendo il nome della cartella compressa; \\
						\end{itemize}
						\item \textbf{Relazioni con le altre classi}:
						\begin{itemize}
							\item OUT \hyperlink{SWEDesigner::Client::Model::ProjectModel}{\emph{SWEDesigner::Client::Model::ProjectModel}}: si occupa di gestire la parte logica dell'editor;
							\item OUT \hyperlink{SWEDesigner::Client::Model::Project}{\emph{SWEDesigner::Client::Model::Project}}: si occupa di gestire gli elementi contenuti nel diagramma.
						\end{itemize}
					\end{itemize}

			\subsection{SWEDesigner::Client::View}
				\hypertarget{SWEDesigner::Client::View}
				Questo package non contiene dei sottopackage.
				Le classi contenute al suo interno verranno elencate qui di seguito.
				\subsubsection{SWEDesigner::Client::View::projectView}
					% IMMAGINE ARCHITETTURA MAINVIEW
					\begin{figure}[H]\label{fig:MainModel}
						\centering
						\includegraphics[scale=0.44]{Immagini/DiagrammaArchitettura/MainView.png}
						\caption{Architettura di projectView}
					\end{figure}

				È il componente del programma che si occupa di gestire l'interfaccia grafica. Nella particolare declinazione MVC adottata da Backbone.js, si occupa anche di gestire gli input dell'utente.\\
					FAN-IN:
					\begin{itemize}
						\item pathView: gestisce l'interfaccia grafica della barra di indirizzo;
						\item editPanelView: gestisce l'interfaccia grafica del pannello di editing.
					\end{itemize}
					FAN-OUT:
					\begin{itemize}
						\item projectModel: si occupa di gestire la parte logica dell'editor;
						\item Paper: gestisce l'interfaccia grafica dell'area dei diagrammi.
					\end{itemize}

				\subsubsection{SWEDesigner::Client::View::titlebarView}
				È il componente del programma che fa la funzione di view per la barra del titolo, dove saranno collocati il menu dell’applicazione e gli shortcut.\\
					FAN-IN:\\
					Non ci sono dipendenze IN.\\
					FAN-OUT:
					\begin{itemize}
						\item requestHandler: gestisce la comunicazione con il server;
						\item dataManager: gestisce la persistenza dei dati su file system.
					\end{itemize}

				\subsubsection{SWEDesigner::Client::View::toolbarView}
				È il componente del programma che fa la funzione di view per la toolbar dove saranno collocati gli strumenti per editare i diagrammi.\\
					FAN-IN:\\
					Non ci sono dipendenze IN.\\
					FAN-OUT:
					\begin{itemize}
						\item projectModel: si occupa di gestire la parte logica dell'editor;
						\item toolbarModel: si occupa di gestire la parte logica della toolbar.
					\end{itemize}

				\subsubsection{SWEDesigner::Client::View::pathView}
				È il componente del programma che fa la funzione di view per il cosiddetto breadcrumb dove viene inserita la posizione corrente.\\
					FAN-IN:
					Non ci sono dipendenze IN. \\
					FAN-OUT:
					\begin{itemize}
						\item projectModel: si occupa di gestire la parte logica dell'editor;
						\item projectView: si occupa di gestire la parte grafica del model.
					\end{itemize}
				
				\subsubsection{SWEDesigner::Client::View::EditPanelView}
				È il componente del programma che fa la funzione di view per le informazioni editabili degli elementi che fanno parte dei diversi diagrammi.\\
					FAN-IN:\\
					Non ci sono dipendenze IN. \\
					FAN-OUT:
					\begin{itemize}
						\item projectView: si occupa di gestire la parte grafica del model.
					\end{itemize}
			
			\subsection{SWEDesigner::Server}
				% IMMAGINE ARCHITETTURA SERVER GENERALE
				\begin{figure}[H]\label{fig:ServerSubsystem}
					\centering
					\includegraphics[scale=0.4]{Immagini/DiagrammaArchitettura/ServerSubsystem.png}
					\caption{Architettura del server}
				\end{figure}
				I package contenuti al suo interno sono:
				\begin{itemize}
					\item SWEDesigner::Server::CodeGenerator;
					\item SWEDesigner::Server::DAORequestHandler;
					\item SWEDesigner::Server::RequestHandler.
				\end{itemize}
				Questo package non contiene delle classi.

			\subsection{SWEDesigner::Server::CodeGenerator}
				I package contenuti al suo interno sono:
				\begin{itemize}
					\item SWEDesigner::Server::CodeGenerator::Builder;
					\item SWEDesigner::Server::CodeGenerator::Coder;
					\item SWEDesigner::Server::CodeGenerator::Parser;
					\item SWEDesigner::Server::CodeGenerator::Zipper.
				\end{itemize}
				Le classi contenute al suo interno verranno elencate qui di seguito.

				\subsubsection{SWEDesigner::Server::CodeGenerator::CodeGenerator}
				\hypertarget{SWEDesigner::Server::CodeGenerator::CodeGenerator}
				E' il componente che rende disponibile la funzionalità per cui, dato un file valido in formato JSON, restituisce un pacchetto in formato .zip contenente i file del codice sorgente che costituiscono il programma rappresentato dal file in input. I file prodotti sono strutturati in packages, come indicato nel file JSON in input.\\
					FAN-IN:
					\begin{itemize}
						\item Server::RequestHandler::Receiver: si occupa di gestire le comunicazioni in entrata dal client.
					\end{itemize}
					FAN-OUT:
					\begin{itemize}
						\item Server::RequestHandler::Sender: si occupa di gestire le comunicazioni in uscita verso il client;
						\item Parser: si occupa di creare un oggetto che contiene le informazioni ricevute in input;
						\item Coder: si occupa della traduzione in codice dell'oggetto ottenuto dal Parser;
						\item Builder: si occupa di organizzare in maniera organica il codice generato dal Coder;
						\item Zipper: si occupa di creare un archivio .zip contenente in codici sorgente precedentemente creati.
					\end{itemize}

			\subsection{SWEDesigner::Server::CodeGenerator::Builder}
				Questo package non contiene dei sottopackage.\\
				Le classi contenute al suo interno verranno elencate qui di seguito.
				\subsubsection{SWEDesigner::Server::CodeGenerator::Builder::Builder}
				È il componente che rende disponibile la funzionalità, dato un file JSON in input che rappresenti un programma, di ottenere un oggetto contenitore del codice sorgente corrispondente al contenuto del file di input. Tale codice è suddiviso e strutturato come indicato nel file di input.\\
					FAN-IN:
					\begin{itemize}
						\item Zipper: si occupa di creare un archivio .zip contenente in codici sorgente precedentemente creati.
					\end{itemize}
					FAN-OUT:
					\begin{itemize}
						\item Coder: componente che funge da interfaccia alle operazioni di codifica di una stringa permettendo quindi di trasformare le informazioni del file in formato JSON in codice sorgente.
					\end{itemize}

			\subsection{SWEDesigner::Server::CodeGenerator::Coder}
				% IMMAGINE ARCHITETTURA CODER
				\begin{figure}[H]\label{fig:Coder}
					\centering
					\includegraphics[scale=0.46]{Immagini/DiagrammaArchitettura/Coder.png}
					\caption{Architettura di Coder}
				\end{figure}

				Questo package non contiene dei sottopackage.\\
				Le classi contenute al suo interno verranno elencate qui di seguito.
				\subsubsection{SWEDesigner::Server::CodeGenerator::Coder::Coder}
				Componente che funge da interfaccia alle operazioni di codifica di una stringa, in formato JSON che rappresenta un programma valido; tali operazioni permettono di ottenere un oggetto contenente il codice sorgente, in Java o Javascript, corrispondente alla stringa in input.\\
					FAN-IN:
					\begin{itemize}
						\item JavaCoder: si occupa di trasformare un oggetto JSON ricevuto in input in un oggetto contenente il codice sorgente scritto in java;
						\item JavaScriptCoder: si occupa di trasformare un oggetto JSON ricevuto in input in un oggetto contenente il codice sorgente scritto in javascript.
					\end{itemize}
					FAN-OUT:
					\begin{itemize}
						\item CodedProg: componente che contiene il codice prodotto dal Coder;
						\item CoderElement: componente astratto che offre la funzionalità che permette di associare ad ogni stringa contenuta nel file JSON il corrispondente codice sorgente.
					\end{itemize}

				\subsubsection{SWEDesigner::Server::CodeGenerator::Coder::JavaCoder}
				È il componente che rende disponibile la funzionalità, dato un oggetto in input che rappresenta un file JSON parsificato, di ottenere un oggetto contenente il codice sorgente, in linguaggio Java, corrispondente all'oggetto in input.\\
					Non ci sono dipendenze IN.\\
					FAN-OUT:
					\begin{itemize}
						\item Coder: componente che funge da interfaccia alle operazioni di codifica di una stringa permettendo quindi di trasformare le informazioni del file in formato JSON in codice sorgente.
					\end{itemize}

				\subsubsection{SWEDesigner::Server::CodeGenerator::Coder::JavascriptCoder}
				È il componente che rende disponibile la funzionalità, dato un oggetto in input che rappresenta un file JSON parsificato, di ottenere un oggetto contenente il codice sorgente, in linguaggio Javascript, corrispondente all'oggetto in input.\\
					Non ci sono dipendenze IN.\\
					FAN-OUT:
					\begin{itemize}
						\item Coder: componente che funge da interfaccia alle operazioni di codifica di una stringa permettendo quindi di trasformare le informazioni del file in formato JSON in codice sorgente.
					\end{itemize}

				\subsubsection{SWEDesigner::Server::CodeGenerator::Coder::CoderClass}
				È il componente che mette a disposizione la funzionalità, data una stringa in input in formato JSON che rappresenta una classe valida, di ottenere il corrispondente codice sorgente di tale classe.\\
					Non ci sono dipendenze IN.\\
					FAN-OUT:
					\begin{itemize}
						\item CoderElement: componente astratto che offre la funzionalità che permette di associare ad ogni stringa contenuta nel file JSON il corrispondente codice sorgente.
					\end{itemize}

				\subsubsection{SWEDesigner::Server::CodeGenerator::Coder::CoderOperation}
				È il componente che mette a disposizione la funzionalità, data una stringa in input in formato JSON che rappresenta un'operazione valida, di ottenere il corrispondente codice sorgente di tale operazione.\\
					Non ci sono dipendenze IN.\\
					FAN-OUT:
					\begin{itemize}
						\item CoderElement: componente astratto che offre la funzionalità che permette di associare ad ogni stringa contenuta nel file JSON il corrispondente codice sorgente.
					\end{itemize}

				\subsubsection{SWEDesigner::Server::CodeGenerator::Coder::CoderParameter}
				È il componente che mette a disposizione la funzionalità, data una stringa in input in formato JSON che rappresenta un parametro di una lista valido, di ottenere il corrispondente codice sorgente di tale parametro. È possibile scegliere fra la codifica in Java o Javascript.\\
					Non ci sono dipendenze IN.\\
					FAN-OUT:
					\begin{itemize}
						\item CoderElement: componente astratto che offre la funzionalità che permette di associare ad ogni stringa contenuta nel file JSON il corrispondente codice sorgente.
					\end{itemize}

				\subsubsection{SWEDesigner::Server::CodeGenerator::Coder::CoderAttribute}
				È il componente che mette a disposizione la funzionalità, data una stringa in input in formato JSON che rappresenta un attributo valido, di ottenere il corrispondente codice sorgente di tale attributo. È possibile scegliere fra la codifica in Java o Javascript.\\
				Non ci sono dipendenze IN.\\
					FAN-OUT:
					\begin{itemize}
						\item CoderElement: componente astratto che offre la funzionalità che permette di associare ad ogni stringa contenuta nel file JSON il corrispondente codice sorgente.
					\end{itemize}

				\subsubsection{SWEDesigner::Server::CodeGenerator::Coder::CoderActivity}
				È il componente che mette a disposizione la funzionalità, data una stringa in input in formato JSON che rappresenta un diagramma delle attività valido, di ottenere il corrispondente codice sorgente di tale attività. È possibile scegliere fra la codifica in Java o Javascript.\\
					Non ci sono dipendenze IN.\\
					FAN-OUT:
					\begin{itemize}
						\item CoderElement: componente astratto che offre la funzionalità che permette di associare ad ogni stringa contenuta nel file JSON il corrispondente codice sorgente;
						\item DAO: si occupa di gestire il database delle bubble.
					\end{itemize}

				\subsubsection{SWEDesigner::Server::CodeGenerator::Coder::CodedProg}
				È il componente che contiene il codice sorgente prodotto dal Coder.\\
					FAN-IN:
					\begin{itemize}
						\item Coder: componente che funge da interfaccia alle operazioni di codifica di una stringa permettendo quindi di trasformare le informazioni del file in formato JSON in codice sorgente.
					\end{itemize}
					Non ci sono dipendenze OUT.
				
				\subsubsection{SWEDesigner::Server::CodeGenerator::Coder::CoderElement}
				Componente astratta che offre la funzionalità di ottenere, data una stringa in input in formato JSON che rappresenta un elemento di classe valido, il corrispondente codice sorgente, in Java o Javascript.\\
					FAN-IN:
					\begin{itemize}
						\item Coder: componente che funge da interfaccia alle operazioni di codifica di una stringa permettendo quindi di trasformare le informazioni del file in formato JSON in codice sorgente;
						\item CoderClass: componente che permette data una stringa in input in formato JSON che rappresenta un diagramma delle classi valido, di ottenere il corrispondente codice sorgente di tale classe;
						\item CoderOperations: componente che permette data una stringa in input in formato JSON che rappresenta un'operazione valida, di ottenere il corrispondente codice sorgente di tale operazione;
						\item CoderAttributes: componente che permette data una stringa in input in formato JSON che rappresenta un attributo valido, di ottenere il corrispondente codice sorgente di tale attributo;
						\item CoderActivity: componente che permette data una stringa in input in formato JSON che rappresenta un diagramma delle attività valido, di ottenere il corrispondente codice sorgente di tale attività;
						\item CoderParameter: componente che permette data una stringa in input in formato JSON che rappresenta un parametro valido, di ottenere il corrispondente codice sorgente di tale parametro.
					\end{itemize}
					Non ci sono dipendenze OUT.

			\subsection{SWEDesigner::Server::CodeGenerator::Parser}
				Questo package non contiene dei sottopackage.
				Le classi contenute al suo interno verranno elencate qui di seguito.
				\subsubsection{SWEDesigner::Server::CodeGenerator::Parser::Parser}
				È il componente che rende disponibile la funzionalità, dato un file JSON valido in input, di ottenere un oggetto contenente le informazioni che costituiscono il file in input.\\
					FAN-IN:
					\begin{itemize}
						\item CodeGenerator: si occupa di restituire in output un archivio zip contenente i codici sorgenti generati a partire dal file JSON ricevuto in input;
						\item Coder: componente che funge da interfaccia alle operazioni di codifica di una stringa permettendo quindi di trasformare le informazioni del file in formato JSON in codice sorgente.
					\end{itemize}
					Non ci sono dipendenze OUT.

			\subsection{SWEDesigner::Server::CodeGenerator::Zipper}
				Questo package non contiene dei sottopackage.\\
				Le classi contenute al suo interno verranno elencate qui di seguito.
				\subsubsection{SWEDesigner::Server::CodeGenerator::Zipper::Zipper}
				E' il componente che rende disponibile la funzionalità per cui, dato un file valido in formato JSON, restituisce un pacchetto in formato .zip contenente i file del codice sorgente che costituiscono il programma rappresentato dal file in input. I file prodotti sono strutturati in packages, come indicato nel file JSON in input.\\
					FAN-IN:
					\begin{itemize}
						\item CodeGenerator: si occupa di restituire in output un archivio zip contenente i codici sorgenti generati a partire dal file JSON ricevuto in input.
					\end{itemize}
					FAN-OUT:
					\begin{itemize}
						\item Builder: componente che si occupa di creare un oggetto contenitore con il codice sorgente, partendo dalle informazioni prese dal file JSON ricevuto in input che rappresenta un programma.
					\end{itemize}

				\subsubsection{SWEDesigner::Server::DAO}
				\hypertarget{SWEDesigner::Server::DAO}
				Questa classe si occupa di gestire il database delle bubble.\\
					FAN-IN:
					\begin{itemize}
						\item Coder: componente che funge da interfaccia alle operazioni di codifica di una stringa permettendo quindi di trasformare le informazioni del file in formato JSON in codice sorgente.
					\end{itemize}
					Non ci sono dipendenze OUT.

			\subsection{SWEDesigner::Server::RequestHandler}
				\hypertarget{SWEDesigner::Server::RequestHandler}
				Questo package non contiene dei sottopackage.\\
				Le classi contenute al suo interno verranno elencate qui di seguito.
				\subsubsection{SWEDesigner::Server::RequestHandler::Sender}
				Si occupa di gestire le comunicazioni in uscita verso il client.\\
					FAN-IN:
					\begin{itemize}
						\item CodeGenerator: si occupa di restituire in output un archivio zip contenente i codici sorgenti generati a partire dal file JSON ricevuto in input.
					\end{itemize}
					FAN-OUT:
					\begin{itemize}
						\item Client::Model::RequestHandler::Receiver: si occupa di gestire le comunicazioni in entrata dal server.
					\end{itemize}
					
				\subsubsection{SWEDesigner::Server::RequestHandler::Receiver}
				Si occupa di gestire le comunicazioni in entrata dal client.\\
					FAN-IN:
					\begin{itemize}
						\item Client::Model::RequestHandler::Sender: si occupa di gestire le comunicazioni in uscita verso il server.
					\end{itemize}
					FAN-OUT:
					\begin{itemize}
						\item CodeGenerator: si occupa di restituire in output un archivio zip contenente i codici sorgenti generati a partire dal file JSON ricevuto in input.
					\end{itemize}
\end{document}
