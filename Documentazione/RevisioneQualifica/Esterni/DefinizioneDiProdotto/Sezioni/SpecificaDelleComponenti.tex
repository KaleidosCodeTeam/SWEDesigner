\documentclass[../DefinizioneDiProdotto.tex]{subfiles}
\begin{document}
		\section{Specifica delle componenti}
			\subsection{SWEDesigner}
				I package contenuti al suo interno sono:
				\begin{itemize}
					\item SWEDesigner::Client;
					\item SWEDesigner::Server.
				\end{itemize}
				Questo package non contiene delle classi.
			\subsection{SWEDesigner::Client}
				% IMMAGINE ARCHITETTURA CLIENT GENERALE
				\begin{figure}[H]\label{fig:ClientSubsystem}
					\centering
					\includegraphics[scale=0.46]{Immagini/DiagrammaArchitettura/ClientSubsystem.png}
					\caption{Architettura del client}
				\end{figure}
				I package contenuti al suo interno sono:
				\begin{itemize}
					\item SWEDesigner::Client::Model;
					\item SWEDesigner::Client::View.
				\end{itemize}
				Questo package non contiene delle classi.
			\subsection{SWEDesigner::Client::Model}
				\hypertarget{SWEDesigner::Client::Model}
				I package contenuti al suo interno sono:
				\begin{itemize}
					\item SWEDesigner::Client::Model::RequestHandler.
				\end{itemize}
				Le classi contenute al suo interno verranno elencate qui di seguito.
				\subsubsection{SWEDesigner::Client::Model::Command}
				È l'interfaccia che rappresenta un generico comando impartito dai moduli View ai Model.\\
					FAN-IN:
					\begin{itemize}
						\item ConcreteCommand: implementa l'interfaccia Command per la rappresentazione concreta dei singoli comandi impartiti dai moduli View ai Model;
						\item View: il componente del programma che si occupa di gestire l'interfaccia grafica;
						\item State: gestisce la cronologia delle operazioni svolte permettendo le operazioni di unDo e reDo.
					\end{itemize}
					Non ci sono dipendenze OUT.

				\subsubsection{SWEDesigner::Client::Model::ConcreteCommand}
				Implementa l'interfaccia Command per la rappresentazione concreta dei singoli comandi impartiti dai moduli View ai Model.\\
					Non ci sono dipendenze IN.\\
					FAN-OUT:
					\begin{itemize}
						\item Command: è l'interfaccia che rappresenta un generico comando impartito dai moduli View ai Model;
						\item projectModel: si occupa di gestire la parte logica dell'editor.
					\end{itemize}

				\subsubsection{SWEDesigner::Client::Model::State}
				\hypertarget{SWEDesigner::Client::Model::State}
				Gestisce la cronologia delle operazioni svolte permettendo le operazioni di unDo e reDo.\\
					FAN-IN:
					\begin{itemize}
						\item View: il componente del programma che si occupa di gestire l'interfaccia grafica.
					\end{itemize}
					FAN-OUT:
					\begin{itemize}
						\item Command: è l'interfaccia che rappresenta un generico comando impartito dai moduli View ai Model.
					\end{itemize}

				\subsubsection{SWEDesigner::Client::Model::dataManager}
				\hypertarget{SWEDesigner::Client::Model::dataManager}
				Si occupa della persistenza dei dati, in particolare del salvataggio su file system locale del progetto già esistente.\\
					FAN-IN:
					\begin{itemize}
						\item View: il componente del programma che si occupa di gestire l'interfaccia grafica.
					\end{itemize}
					FAN-OUT:
					\begin{itemize}
						\item projectModel: si occupa di gestire la parte logica dell'editor;
						\item project: si occupa di gestire gli elementi contenuti nel diagramma.
					\end{itemize}

				\subsubsection{SWEDesigner::Client::Model::projectModel}
					% IMMAGINE ARCHITETTURA MAINMODEL
					\begin{figure}[H]\label{fig:Model}
						\centering
						\includegraphics[scale=0.46]{Immagini/DiagrammaArchitettura/MainModel.png}
						\caption{Architettura di Model}
					\end{figure}

				È il componente del programma che si occupa di gestire la parte logica dell’editor.\\
					FAN-IN:
					\begin{itemize}
						\item ConcreteCommand: rappresenta i comandi inviati dalle View ed eseguiti poi da Model;
						\item dataManager: si occupa della persistenza dei dati, in particolare del salvataggio su file system locale del progetto e del caricamento di un progetto già esistente;
						\item View: invoca il metodo execCommand;
						\item Client::RequestHandler::Receiver: si occupa di gestire i dati ricevuti dal server.
					\end{itemize}
					FAN-OUT:
					\begin{itemize}
						\item Client::RequestHandler::Sender: si occupa di gestire le comunicazioni in uscita verso il server.
					\end{itemize}

				\subsubsection{SWEDesigner::Client::Model::toolbarModel}
				\hypertarget{SWEDesigner::Client::Model::toolbarModel}{}
				È il componente del programma che si occupa di gestire la parte logica della toolbar.\\
					FAN-IN:\\
					Non ci sono dipendenze IN. \\
					FAN-OUT:
					\begin{itemize}
						\item projectModel: si occupa di gestire la parte logica dell'editor;
						\item swedesignerItems: definisce il comportamento degli oggetti contenuti nel diagramma.
					\end{itemize}

				\subsubsection{SWEDesigner::Client::Model::project}
				si occupa di gestire gli elementi contenuti nel diagramma.\\
					FAN-IN:
					\begin{itemize}
						\item projectModel: si occupa di gestire la parte logica dell'editor;
						\item dataManager: Si occupa della persistenza dei dati, in particolare del salvataggio su file system locale del progetto già esistente;
						\item requestHandler: Si occupa di gestire le comunicazioni con il server.
					\end{itemize}
					FAN-OUT:\\
					Non ci sono dipendenze OUT. \\
				
			\subsection{SWEDesigner::Client::Model::RequestHandler}
				\hypertarget{SWEDesigner::Client::Model::RequestHandler}
				Questo package non contiene dei sottopackage.\\
				Le classi contenute al suo interno verranno elencate qui di seguito.
				\subsubsection{SWEDesigner::Client::Model::RequestHandler::Sender}
				Si occupa di gestire le comunicazioni in uscita verso il server.\\
					FAN-IN:
					\begin{itemize}
						\item View: ne invoca i metodi.
					\end{itemize}
					FAN-OUT:
					\begin{itemize}
						\item projectModel: si occupa di gestire la parte logica dell'editor;
						\item project: si occupa di gestire gli elementi contenuti nel diagramma;
						\item Server::RequestHandler::Receiver: si occupa di gestire le comunicazioni in entrata dal Client.
					\end{itemize}

				\subsubsection{SWEDesigner::Client::Model::RequestHandler::Receiver}
				Si occupa di gestire le comunicazioni in entrata dal server.\\
					FAN-IN:
					\begin{itemize}
						\item Server::RequestHandler::Sender: si occupa di gestire le comunicazioni in uscita verso il Client.
					\end{itemize}
					FAN-OUT:
					\begin{itemize}
						\item projectModel: si occupa di gestire la parte logica dell'editor;
						\item project: si occupa di gestire gli elementi contenuti nel diagramma.
					\end{itemize}
		
			%inizio parte Giulio
			
			\subsection{SWEDesigner::Client::Model::Items}
			\hypertarget{SWEDesigner::Client::Model::Items}{}
			Questo package non contiene dei sottopackage. Le classi contenute al suo interno verranno
			elencate qui di seguito.
			
			\subsubsection{SWEDesigner::Client::Model::Items::Swedesigner}
			\hypertarget{SWEDesigner::Client::Model::Items::Swedesigner}{}
			\begin{itemize}
				\item \textbf{Tipo}: \emph{Class};
				\item \textbf{Descrizione}: Collezione di oggetti che si possono inserire all'interno di un diagramma suddivisi per tipo di diagramma;
				\item \textbf{Relazioni con le altre classi}:
				\begin{itemize}
					\item IN SWEDesigner::Client::Model::ProjectModel
					\item IN SWEDesigner::Client::Model::toolbarModel
				\end{itemize}
			\end{itemize}
			
			%--------------------------------------PACKAGE DIAGRAM--------------------------------------
			
			\subsubsection{SWEDesigner::Client::Model::Items::Swedesigner.model.packageDiagram.items.Base}
			\hypertarget{SWEDesigner::Client::Model::Items::Swedesigner.model.packageDiagram.items.Base}{}
			\begin{itemize}
				\item \textbf{Tipo}: \emph{Class};
				\item \textbf{Descrizione}: Elemento base generico per diagramma dei package UML;
				\item \textbf{Padre}: \emph{joint.shapes.basic.Generic};
				\item \textbf{Attributi}:
				\begin{itemize}
					\item \emph{toolMarkup: string}\\
					Markup HTML per la rappresentazione grafica;
					\item \emph{defaults: Object}\\
					Attributi di default per l'oggetto;
				\end{itemize}
				\item \textbf{Metodi}:
				\begin{itemize}
					\item \emph{initialize(): void}\\
					Inizializzazione di Base: imposta evento al verificarsi del cambio dei valori e chiama il metodo per la renderizzazione dell'item;
					\item \emph{updateRectangles(): void}\\
					Render dell'item;
					\item \emph{getValues(): Object}\\
					Ritorna i valori dell'item;
					\item \emph{setToValue(value: Object, path: string): void}\\
					Imposta "values.path" a "value";
					Parametri:
					\begin{itemize}
						\item \emph{value: Object} \\
						Valore da assegnare;
						\item \emph{path: string} \\
						Percorso al membro;
					\end{itemize}
				\end{itemize}
			\end{itemize}
			
			
			\subsubsection{SWEDesigner::Client::Model::Items::Swedesigner.model.packageDiagram.items.BaseView}
			\hypertarget{SWEDesigner::Client::Model::Items::Swedesigner.model.packageDiagram.items.BaseView}{}
			\begin{itemize}
				\item \textbf{Tipo}: \emph{Class};
				\item \textbf{Descrizione}: View per item "Base";
				\item \textbf{Padre}: \emph{joint.dia.ElementView};
				\item \textbf{Metodi}:
				\begin{itemize}
					\item \emph{initialize(): void}\\
					Inizializzazione di BaseView: chiama il metodo "initialize" della classe "Base" e imposta un evento alla reazione del model chiamando sequenzialmente i metodi "update" e "resize";
					\item \emph{render(): Object}\\
					Render dell'item;
					\item \emph{renderTools(): Object}\\
					Assistenza al metodo "render" per la renderizzazione dell'item;
				\end{itemize}
			\end{itemize}
			
			\subsubsection{SWEDesigner::Client::Model::Items::Swedesigner.model.packageDiagram.items.Package}
			\hypertarget{SWEDesigner::Client::Model::Items::Swedesigner.model.packageDiagram.items.Package}{}
			\begin{itemize}
				\item \textbf{Tipo}: \emph{Class};
				\item \textbf{Descrizione}: Elemento package per diagramma dei package UML;
				\item \textbf{Padre}: \emph{Swedesigner.model.packageDiagram.items.Base};
				\item \textbf{Attributi}:
				\begin{itemize}
					\item \emph{markup: string}\\
					Markup HTML per la rappresentazione grafica;
					\item \emph{defaults: Objects}\\
					Attributi di default per l'oggetto Package (tipo, posizione, dimensione, attributi CSS, stato e contenuto dell'oggetto);
				\end{itemize}
				\item \textbf{Metodi}:
				\begin{itemize}
					\item \emph{initialize(): void}\\
					Inizializzazione di Package: chiama il metodo "initialize" della classe base e crea l'istanza di Diagram associata al diagramma delle classi relativo al package;
					\item \emph{getPackageName(): string}\\
					Ritorna il nome del Package;
					\item \emph{updateRectangles(): void}\\
					Render del package;
				\end{itemize}
			\end{itemize}
			
			\subsubsection{SWEDesigner::Client::Model::Items::Swedesigner.model.packageDiagram.items.PkgComment}
			\hypertarget{SWEDesigner::Client::Model::Items::Swedesigner.model.packageDiagram.items.PkgComment}{}
			\begin{itemize}
				\item \textbf{Tipo}: \emph{Class};
				\item \textbf{Descrizione}: Commento per diagramma dei package UML;
				\item \textbf{Padre}: \emph{joint.shapes.basic.TextBlock};
				\item \textbf{Attributi}:
				\begin{itemize}
					\item \emph{toolMarkup: string}\\ 
					Markup HTML per la rappresentazione grafica;
					\item \emph{defaults: Objects}\\
					Attributi di default per l'oggetto PkgComment;
				\end{itemize}
				\item \textbf{Metodi}:
				\begin{itemize}
					\item \emph{initialize(): void}\\
					Inizializzazione di PkgComment;
					\item \emph{getPackageName(): string}\\
					Ritorna il nome del Package;
					\item \emph{getValues(): Objects}\\
					Ritorna i valori dell'item PkgComment;
					\item \emph{setToValue(value: Object, path: string): void}\\
					Imposta "values.path" a "value";
					Parametri:
					\begin{itemize}
						\item \emph{value: Object} \\
						Valore da assegnare;
						\item \emph{path: string} \\
						Percorso al membro;
					\end{itemize}
					\item \emph{updateContent(): void}\\
					Aggiorna il contenuto dell'item PkgComment;
				\end{itemize}
			\end{itemize}
			
			\subsubsection{SWEDesigner::Client::Model::Items::Swedesigner.model.packageDiagram.items.PkgCommentView}
			\hypertarget{SWEDesigner::Client::Model::Items::Swedesigner.model.packageDiagram.items.PkgCommentView}{}
			\begin{itemize}
				\item \textbf{Tipo}: \emph{Class};
				\item \textbf{Descrizione}: View per oggetto "PkgComment";
				\item \textbf{Padre}: \emph{joint.shapes.basic.TextBlockView};
				\item \textbf{Metodi}:
				\begin{itemize}
					\item \emph{initialize(): void}\\
					Inizializzazione di PkgCommentView;
					\item \emph{render(): Object}\\
					Render dell'item PkgCommentView;
					\item \emph{renderTools(): Objects}\\
					Assistenza al metodo "render" per la renderizzazione dell'item;
				\end{itemize}
			\end{itemize}
			
			\subsubsection{SWEDesigner::Client::Model::Items::Swedesigner.model.packageDiagram.items.packageDiagramLink}
			\hypertarget{SWEDesigner::Client::Model::Items::Swedesigner.model.packageDiagram.items.packageDiagramLink}{}
			\begin{itemize}
				\item \textbf{Tipo}: \emph{Class};
				\item \textbf{Descrizione}: Collegamento tra due componenti di un diagramma dei package UML;
				\item \textbf{Padre}: \emph{joint.dia.Link};
				\item \textbf{Attributi}:
				\begin{itemize}
					\item \emph{defaults: Objects}\\
					Attributi di default per l'oggetto;
				\end{itemize}
				\item \textbf{Metodi}:
				\begin{itemize}
					\item \emph{initialize(): void}\\
					Inizializzazione di PackageDiagramLink;
					\item \emph{getValues(): Object}\\
					Ritorna i valori del collegamento;
					\item \emph{setToValue(value: Object, path: string): void}\\
					Imposta "values.path" a "value";
					Parametri:
					\begin{itemize}
						\item \emph{value: Object} \\
						Valore da assegnare;
						\item \emph{path: string} \\
						Percorso al membro;
					\end{itemize}
				\end{itemize}
			\end{itemize}
			
			\subsubsection{SWEDesigner::Client::Model::Items::Swedesigner.model.packageDiagram.items.PkgCommentLink}
			\hypertarget{SWEDesigner::Client::Model::Items::Swedesigner.model.packageDiagram.items.PkgCommentLink}{}
			\begin{itemize}
				\item \textbf{Tipo}: \emph{Class};
				\item \textbf{Descrizione}: Link tra un commento e un componente UML del diagramma dei package;
				\item \textbf{Padre}: \emph{Swedesigner.model.packageDiagram.items.packageDiagramLink};
				\item \textbf{Attributi}:
				\begin{itemize}
					\item \emph{defaults: Objects}\\
					Attributi di default per l'oggetto;
				\end{itemize}
			\end{itemize}
			
			\subsubsection{SWEDesigner::Client::Model::Items::Swedesigner.model.packageDiagram.items.PkgDependency}
			\hypertarget{SWEDesigner::Client::Model::Items::Swedesigner.model.packageDiagram.items.PkgDependency}{}
			\begin{itemize}
				\item \textbf{Tipo}: \emph{Class};
				\item \textbf{Descrizione}: Dipendenza tra due package UML del diagramma dei package;
				\item \textbf{Padre}: \emph{Swedesigner.model.packageDiagram.items.packageDiagramLink};
				\item \textbf{Attributi}:
				\begin{itemize}
					\item \emph{defaults: Objects}\\
					Attributi di default per l'oggetto;
				\end{itemize}
			\end{itemize}
			
			%--------------------------------CLASS DIAGRAM-----------------------------------
			
			\subsubsection{SWEDesigner::Client::Model::Items::Swedesigner.model.classDiagram.items.Base}
			\hypertarget{SWEDesigner::Client::Model::Items::Swedesigner.model.classDiagram.items.Base}{}
			\begin{itemize}
				\item \textbf{Tipo}: \emph{Class};
				\item \textbf{Descrizione}: Elemento base generico per diagramma dei package UML;
				\item \textbf{Padre}: \emph{joint.shapes.basic.Generic};
				\item \textbf{Attributi}:
				\begin{itemize}
					\item \emph{markup: string}\\
					Markup HTML per la rappresentazione grafica;
					\item \emph{defaults: Object}\\
					Attributi di default per l'oggetto;
				\end{itemize}
				\item \textbf{Metodi}:
				\begin{itemize}
					\item \emph{initialize(): void}\\
					Inizializzazione di Base: imposta evento al verificarsi del cambio dei valori e chiama il metodo per la renderizzazione dell'item;
					\item \emph{getValues(): Object}\\
					Ritorna i valori dell'item;
					\item \emph{updateRectangles(): void}\\
					Render dell'item;	
					\item \emph{setToValue(value: Object, path: string): void}\\
					Imposta "values.path" a "value";
					Parametri:
					\begin{itemize}
						\item \emph{value: Object} \\
						Valore da assegnare;
						\item \emph{path: string} \\
						Percorso al membro;
					\end{itemize}
					\item \emph{executeMethod(met: function): void}\\
					Esegue il metodo avente il nome passato in input;
					Parametri:
					\begin{itemize}
						\item \emph{met: function} \\
						Metodo da essere eseguito;
					\end{itemize}
				\end{itemize}
			\end{itemize}
			
			\subsubsection{SWEDesigner::Client::Model::Items::Swedesigner.model.classDiagram.items.BaseView}
			\hypertarget{SWEDesigner::Client::Model::Items::Swedesigner.model.classDiagram.items.BaseView}{}
			\begin{itemize}
				\item \textbf{Tipo}: \emph{Class};
				\item \textbf{Descrizione}: View per oggetto "Base";
				\item \textbf{Padre}: \emph{joint.dia.ElementView};
				\item \textbf{Metodi}:
				\begin{itemize}
					\item \emph{initialize(): void}\\
					Inizializzazione di BaseView: chiama il metodo "initialize" della classe base e imposta un evento alla reazione del model chiamando sequenzialmente i metodi "update" e "resize";
					\item \emph{render(): Object}\\
					Render dell'item;
					\item \emph{renderTools(): Object}\\
					Assistenza al metodo "render" per la renderizzazione dell'item;
				\end{itemize}
			\end{itemize}
			
			\subsubsection{SWEDesigner::Client::Model::Items::Swedesigner.model.classDiagram.items.Class}
			\hypertarget{SWEDesigner::Client::Model::Items::Swedesigner.model.classDiagram.items.Class}{}
			\begin{itemize}
				\item \textbf{Tipo}: \emph{Class};
				\item \textbf{Descrizione}: Elemento classe per diagramma delle classi UML;
				\item \textbf{Padre}: \emph{Swedesigner.model.classDiagram.items.Base};
				\item \textbf{Attributi}:
				\begin{itemize}
					\item \emph{markup: string}\\
					Markup HTML per la rappresentazione grafica;
					\item \emph{defaults: Objects}\\
					Attributi di default per l'oggetto Class (tipo, posizione, dimensione, attributi CSS, stato e contenuto dell'oggetto);
				\end{itemize}
				\item \textbf{Metodi}:
				\begin{itemize}
					\item \emph{initialize(): void}\\
					Inizializzazione di Class: chiama il metodo "initialize" della classe base;
					\item \emph{updateRectangles(): void}\\
					Render della classe;
					\item \emph{addAttribute(): void}\\
					Aggiunge un nuovo attributo alla classe;
					\item \emph{addOperation(): void}\\
					Aggiunge una nuova operazione alla classe;	
					\item \emph{addParameter(ind: Number): void}\\
					Aggiunge un nuovo parametro alla classe;
					Parametri:
					\begin{itemize}
						\item \emph{ind: Number} \\
						Indice dell'operazione;
					\end{itemize}
					\item \emph{deleteParameter(ind: Number): void}\\
					Rimuove un parametro dall'operazione passata in input;
					Parametri:
					\begin{itemize}
						\item \emph{ind: Number} \\
						Indice dell'operazione;
					\end{itemize}
					\item \emph{deleteAttribute(ind: Number): void}\\
					Rimuove un attributo alla classe;
					Parametri:
					\begin{itemize}
						\item \emph{ind: Number} \\
						Indice dell'operazione;
					\end{itemize}
					\item \emph{deleteOperation(ind: Number): void}\\
					Rimuove un'operazione dalla classe;
					Parametri:
					\begin{itemize}
						\item \emph{ind: Number} \\
						Indice dell'operazione;
					\end{itemize}
					\item \emph{getAttrsDesc(): Object[]}\\
					Ritorna la lista di attributi della classe;
					\item \emph{getOpDesc(): Object[]}\\
					Ritorna la lista di operazioni della classe;
					\item \emph{getItemDesc(): Object[]}\\
					Ritorna le informazioni della classe;		
					\item \emph{getWidth(): Number}\\
					Ritorna la larghezza dell'oggetto grafico;	
				\end{itemize}
			\end{itemize}
			
			\subsubsection{SWEDesigner::Client::Model::Items::Swedesigner.model.classDiagram.items.Interface}
			\hypertarget{SWEDesigner::Client::Model::Items::Swedesigner.model.classDiagram.items.Interface}{}
			\begin{itemize}
				\item \textbf{Tipo}: \emph{Class};
				\item \textbf{Descrizione}: Interfaccia UML;
				\item \textbf{Padre}: \emph{Swedesigner.model.classDiagram.items.Class};
				\item \textbf{Attributi}:
				\begin{itemize}
					\item \emph{markup: string}\\
					Markup HTML per la rappresentazione grafica;
					\item \emph{defaults: Objects}\\
					Attributi di default per l'oggetto (tipo, posizione, dimensione, attributi CSS, stato e contenuto dell'oggetto);
				\end{itemize}
				\item \textbf{Metodi}:
				\begin{itemize}
					\item \emph{initialize(): void}\\
					Inizializzazione di Interface;
					\item \emph{updateRectangles(): void}\\
					Render dell'interfaccia;
					\item \emph{addOperation(): void}\\
					Aggiunge una nuova operazione alla classe;	
					\item \emph{addParameter(ind: Number): void}\\
					Aggiunge un parametro all'operazione passata in input;
					Parametri:
					\begin{itemize}
						\item \emph{ind: Number} \\
						Indice dell'operazione;
					\end{itemize}
					\item \emph{deleteParameter(ind: Number): void}\\
					Rimuove un parametro dall'operazione passata in input;
					Parametri:
					\begin{itemize}
						\item \emph{ind: Number} \\
						Indice dell'operazione;
					\end{itemize}
					\item \emph{deleteOperation(ind: Number): void}\\
					Rimuove un'operazione dalla classe;
					Parametri:
					\begin{itemize}
						\item \emph{ind: Number} \\
						Indice dell'operazione;
					\end{itemize}
					\item \emph{getOpDesc(): Object[]}\\
					Ritorna la lista di operazioni della classe;
					\item \emph{getItemDesc(): Object[]}\\
					Ritorna le informazioni della classe;		
					\item \emph{getWidth(): Number}\\
					Ritorna la larghezza dell'oggetto grafico;	
				\end{itemize}
			\end{itemize}
			
			\subsubsection{SWEDesigner::Client::Model::Items::Swedesigner.model.classDiagram.items.ClComment}
			\hypertarget{SWEDesigner::Client::Model::Items::Swedesigner.model.classDiagram.items.ClComment}{}
			\begin{itemize}
				\item \textbf{Tipo}: \emph{Class};
				\item \textbf{Descrizione}: Commento per diagramma delle classi UML;
				\item \textbf{Padre}: \emph{joint.shapes.basic.TextBlock};
				\item \textbf{Attributi}:
				\begin{itemize}
					\item \emph{toolMarkup: string}\\
					Markup HTML per la rappresentazione grafica;
					\item \emph{defaults: Objects}\\
					Attributi di default per l'oggetto ClComment 1211
				\end{itemize}
				\item \textbf{Metodi}:
				\begin{itemize}
					\item \emph{initialize(): void}\\
					Inizializzazione di ClComment;
					\item \emph{getValues(): void}\\
					Ritorna i valori dell'item ClComment;
					\item \emph{setToValue(value: Object, path: string): void}\\
					Imposta "values.path" a "value";
					Parametri:
					\begin{itemize}
						\item \emph{value: Object} \\
						Valore da assegnare;
						\item \emph{path: string} \\
						Percorso al membro;
					\end{itemize}
					\item \emph{updateContent(): void}\\
					Aggiorna il contenuto dell'item ClComment;	
				\end{itemize}
			\end{itemize}
			
			\subsubsection{SWEDesigner::Client::Model::Items::Swedesigner.model.classDiagram.items.ClCommentView}
			\hypertarget{SWEDesigner::Client::Model::Items::Swedesigner.model.classDiagram.items.ClCommentView}{}
			\begin{itemize}
				\item \textbf{Tipo}: \emph{Class};
				\item \textbf{Descrizione}: View per oggetto "ClComment";
				\item \textbf{Padre}: \emph{joint.shapes.basic.TextBlockView};
				\item \textbf{Metodi}:
				\begin{itemize}
					\item \emph{initialize(): void}\\
					Inizializzazione di ClCommentView;
					\item \emph{render(): Object}\\
					Render dell'item ClCommentView;
					\item \emph{renderTools(): Objects}\\
					Assistenza al metodo "render" per la renderizzazione dell'item;
				\end{itemize}
			\end{itemize}
			
			\subsubsection{SWEDesigner::Client::Model::Items::Swedesigner.model.classDiagram.items.classDiagramLink}
			\hypertarget{SWEDesigner::Client::Model::Items::Swedesigner.model.classDiagram.items.classDiagramLink}{}
			\begin{itemize}
				\item \textbf{Tipo}: \emph{Class};
				\item \textbf{Descrizione}: Collegamento tra due componenti di un diagramma delle classi UML;
				\item \textbf{Padre}: \emph{joint.dia.Link};
				\item \textbf{Attributi}:
				\begin{itemize}
					\item \emph{defaults: Objects}\\
					Attributi di default per l'oggetto;
				\end{itemize}
				\item \textbf{Metodi}:
				\begin{itemize}
					\item \emph{initialize(): void}\\
					Inizializzazione di classDiagramLink;
					\item \emph{getValues(): Object}\\
					Ritorna i valori del collegamento;
					\item \emph{setToValue(value: Object, path: string): void}\\
					Imposta "values.path" a "value";
					Parametri:
					\begin{itemize}
						\item \emph{value: Object} \\
						Valore da assegnare;
						\item \emph{path: string} \\
						Percorso al membro;
					\end{itemize}
				\end{itemize}
			\end{itemize}
			
			\subsubsection{SWEDesigner::Client::Model::Items::Swedesigner.model.classDiagram.items.ClCommentLink}
			\hypertarget{SWEDesigner::Client::Model::Items::Swedesigner.model.classDiagram.items.ClCommentLink}{}
			\begin{itemize}
				\item \textbf{Tipo}: \emph{Class};
				\item \textbf{Descrizione}: Link tra un commento e un componente UML del diagramma delle classi;
				\item \textbf{Padre}: \emph{Swedesigner.model.classDiagram.items.classDiagramLink};
				\item \textbf{Attributi}:
				\begin{itemize}
					\item \emph{defaults: Objects}\\
					Attributi di default per l'oggetto;
				\end{itemize}
			\end{itemize}
			\subsubsection{SWEDesigner::Client::Model::Items::Swedesigner.model.classDiagram.items.Generalization}
			\hypertarget{SWEDesigner::Client::Model::Items::Swedesigner.model.classDiagram.items.Generalization}{}
			\begin{itemize}
				\item \textbf{Tipo}: \emph{Class};
				\item \textbf{Descrizione}: Generalizzazione tra due componenti UML;
				\item \textbf{Padre}: \emph{Swedesigner.model.classDiagram.items.classDiagramLink};
				\item \textbf{Attributi}:
				\begin{itemize}
					\item \emph{defaults: Objects}\\
					Attributi di default per l'oggetto;
				\end{itemize}
			\end{itemize}
			\subsubsection{SWEDesigner::Client::Model::Items::Swedesigner.model.classDiagram.items.Implementation}
			\hypertarget{SWEDesigner::Client::Model::Items::Swedesigner.model.classDiagram.items.Implementation}{}
			\begin{itemize}
				\item \textbf{Tipo}: \emph{Class};
				\item \textbf{Descrizione}: Implementazione tra due componenti UML;
				\item \textbf{Padre}: \emph{Swedesigner.model.classDiagram.items.classDiagramLink};
				\item \textbf{Attributi}:
				\begin{itemize}
					\item \emph{defaults: Objects}\\
					Attributi di default per l'oggetto;
				\end{itemize}
			\end{itemize}
			\subsubsection{SWEDesigner::Client::Model::Items::Swedesigner.model.classDiagram.items.Aggregation}
			\hypertarget{SWEDesigner::Client::Model::Items::Swedesigner.model.classDiagram.items.Aggregation}{}
			\begin{itemize}
				\item \textbf{Tipo}: \emph{Class};
				\item \textbf{Descrizione}: Aggregazione tra due componenti UML;
				\item \textbf{Padre}: \emph{Swedesigner.model.classDiagram.items.classDiagramLink};
				\item \textbf{Attributi}:
				\begin{itemize}
					\item \emph{defaults: Objects}\\
					Attributi di default per l'oggetto;
				\end{itemize}
			\end{itemize}
			\subsubsection{SWEDesigner::Client::Model::Items::Swedesigner.model.classDiagram.items.Composition}
			\hypertarget{SWEDesigner::Client::Model::Items::Swedesigner.model.classDiagram.items.Composition}{}
			\begin{itemize}
				\item \textbf{Tipo}: \emph{Class};
				\item \textbf{Descrizione}: Composizione tra due componenti UML;
				\item \textbf{Padre}: \emph{Swedesigner.model.classDiagram.items.classDiagramLink};
				\item \textbf{Attributi}:
				\begin{itemize}
					\item \emph{defaults: Objects}\\
					Attributi di default per l'oggetto;
				\end{itemize}
			\end{itemize}
			\subsubsection{SWEDesigner::Client::Model::Items::Swedesigner.model.classDiagram.items.Association}
			\hypertarget{SWEDesigner::Client::Model::Items::Swedesigner.model.classDiagram.items.Association}{}
			\begin{itemize}
				\item \textbf{Tipo}: \emph{Class};
				\item \textbf{Descrizione}: Associazione tra due componenti UML;
				\item \textbf{Padre}: \emph{Swedesigner.model.classDiagram.items.classDiagramLink};
				\item \textbf{Attributi}:
				\begin{itemize}
					\item \emph{defaults: Objects}\\
					Attributi di default per l'oggetto;
					\item \textbf{Metodi}:
					\begin{itemize}
						\item \emph{updatelabel(): void}\\
						Aggiornamento della label;
						\item \emph{getcard(): Number}\\
						Ritorna la cardinalità della label;
						\item \emph{getAttribute(): string}\\
						Ritorna l'attributo della label;
						\item \emph{initialize(): void}\\
						Inizializzazione della Association;
						\item \emph{setToValue(value: Object, path: string): void}\\
						Imposta "values.path" a "value";
						Parametri:
						\begin{itemize}
							\item \emph{value: Object} \\
							Valore da assegnare;
							\item \emph{path: string} \\
							Percorso al membro;
						\end{itemize}
					\end{itemize}
				\end{itemize}
			\end{itemize}
			
			%------------------------------------------BUBBLE DIAGRAM------------------------------------------
			
			\subsubsection{SWEDesigner::Client::Model::Items::Swedesigner.model.bubbleDiagram.items.Base}
			\hypertarget{SWEDesigner::Client::Model::Items::Swedesigner.model.bubbleDiagram.items.Base}{}
			\begin{itemize}
				\item \textbf{Tipo}: \emph{Class};
				\item \textbf{Descrizione}: Elemento base generico per il diagramma delle bubble;
				\item \textbf{Padre}: \emph{joint.shapes.basic.Generic};
				\item \textbf{Attributi}:
				\begin{itemize}
					\item \emph{markup: string}\\
					Markup HTML per la rappresentazione grafica;
					\item \emph{defaults: Object}\\
					Attributi di default per l'oggetto;
				\end{itemize}
				\item \textbf{Metodi}:
				\begin{itemize}
					\item \emph{initialize(): void}\\
					Inizializzazione di Base: imposta evento al verificarsi del cambio dei valori e chiama il metodo per la renderizzazione dell'item;
					\item \emph{getValues(): Object}\\
					Ritorna i valori dell'item;
					\item \emph{updateRectangles(): void}\\
					Render dell'item;	
				\end{itemize}
			\end{itemize}
			
			\subsubsection{SWEDesigner::Client::Model::Items::Swedesigner.model.bubbleDiagram.items.BaseView}
			\hypertarget{SWEDesigner::Client::Model::Items::Swedesigner.model.bubbleDiagram.items.BaseView}{}
			\begin{itemize}
				\item \textbf{Tipo}: \emph{Class};
				\item \textbf{Descrizione}: Elemento view base generico per il diagramma delle bubble;
				\item \textbf{Padre}: \emph{joint.dia.ElementView};
				\item \textbf{Metodi}:
				\begin{itemize}
					\item \emph{initialize(): void}\\
					Inizializzazione di BaseView: chiama il metodo "initialize" della classe base e imposta un evento alla reazione del model chiamando sequenzialmente i metodi "update" e "resize";
					\item \emph{render(): Object}\\
					Render dell'item;
					\item \emph{renderTools(): Object}\\
					Assistenza al metodo "render" per la renderizzazione dell'item;
				\end{itemize}
			\end{itemize}
			
			\subsubsection{SWEDesigner::Client::Model::Items::Swedesigner.model.bubbleDiagram.items.customBubble}
			\hypertarget{SWEDesigner::Client::Model::Items::Swedesigner.model.bubbleDiagram.items.customBubble}{}
			\begin{itemize}
				\item \textbf{Tipo}: \emph{Class};
				\item \textbf{Descrizione}: Elemento custom bubble per il diagramma delle bubble;
				\item \textbf{Padre}: \emph{Swedesigner.model.bubbleDiagram.items.Base};
				\item \textbf{Attributi}:
				\begin{itemize}
					\item \emph{markup: string}\\
					Markup HTML per la rappresentazione grafica;
					\item \emph{defaults: Objects}\\
					Attributi di default per l'oggetto customBubble (tipo, posizione, dimensione, attributi CSS, stato e contenuto dell'oggetto);
				\end{itemize}
				\item \textbf{Metodi}:
				\begin{itemize}
					\item \emph{initialize(): void}\\
					Inizializzazione di customBubble: chiama il metodo "initialize" della classe base e crea l'istanza dell'oggetto customBubble;
					\item \emph{updateRectangles(): void}\\
					Render della custom bubble;
					\item \emph{setToValue(value: Object, path: string): void}\\
					Imposta "values.path" a "value";
					Parametri:
					\begin{itemize}
						\item \emph{value: Object} \\
						Valore da assegnare;
						\item \emph{path: string} \\
						Percorso al membro;
					\end{itemize}
				\end{itemize}
			\end{itemize}
			
			\subsubsection{SWEDesigner::Client::Model::Items::Swedesigner.model.bubbleDiagram.items.bubbleIf}
			\hypertarget{SWEDesigner::Client::Model::Items::Swedesigner.model.bubbleDiagram.items.bubbleIf}{}
			\begin{itemize}
				\item \textbf{Tipo}: \emph{Class};
				\item \textbf{Descrizione}: Rappresenta un'istruzione condizionale;
				\item \textbf{Padre}: \emph{Swedesigner.model.bubbleDiagram.items.Base};
				\item \textbf{Attributi}:
				\begin{itemize}
					\item \emph{markup: string}\\
					Markup HTML per la rappresentazione grafica;
					\item \emph{defaults: Objects}\\
					Attributi di default per l'oggetto bubbleIf (tipo, posizione, dimensione, attributi CSS, stato e contenuto dell'oggetto);
				\end{itemize}
				\item \textbf{Metodi}:
				\begin{itemize}
					\item \emph{initialize(): void}\\
					Inizializzazione di bubbleIf: chiama il metodo "initialize" della classe base e crea l'istanza dell'oggetto bubbleIf;
					\item \emph{updateRectangles(): void}\\
					Render della bubbleIf;
					\item \emph{setToValue(value: Object, path: string): void}\\
					Imposta "values.path" a "value";
					Parametri:
					\begin{itemize}
						\item \emph{value: Object} \\
						Valore da assegnare;
						\item \emph{path: string} \\
						Percorso al membro;
					\end{itemize}
				\end{itemize}
			\end{itemize}
			
			\subsubsection{SWEDesigner::Client::Model::Items::Swedesigner.model.bubbleDiagram.items.bubbleElse}
			\hypertarget{SWEDesigner::Client::Model::Items::Swedesigner.model.bubbleDiagram.items.bubbleElse}{}
			\begin{itemize}
				\item \textbf{Tipo}: \emph{Class};
				\item \textbf{Descrizione}: Rappresenta il ramo 'else' di un'istruzione condizionale;
				\item \textbf{Padre}: \emph{Swedesigner.model.bubbleDiagram.items.Base};
				\item \textbf{Attributi}:
				\begin{itemize}
					\item \emph{markup: string}\\
					Markup HTML per la rappresentazione grafica;
					\item \emph{defaults: Objects}\\
					Attributi di default per l'oggetto bubbleElse (tipo, posizione, dimensione, attributi CSS, stato e contenuto dell'oggetto);
				\end{itemize}
				\item \textbf{Metodi}:
				\begin{itemize}
					\item \emph{initialize(): void}\\
					Inizializzazione di bubbleIf: chiama il metodo "initialize" della classe base e crea l'istanza dell'oggetto bubbleElse;
					\item \emph{updateRectangles(): void}\\
					Render della bubbleElse;
					\item \emph{setToValue(value: Object, path: string): void}\\
					Imposta "values.path" a "value";
					Parametri:
					\begin{itemize}
						\item \emph{value: Object} \\
						Valore da assegnare;
						\item \emph{path: string} \\
						Percorso al membro;
					\end{itemize}
				\end{itemize}
			\end{itemize}
			
			\subsubsection{SWEDesigner::Client::Model::Items::Swedesigner.model.bubbleDiagram.items.bubbleFor}
			\hypertarget{SWEDesigner::Client::Model::Items::Swedesigner.model.bubbleDiagram.items.bubbleFor}{}
			\begin{itemize}
				\item \textbf{Tipo}: \emph{Class};
				\item \textbf{Descrizione}: Rappresenta un'iterazione lungo una sequenza di istruzioni;
				\item \textbf{Padre}: \emph{Swedesigner.model.bubbleDiagram.items.Base};
				\item \textbf{Attributi}:
				\begin{itemize}
					\item \emph{markup: string}\\
					Markup HTML per la rappresentazione grafica;
					\item \emph{defaults: Objects}\\
					Attributi di default per l'oggetto bubbleFor (tipo, posizione, dimensione, attributi CSS, stato e contenuto dell'oggetto);
				\end{itemize}
				\item \textbf{Metodi}:
				\begin{itemize}
					\item \emph{initialize(): void}\\
					Inizializzazione di bubbleFor: chiama il metodo "initialize" della classe base e crea l'istanza dell'oggetto bubbleFor;
					\item \emph{updateRectangles(): void}\\
					Render della bubbleFor;
					\item \emph{setToValue(value: Object, path: string): void}\\
					Imposta "values.path" a "value";
					Parametri:
					\begin{itemize}
						\item \emph{value: Object} \\
						Valore da assegnare;
						\item \emph{path: string} \\
						Percorso al membro;
					\end{itemize}
				\end{itemize}
			\end{itemize}
			
			\subsubsection{SWEDesigner::Client::Model::Items::Swedesigner.model.bubbleDiagram.items.bubbleReturn}
			\hypertarget{SWEDesigner::Client::Model::Items::Swedesigner.model.bubbleDiagram.items.bubbleReturn}{}
			\begin{itemize}
				\item \textbf{Tipo}: \emph{Class};
				\item \textbf{Descrizione}: Rappresenta un'istruzione per uscire da un metodo e ritornare degli argomenti al chiamante;
				\item \textbf{Padre}: \emph{Swedesigner.model.bubbleDiagram.items.Base};
				\item \textbf{Attributi}:
				\begin{itemize}
					\item \emph{markup: string}\\
					Markup HTML per la rappresentazione grafica;
					\item \emph{defaults: Objects}\\
					Attributi di default per l'oggetto bubbleReturn (tipo, posizione, dimensione, attributi CSS, stato e contenuto dell'oggetto);
				\end{itemize}
				\item \textbf{Metodi}:
				\begin{itemize}
					\item \emph{initialize(): void}\\
					Inizializzazione di bubbleReturn: chiama il metodo "initialize" della classe base e crea l'istanza dell'oggetto bubbleReturn;
					\item \emph{updateRectangles(): void}\\
					Render della bubbleReturn;
					\item \emph{setToValue(value: Object, path: string): void}\\
					Imposta "values.path" a "value";
					Parametri:
					\begin{itemize}
						\item \emph{value: Object} \\
						Valore da assegnare;
						\item \emph{path: string} \\
						Percorso al membro;
					\end{itemize}
				\end{itemize}
			\end{itemize}
			
			\subsubsection{SWEDesigner::Client::Model::Items::Swedesigner.model.bubbleDiagram.items.bubbleStart}
			\hypertarget{SWEDesigner::Client::Model::Items::Swedesigner.model.bubbleDiagram.items.bubbleStart}{}
			\begin{itemize}
				\item \textbf{Tipo}: \emph{Class};
				\item \textbf{Descrizione}: Rappresenta la prima istruzione di un metodo;
				\item \textbf{Padre}: \emph{Swedesigner.model.bubbleDiagram.items.Base};
				\item \textbf{Attributi}:
				\begin{itemize}
					\item \emph{markup: string}\\
					Markup HTML per la rappresentazione grafica;
					\item \emph{defaults: Objects}\\
					Attributi di default per l'oggetto bubbleStart (tipo, posizione, dimensione, attributi CSS, stato e contenuto dell'oggetto);
				\end{itemize}
				\item \textbf{Metodi}:
				\begin{itemize}
					\item \emph{initialize(): void}\\
					Inizializzazione di bubbleStart: chiama il metodo "initialize" della classe base e crea l'istanza dell'oggetto bubbleStart;
					\item \emph{updateRectangles(): void}\\
					Render della bubbleStart;
					\item \emph{setToValue(value: Object, path: string): void}\\
					Imposta "values.path" a "value";
					Parametri:
					\begin{itemize}
						\item \emph{value: Object} \\
						Valore da assegnare;
						\item \emph{path: string} \\
						Percorso al membro;
					\end{itemize}
				\end{itemize}
			\end{itemize}
			
			\subsubsection{SWEDesigner::Client::Model::Items::Swedesigner.model.bubbleDiagram.items.bubbleWhile}
			\hypertarget{SWEDesigner::Client::Model::Items::Swedesigner.model.bubbleDiagram.items.bubbleWhile}{}
			\begin{itemize}
				\item \textbf{Tipo}: \emph{Class};
				\item \textbf{Descrizione}: Rappresenta un loop con controllo di condizione lungo una sequenza di istruzioni;
				\item \textbf{Padre}: \emph{Swedesigner.model.bubbleDiagram.items.Base};
				\item \textbf{Attributi}:
				\begin{itemize}
					\item \emph{markup: string}\\
					Markup HTML per la rappresentazione grafica;
					\item \emph{defaults: Objects}\\
					Attributi di default per l'oggetto bubbleWhile (tipo, posizione, dimensione, attributi CSS, stato e contenuto dell'oggetto);
				\end{itemize}
				\item \textbf{Metodi}:
				\begin{itemize}
					\item \emph{initialize(): void}\\
					Inizializzazione di bubbleWhile: chiama il metodo "initialize" della classe base e crea l'istanza dell'oggetto bubbleWhile;
					\item \emph{updateRectangles(): void}\\
					Render della bubbleWhile;
					\item \emph{setToValue(value: Object, path: string): void}\\
					Imposta "values.path" a "value";
					Parametri:
					\begin{itemize}
						\item \emph{value: Object} \\
						Valore da assegnare;
						\item \emph{path: string} \\
						Percorso al membro;
					\end{itemize}
				\end{itemize}
			\end{itemize}
			
			\subsubsection{SWEDesigner::Client::Model::Items::Swedesigner.model.bubbleDiagram.items.bubbleDiagramLink}
			\hypertarget{SWEDesigner::Client::Model::Items::Swedesigner.model.bubbleDiagram.items.bubbleDiagramLink}{}
			\begin{itemize}
				\item \textbf{Tipo}: \emph{Class};
				\item \textbf{Descrizione}: Collegamento tra due componenti di un diagramma delle bubble;
				\item \textbf{Padre}: \emph{joint.dia.Link};
				\item \textbf{Attributi}:
				\begin{itemize}
					\item \emph{defaults: Objects}\\
					Attributi di default per l'oggetto;
				\end{itemize}
				\item \textbf{Metodi}:
				\begin{itemize}
					\item \emph{initialize(): void}\\
					Inizializzazione di bubbleDiagramLink;
					\item \emph{setToValue(value: Object, path: string): void}\\
					Imposta "values.path" a "value";
					Parametri:
					\begin{itemize}
						\item \emph{value: Object} \\
						Valore da assegnare;
						\item \emph{path: string} \\
						Percorso al membro;
					\end{itemize}
				\end{itemize}
			\end{itemize}
			
			\subsubsection{SWEDesigner::Client::Model::Items::Swedesigner.model.bubbleDiagram.items.bubbleLink}
			\hypertarget{SWEDesigner::Client::Model::Items::Swedesigner.model.bubbleDiagram.items.bubbleLink}{}
			\begin{itemize}
				\item \textbf{Tipo}: \emph{Class};
				\item \textbf{Descrizione}: Link tra due elementi del diagramma delle bubble;
				\item \textbf{Padre}: \emph{Swedesigner.model.bubbleDiagram.items.bubbleDiagramLink};
				\item \textbf{Attributi}:
				\begin{itemize}
					\item \emph{defaults: Objects}\\
					Attributi di default per l'oggetto;
				\end{itemize}\
			\end{itemize}
					
			%fine parte Giulio
			\subsection{SWEDesigner::Client::View}
				\hypertarget{SWEDesigner::Client::View}
				Questo package non contiene dei sottopackage.
				Le classi contenute al suo interno verranno elencate qui di seguito.
				\subsubsection{SWEDesigner::Client::View::projectView}
					% IMMAGINE ARCHITETTURA MAINVIEW
					\begin{figure}[H]\label{fig:MainModel}
						\centering
						\includegraphics[scale=0.44]{Immagini/DiagrammaArchitettura/MainView.png}
						\caption{Architettura di projectView}
					\end{figure}

				È il componente del programma che si occupa di gestire l'interfaccia grafica. Nella particolare declinazione MVC adottata da Backbone.js, si occupa anche di gestire gli input dell'utente.\\
					FAN-IN:
					\begin{itemize}
						\item pathView: gestisce l'interfaccia grafica della barra di indirizzo;
						\item editPanelView: gestisce l'interfaccia grafica del pannello di editing.
					\end{itemize}
					FAN-OUT:
					\begin{itemize}
						\item projectModel: si occupa di gestire la parte logica dell'editor;
						\item Paper: gestisce l'interfaccia grafica dell'area dei diagrammi.
					\end{itemize}

				\subsubsection{SWEDesigner::Client::View::titlebarView}
				È il componente del programma che fa la funzione di view per la barra del titolo, dove saranno collocati il menu dell’applicazione e gli shortcut.\\
					FAN-IN:\\
					Non ci sono dipendenze IN.\\
					FAN-OUT:
					\begin{itemize}
						\item requestHandler: gestisce la comunicazione con il server;
						\item dataManager: gestisce la persistenza dei dati su file system.
					\end{itemize}

				\subsubsection{SWEDesigner::Client::View::toolbarView}
				È il componente del programma che fa la funzione di view per la toolbar dove saranno collocati gli strumenti per editare i diagrammi.\\
					FAN-IN:\\
					Non ci sono dipendenze IN.\\
					FAN-OUT:
					\begin{itemize}
						\item projectModel: si occupa di gestire la parte logica dell'editor;
						\item toolbarModel: si occupa di gestire la parte logica della toolbar.
					\end{itemize}

				\subsubsection{SWEDesigner::Client::View::pathView}
				È il componente del programma che fa la funzione di view per il cosiddetto breadcrumb dove viene inserita la posizione corrente.\\
					FAN-IN:
					Non ci sono dipendenze IN. \\
					FAN-OUT:
					\begin{itemize}
						\item projectModel: si occupa di gestire la parte logica dell'editor;
						\item projectView: si occupa di gestire la parte grafica del model.
					\end{itemize}
				
				\subsubsection{SWEDesigner::Client::View::EditPanelView}
				È il componente del programma che fa la funzione di view per le informazioni editabili degli elementi che fanno parte dei diversi diagrammi.\\
					FAN-IN:\\
					Non ci sono dipendenze IN. \\
					FAN-OUT:
					\begin{itemize}
						\item projectView: si occupa di gestire la parte grafica del model.
					\end{itemize}
			
			\subsection{SWEDesigner::Server}
				% IMMAGINE ARCHITETTURA SERVER GENERALE
				\begin{figure}[H]\label{fig:ServerSubsystem}
					\centering
					\includegraphics[scale=0.4]{Immagini/DiagrammaArchitettura/ServerSubsystem.png}
					\caption{Architettura del server}
				\end{figure}
				I package contenuti al suo interno sono:
				\begin{itemize}
					\item SWEDesigner::Server::CodeGenerator;
					\item SWEDesigner::Server::DAORequestHandler;
					\item SWEDesigner::Server::RequestHandler.
				\end{itemize}
				Questo package non contiene delle classi.

			\subsection{SWEDesigner::Server::CodeGenerator}
				I package contenuti al suo interno sono:
				\begin{itemize}
					\item SWEDesigner::Server::CodeGenerator::Builder;
					\item SWEDesigner::Server::CodeGenerator::Coder;
					\item SWEDesigner::Server::CodeGenerator::Parser;
					\item SWEDesigner::Server::CodeGenerator::Zipper.
				\end{itemize}
				Le classi contenute al suo interno verranno elencate qui di seguito.

				\subsubsection{SWEDesigner::Server::CodeGenerator::CodeGenerator}
				\hypertarget{SWEDesigner::Server::CodeGenerator::CodeGenerator}
				E' il componente che rende disponibile la funzionalità per cui, dato un file valido in formato JSON, restituisce un pacchetto in formato .zip contenente i file del codice sorgente che costituiscono il programma rappresentato dal file in input. I file prodotti sono strutturati in packages, come indicato nel file JSON in input.\\
					FAN-IN:
					\begin{itemize}
						\item Server::RequestHandler::Receiver: si occupa di gestire le comunicazioni in entrata dal client.
					\end{itemize}
					FAN-OUT:
					\begin{itemize}
						\item Server::RequestHandler::Sender: si occupa di gestire le comunicazioni in uscita verso il client;
						\item Parser: si occupa di creare un oggetto che contiene le informazioni ricevute in input;
						\item Coder: si occupa della traduzione in codice dell'oggetto ottenuto dal Parser;
						\item Builder: si occupa di organizzare in maniera organica il codice generato dal Coder;
						\item Zipper: si occupa di creare un archivio .zip contenente in codici sorgente precedentemente creati.
					\end{itemize}

			\subsection{SWEDesigner::Server::CodeGenerator::Builder}
				Questo package non contiene dei sottopackage.\\
				Le classi contenute al suo interno verranno elencate qui di seguito.
				\subsubsection{SWEDesigner::Server::CodeGenerator::Builder::Builder}
				È il componente che rende disponibile la funzionalità, dato un file JSON in input che rappresenti un programma, di ottenere un oggetto contenitore del codice sorgente corrispondente al contenuto del file di input. Tale codice è suddiviso e strutturato come indicato nel file di input.\\
					FAN-IN:
					\begin{itemize}
						\item Zipper: si occupa di creare un archivio .zip contenente in codici sorgente precedentemente creati.
					\end{itemize}
					FAN-OUT:
					\begin{itemize}
						\item Coder: componente che funge da interfaccia alle operazioni di codifica di una stringa permettendo quindi di trasformare le informazioni del file in formato JSON in codice sorgente.
					\end{itemize}

			\subsection{SWEDesigner::Server::CodeGenerator::Coder}
				% IMMAGINE ARCHITETTURA CODER
				\begin{figure}[H]\label{fig:Coder}
					\centering
					\includegraphics[scale=0.46]{Immagini/DiagrammaArchitettura/Coder.png}
					\caption{Architettura di Coder}
				\end{figure}

				Questo package non contiene dei sottopackage.\\
				Le classi contenute al suo interno verranno elencate qui di seguito.
				\subsubsection{SWEDesigner::Server::CodeGenerator::Coder::Coder}
				Componente che funge da interfaccia alle operazioni di codifica di una stringa, in formato JSON che rappresenta un programma valido; tali operazioni permettono di ottenere un oggetto contenente il codice sorgente, in Java o Javascript, corrispondente alla stringa in input.\\
					FAN-IN:
					\begin{itemize}
						\item JavaCoder: si occupa di trasformare un oggetto JSON ricevuto in input in un oggetto contenente il codice sorgente scritto in java;
						\item JavaScriptCoder: si occupa di trasformare un oggetto JSON ricevuto in input in un oggetto contenente il codice sorgente scritto in javascript.
					\end{itemize}
					FAN-OUT:
					\begin{itemize}
						\item CodedProg: componente che contiene il codice prodotto dal Coder;
						\item CoderElement: componente astratto che offre la funzionalità che permette di associare ad ogni stringa contenuta nel file JSON il corrispondente codice sorgente.
					\end{itemize}

				\subsubsection{SWEDesigner::Server::CodeGenerator::Coder::JavaCoder}
				È il componente che rende disponibile la funzionalità, dato un oggetto in input che rappresenta un file JSON parsificato, di ottenere un oggetto contenente il codice sorgente, in linguaggio Java, corrispondente all'oggetto in input.\\
					Non ci sono dipendenze IN.\\
					FAN-OUT:
					\begin{itemize}
						\item Coder: componente che funge da interfaccia alle operazioni di codifica di una stringa permettendo quindi di trasformare le informazioni del file in formato JSON in codice sorgente.
					\end{itemize}

				\subsubsection{SWEDesigner::Server::CodeGenerator::Coder::JavascriptCoder}
				È il componente che rende disponibile la funzionalità, dato un oggetto in input che rappresenta un file JSON parsificato, di ottenere un oggetto contenente il codice sorgente, in linguaggio Javascript, corrispondente all'oggetto in input.\\
					Non ci sono dipendenze IN.\\
					FAN-OUT:
					\begin{itemize}
						\item Coder: componente che funge da interfaccia alle operazioni di codifica di una stringa permettendo quindi di trasformare le informazioni del file in formato JSON in codice sorgente.
					\end{itemize}

				\subsubsection{SWEDesigner::Server::CodeGenerator::Coder::CoderClass}
				È il componente che mette a disposizione la funzionalità, data una stringa in input in formato JSON che rappresenta una classe valida, di ottenere il corrispondente codice sorgente di tale classe.\\
					Non ci sono dipendenze IN.\\
					FAN-OUT:
					\begin{itemize}
						\item CoderElement: componente astratto che offre la funzionalità che permette di associare ad ogni stringa contenuta nel file JSON il corrispondente codice sorgente.
					\end{itemize}

				\subsubsection{SWEDesigner::Server::CodeGenerator::Coder::CoderOperation}
				È il componente che mette a disposizione la funzionalità, data una stringa in input in formato JSON che rappresenta un'operazione valida, di ottenere il corrispondente codice sorgente di tale operazione.\\
					Non ci sono dipendenze IN.\\
					FAN-OUT:
					\begin{itemize}
						\item CoderElement: componente astratto che offre la funzionalità che permette di associare ad ogni stringa contenuta nel file JSON il corrispondente codice sorgente.
					\end{itemize}

				\subsubsection{SWEDesigner::Server::CodeGenerator::Coder::CoderParameter}
				È il componente che mette a disposizione la funzionalità, data una stringa in input in formato JSON che rappresenta un parametro di una lista valido, di ottenere il corrispondente codice sorgente di tale parametro. È possibile scegliere fra la codifica in Java o Javascript.\\
					Non ci sono dipendenze IN.\\
					FAN-OUT:
					\begin{itemize}
						\item CoderElement: componente astratto che offre la funzionalità che permette di associare ad ogni stringa contenuta nel file JSON il corrispondente codice sorgente.
					\end{itemize}

				\subsubsection{SWEDesigner::Server::CodeGenerator::Coder::CoderAttribute}
				È il componente che mette a disposizione la funzionalità, data una stringa in input in formato JSON che rappresenta un attributo valido, di ottenere il corrispondente codice sorgente di tale attributo. È possibile scegliere fra la codifica in Java o Javascript.\\
				Non ci sono dipendenze IN.\\
					FAN-OUT:
					\begin{itemize}
						\item CoderElement: componente astratto che offre la funzionalità che permette di associare ad ogni stringa contenuta nel file JSON il corrispondente codice sorgente.
					\end{itemize}

				\subsubsection{SWEDesigner::Server::CodeGenerator::Coder::CoderActivity}
				È il componente che mette a disposizione la funzionalità, data una stringa in input in formato JSON che rappresenta un diagramma delle attività valido, di ottenere il corrispondente codice sorgente di tale attività. È possibile scegliere fra la codifica in Java o Javascript.\\
					Non ci sono dipendenze IN.\\
					FAN-OUT:
					\begin{itemize}
						\item CoderElement: componente astratto che offre la funzionalità che permette di associare ad ogni stringa contenuta nel file JSON il corrispondente codice sorgente;
						\item DAO: si occupa di gestire il database delle bubble.
					\end{itemize}

				\subsubsection{SWEDesigner::Server::CodeGenerator::Coder::CodedProg}
				È il componente che contiene il codice sorgente prodotto dal Coder.\\
					FAN-IN:
					\begin{itemize}
						\item Coder: componente che funge da interfaccia alle operazioni di codifica di una stringa permettendo quindi di trasformare le informazioni del file in formato JSON in codice sorgente.
					\end{itemize}
					Non ci sono dipendenze OUT.
				
				\subsubsection{SWEDesigner::Server::CodeGenerator::Coder::CoderElement}
				Componente astratta che offre la funzionalità di ottenere, data una stringa in input in formato JSON che rappresenta un elemento di classe valido, il corrispondente codice sorgente, in Java o Javascript.\\
					FAN-IN:
					\begin{itemize}
						\item Coder: componente che funge da interfaccia alle operazioni di codifica di una stringa permettendo quindi di trasformare le informazioni del file in formato JSON in codice sorgente;
						\item CoderClass: componente che permette data una stringa in input in formato JSON che rappresenta un diagramma delle classi valido, di ottenere il corrispondente codice sorgente di tale classe;
						\item CoderOperations: componente che permette data una stringa in input in formato JSON che rappresenta un'operazione valida, di ottenere il corrispondente codice sorgente di tale operazione;
						\item CoderAttributes: componente che permette data una stringa in input in formato JSON che rappresenta un attributo valido, di ottenere il corrispondente codice sorgente di tale attributo;
						\item CoderActivity: componente che permette data una stringa in input in formato JSON che rappresenta un diagramma delle attività valido, di ottenere il corrispondente codice sorgente di tale attività;
						\item CoderParameter: componente che permette data una stringa in input in formato JSON che rappresenta un parametro valido, di ottenere il corrispondente codice sorgente di tale parametro.
					\end{itemize}
					Non ci sono dipendenze OUT.

			\subsection{SWEDesigner::Server::CodeGenerator::Parser}
				Questo package non contiene dei sottopackage.
				Le classi contenute al suo interno verranno elencate qui di seguito.
				\subsubsection{SWEDesigner::Server::CodeGenerator::Parser::Parser}
				È il componente che rende disponibile la funzionalità, dato un file JSON valido in input, di ottenere un oggetto contenente le informazioni che costituiscono il file in input.\\
					FAN-IN:
					\begin{itemize}
						\item CodeGenerator: si occupa di restituire in output un archivio zip contenente i codici sorgenti generati a partire dal file JSON ricevuto in input;
						\item Coder: componente che funge da interfaccia alle operazioni di codifica di una stringa permettendo quindi di trasformare le informazioni del file in formato JSON in codice sorgente.
					\end{itemize}
					Non ci sono dipendenze OUT.

			\subsection{SWEDesigner::Server::CodeGenerator::Zipper}
				Questo package non contiene dei sottopackage.\\
				Le classi contenute al suo interno verranno elencate qui di seguito.
				\subsubsection{SWEDesigner::Server::CodeGenerator::Zipper::Zipper}
				E' il componente che rende disponibile la funzionalità per cui, dato un file valido in formato JSON, restituisce un pacchetto in formato .zip contenente i file del codice sorgente che costituiscono il programma rappresentato dal file in input. I file prodotti sono strutturati in packages, come indicato nel file JSON in input.\\
					FAN-IN:
					\begin{itemize}
						\item CodeGenerator: si occupa di restituire in output un archivio zip contenente i codici sorgenti generati a partire dal file JSON ricevuto in input.
					\end{itemize}
					FAN-OUT:
					\begin{itemize}
						\item Builder: componente che si occupa di creare un oggetto contenitore con il codice sorgente, partendo dalle informazioni prese dal file JSON ricevuto in input che rappresenta un programma.
					\end{itemize}

				\subsubsection{SWEDesigner::Server::DAO}
				\hypertarget{SWEDesigner::Server::DAO}
				Questa classe si occupa di gestire il database delle bubble.\\
					FAN-IN:
					\begin{itemize}
						\item Coder: componente che funge da interfaccia alle operazioni di codifica di una stringa permettendo quindi di trasformare le informazioni del file in formato JSON in codice sorgente.
					\end{itemize}
					Non ci sono dipendenze OUT.

			\subsection{SWEDesigner::Server::RequestHandler}
				\hypertarget{SWEDesigner::Server::RequestHandler}
				Questo package non contiene dei sottopackage.\\
				Le classi contenute al suo interno verranno elencate qui di seguito.
				\subsubsection{SWEDesigner::Server::RequestHandler::Sender}
				Si occupa di gestire le comunicazioni in uscita verso il client.\\
					FAN-IN:
					\begin{itemize}
						\item CodeGenerator: si occupa di restituire in output un archivio zip contenente i codici sorgenti generati a partire dal file JSON ricevuto in input.
					\end{itemize}
					FAN-OUT:
					\begin{itemize}
						\item Client::Model::RequestHandler::Receiver: si occupa di gestire le comunicazioni in entrata dal server.
					\end{itemize}
					
				\subsubsection{SWEDesigner::Server::RequestHandler::Receiver}
				Si occupa di gestire le comunicazioni in entrata dal client.\\
					FAN-IN:
					\begin{itemize}
						\item Client::Model::RequestHandler::Sender: si occupa di gestire le comunicazioni in uscita verso il server.
					\end{itemize}
					FAN-OUT:
					\begin{itemize}
						\item CodeGenerator: si occupa di restituire in output un archivio zip contenente i codici sorgenti generati a partire dal file JSON ricevuto in input.
					\end{itemize}
\end{document}
