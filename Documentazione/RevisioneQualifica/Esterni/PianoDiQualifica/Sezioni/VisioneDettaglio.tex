\documentclass[../PianoDiQualifica.tex]{subfiles}
\begin{document}
	\section{La strategia di gestione della qualità nel dettaglio}
		\subsection{Risorse}
			\subsubsection{Necessarie}
				Per la realizzazione del prodotto sono necessarie le risorse
				umane e tecnologiche elencate di seguito.
				\begin{itemize}
					\item \textbf{Risorse umane}: sono descritte
					dettagliatamente nel \pianodiprogettov:
					\begin{itemize}
						\item \responsabilediprogetto;
						\item \amministratore;
						\item \analista;
						\item \progettista;
						\item \programmatore;
						\item \verificatore.
					\end{itemize}
					\item \textbf{Risorse software}: sono descritte
					dettagliatamente nelle \normediprogettov. Si tratta di
					software che permettono:
					\begin{itemize}
						\item la comunicazione e la condivisione del lavoro
						tra gli elementi del team;
						\item la stesura della documentazione in
						formato \gl{\LaTeX};
						\item la creazione di diagrammi UML;
						\item la codifica nei linguaggi di programmazione scelti;
						\item la semplificazione delle attività di verifica;
						\item la gestione dei test sul codice.
					\end{itemize}
					\item \textbf{Risorse hardware}: ciascun componente del
					gruppo deve avere un computer con tutti i software necessari
					descritti nelle \normediprogettov. È necessario avere a
					disposizione almeno un luogo dove poter effettuare le
					riunioni interne.
				\end{itemize}
			\subsubsection{Disponibili}
				Ogni membro del team ha a disposizione uno o più computer
				personali dotati degli strumenti necessari.\\
				Le riunioni interne si svolgono presso le aule del dipartimento
				di Matematica dell'Università degli Studi di Padova.
		\subsection{Misure e metriche}\label{sez:MisureEMetriche}
			Il processo di verifica deve essere quantificabile per fornire
			informazioni utili.\\
			Le metriche adottate sono descritte approfonditamente nel documento \normediprogetto.
			Per alcune di esse si definiranno due intervalli di misure (\gl{range}):
			\begin{itemize}
				\item \textbf{Range di accettazione}: intervallo di valori
				vincolante per l'accettazione del prodotto;
				\item \textbf{Range ottimale}: intervallo di valori entro cui è
				consigliabile rientri la misurazione. Il mancato rispetto di
				questa condizione non pregiudica l'accettazione del prodotto, ma
				richiede verifiche più approfondite in merito.
			\end{itemize}
			Di seguito sono definiti gli specifici obiettivi quantitativi da perseguire.
			\subsubsection{Obiettivi di qualità di processo}\label{sez:MetrichePerQualitaDiProcesso}
				\normalsize
				\begin{table}[H]
				\center
					\begin{tabular}{|p{6.5cm}|c|c|}
						\hline
						\rowcolor{blue!30}\textbf{Metrica} & \textbf{Range di accettazione} & \textbf{Range ottimale} \\ \hline
						Numero di violazioni delle norme di progettazione & $0$ - $10$ & $0$ - $5$\\ \hline
						Numero di violazioni delle norme di codifica & $0$ - $10$ & $0$ - $5$\\ \hline
						Percentuale di test di validazione effettuati & $90\%$ - $100\%$ & $100\%$\\ \hline
						Percentuale di test di integrazione effettuati & $90\%$ - $100\%$ & $100\%$\\ \hline
						Percentuale di test di sistema effettuati & $90\%$ - $100\%$ & $100\%$\\ \hline
						Percentuale di test di unità effettuati & $90\%$ - $100\%$ & $100\%$\\ \hline
						Schedule Variance & $\geq -(preventivo*10\%)$ & $\geq 0$ \\ \hline
						Budget Variance & $\geq -(preventivo*10\%)$ & $\geq 0$ \\ \hline
						Indice Gulpease & $40$ - $100$ & $50$ - $100$ \\ \hline
					\end{tabular}
					\caption{Metriche per qualità di processo}
				\end{table}
			\subsubsection{Obiettivi di qualità di prodotto}\label{sez:MetrichePerQualitaDiProdotto}
				\begin{table}[H]
				\center
					\begin{tabular}{|l|c|}
						\hline
						\rowcolor{blue!30}\textbf{Metrica} & \textbf{Obiettivo} \\ \hline
						Soddisfacimento dei requisiti obbligatori & $100\%$\\ \hline
						Soddisfacimento dei requisiti desiderabili & $70\%$ - $100\%$\\ \hline
					\end{tabular}
					\begin{tabular}{|l|c|c|}
						\hline
						\rowcolor{blue!30}\textbf{Metrica} & \textbf{Range di accettazione} & \textbf{Range ottimale}\\ \hline
						Percentuale totale di test superati & $80\%$ - $100\%$ & $90\%$ - $100\%$ \\ \hline
						Grado di accoppiamento afferente & $0$ - $7$ & $0$ - $5$ \\ \hline
						Grado di accoppiamento efferente & $0$ - $7$ & $0$ - $5$ \\ \hline
						Linee di commento su linee di codice & $\geq 0.25$ & $\geq 0.30$ \\ \hline
						Numero di parametri & $0$ - $8$ & $0$ - $5$  \\ \hline
						Numero di campi dati & $0$ - $16$ & $0$ - $10$ \\ \hline
						Complessità ciclomatica & $0$ - $10$ & $0$ - $6$  \\ \hline
						Livello di annidamento & $0$ - $6$ & $0$ - $4$ \\ \hline
						Chiamate innestate di metodi & $0$ - $6$ & $0$ - $4$ \\ \hline
						Copertura del codice & $80\%$ - $100\%$ & $90\%$ - $100\%$ \\ \hline
						Numero di linee per metodo & $\leq 60$ & $\leq 40$ \\ \hline
						Validazione W3C & $0$ - $10$ (per pagina) & $0$ - $0$ (per pagina) \\ \hline
					\end{tabular}
					\caption{Metriche per qualità di prodotto}
				\end{table}
\end{document}
