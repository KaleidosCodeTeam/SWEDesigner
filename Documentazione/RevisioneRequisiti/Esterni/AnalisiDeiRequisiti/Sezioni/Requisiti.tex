\documentclass[../AnalisiDeiRequisiti.tex]{subfiles}

	
\begin{document}
	\section{Requisiti}
	In questa sezione verranno presentati i requisiti individuali che il team ha individuato durante l'analisi del capitolato e dei casi d'uso, quelli discussi con il proponente durante le riunioni esterne e quelli decisi durante le riunioni interne dal gruppo. 
	Ogni requisito avrà un codice identificativo univoco così formato:
	
	\begin{center} R{Tipo}{Importanza}{Codice} \end{center}
	
	dove:
	\begin{itemize}
		\item \textbf{Tipo:} può assumere i seguenti valori:
			\begin{itemize}
				\item \textbf{F:} indica un requisito funzionale;
				\item \textbf{Q:} indica un requisito di qualità;
				\item \textbf{P:} indica un requisito prestazionale;
				\item \textbf{V:} indica un requisito di vincolo.
			\end{itemize}
			\item \textbf{Importanza:} può assumere i seguenti valori:
			\begin{itemize}
				\item \textbf{O:} indica un requisito obbligatorio;
				\item \textbf{D:} indica un requisito desiderabile;
				\item \textbf{F:} indica un requisito facoltativo.
			\end{itemize}
			\item \textbf{Codice:} indica il codice identificativo del requisito. Deve essere univoco e deve essere identificato in forma gerarchica.
	\end{itemize}
	Per ogni requisito verranno inoltre riportate:
	\begin{itemize}
		\item \textbf{Descrizione:} breve testo che dovrà descrivere in modo completo il requisito;
		\item \textbf{Fonte:} che potrà essere una tra le seguenti:
		\begin{itemize}
			\item \textbf{Capitolato:} requisito dedotto direttamente dallo studio e dall'analisi del capitolato di appalto;
			\item \textbf{Verbale Esterno:} requisito emerso da un verbale esterno;
			\item \textbf{Caso d'uso:} requisito derivato  da un caso d'uso; in questo caso deve essere riportato il codice identificativo del caso d'uso associato.
			\item \textbf{Interno:} requisito identificato dagli \textit{Analisti}.
		\end{itemize}
	\end{itemize}

	\subsection{Requisiti funzionali}
	\subsection{Requisiti di qualità}
	\subsection{Requisiti di vincolo}
	\subsection{Tracciamento fonti-requisiti}
	\subsection{Riepilogo requisiti}
	
\end{document}
