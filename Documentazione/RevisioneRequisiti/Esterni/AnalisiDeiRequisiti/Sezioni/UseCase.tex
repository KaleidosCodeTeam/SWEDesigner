\documentclass[../AnalisiDeiRequisiti.tex]{subfiles}
\begin{document}
	\section{Casi d'uso}
		I seguenti casi d'uso sono frutto dell'analisi del capitolato, della discussione degli
		\analisti\ e degli incontri con	\proponente\ ed il committente \vardanega.
		Tali casi d'uso hanno quindi origine sia interna che esterna al gruppo.\\
		Le aspettative di esperienza utente derivano dalla sua conoscenza del
		linguaggio UML.\\
		Ciascun caso d'uso è classificato gerarchicamente con la seguente dicitura:
		\begin{center}
			UC[Codice del padre].[Codice identificativo]
		\end{center}
		Il codice identificativo può includere diversi livelli di gerarchia che saranno
		separati da un punto.
		
		
		
% DA INSERIRE EVENTUALI REQUISITI, INCLUSIONI, ESTENSIONI, GENERALIZZAZIONI.
% DA INSERIRE RIFERIMENTI A CASI D'USO
% DA INSERIRE DIAGRAMMI CASI D'USO

%%%%%%%%%%%%%%%%%%%%%%%% SEZIONE INTERFACCIA %%%%%%%%%%%%%%%%%%%%%%%
		\subsection{Caso d'uso UC : Modificare un'interfaccia}
			\begin{itemize}
				\item \textbf{Attori}: Utente;
				\item \textbf{Descrizione}: L'utente sceglie di modificare un'interfaccia
				all'interno dell'editor del diagramma delle classi;
				\item \textbf{Precondizione}: Nell'editor del diagramma delle classi del
				sistema è stata selezionata un'interfaccia che l'utente desidera modificare;
				\item \textbf{Scenario principale degli eventi}:
					\begin{itemize}
						\item L'utente può rinominare l'interfaccia (UC);
						\item L'utente può aggiungere un'operazione (UC);
						\item L'utente può modificare un'operazione (UC);
						\item L'utente può rimuovere un'operazione (UC).
					\end{itemize}
				\item \textbf{Scenari alternativi}: Viene annullata la modifica, il sistema
				rimane nello stato precedente al tentativo di modifica;
				\item \textbf{Postcondizione}: Nell'editor del diagramma delle classi del
				sistema è visualizzato il diagramma dove sono state apportate le modifiche
				all'interfaccia.
			\end{itemize}
		\subsection{Caso d'uso UC : Rinominare un'interfaccia}
			\begin{itemize}
				\item \textbf{Attori}: Utente;
				\item \textbf{Descrizione}: L'utente cambia il nome dell'interfaccia;
				\item \textbf{Precondizione}: Il sistema è in attesa che l'utente inserisca
				una stringa per rinominare l'interfaccia;
				\item \textbf{Postcondizione}: Nell'editor del diagramma delle classi del
				sistema è visualizzato il diagramma dove è stato cambiato il nome
				all'interfaccia.
			\end{itemize}
		\subsection{Caso d'uso UC : Aggiungere un'operazione}
			\begin{itemize}
				\item \textbf{Attori}: Utente;
				\item \textbf{Descrizione}: L'utente ha scelto di aggiungere un'operazione
				all'interfaccia. L'utente deve definire la nuova operazione;
				\item \textbf{Precondizione}: L'utente desidera aggiungere un'operazione
				all'interfaccia selezionata dall'editor del diagramma delle classi del
				sistema. Il sistema è pronto ad aggiungere una nuova operazione;
				\item \textbf{Scenario principale degli eventi}:
					\begin{itemize}
						\item L'utente può impostare la visibilità (UC);
						\item L'utente può definire il nome dell'operazione (UC);
						\item L'utente può definire la lista parametri dell'operazione (UC);
						\item L'utente può definire il tipo di ritorno dell'operazione (UC);
						\item L'utente può definire proprietà aggiuntive dell'operazione (UC).
					\end{itemize}
				\item \textbf{Scenari alternativi}: Viene annullata la modifica, il sistema
				rimane nello stato precedente al tentativo di modifica;
				\item \textbf{Postcondizione}: Nell'editor del diagramma delle classi del
				sistema è visualizzato il diagramma dove è stata aggiunta la nuova operazione.
			\end{itemize}
		\subsection{Caso d'uso UC : Impostare la visibilità dell'operazione}
			\begin{itemize}
				\item \textbf{Attori}: Utente;
				\item \textbf{Descrizione}: L'utente può impostare la visibilità
				dell'operazione;
				\item \textbf{Precondizione}: Il sistema è in attesa che l'utente selezioni il
				tipo di visualizzazione da impostare all'interfaccia;
				\item \textbf{Postcondizione}: Il sistema ha impostato la visibilità
				dell'operazione.
			\end{itemize}
		\subsection{Caso d'uso UC : Definire il nome dell'operazione}
			\begin{itemize}
				\item \textbf{Attori}: Utente;
				\item \textbf{Descrizione}: L'utente può definire il nome dell'operazione;
				\item \textbf{Precondizione}: Il sistema è in attesa che l'utente inserisca
				una stringa per rinominare l'interfaccia;
				\item \textbf{Postcondizione}: Il sistema ha impostato il nome
				dell'operazione.
			\end{itemize}
		\subsection{Caso d'uso UC : Definire la lista parametri dell'operazione}
			\begin{itemize}
				\item \textbf{Attori}: Utente;
				\item \textbf{Descrizione}: L'utente può definire la lista parametri
				dell'operazione;
				\item \textbf{Precondizione}: Il sistema è in attesa che l'utente definisca
				la lista dei parametri dell'operazione;
				\item \textbf{Scenario principale degli eventi}:
					\begin{itemize}
						\item L'utente può aggiungere un parametro (UC);
						\item L'utente può modificare un parametro (UC);
						\item L'utente può rimuovere un parametro (UC).
					\end{itemize}
				\item \textbf{Postcondizione}: Il sistema ha impostato la lista parametri
				dell'operazione.
			\end{itemize}
		\subsection{Caso d'uso UC : Aggiungere un parametro}
			\begin{itemize}
				\item \textbf{Attori}: Utente;
				\item \textbf{Descrizione}: L'utente può aggiungere un parametro alla lista
				parametri dell'operazione;
				\item \textbf{Precondizione}: Il sistema è pronto per definire il parametro;
				\item \textbf{Scenario principale degli eventi}:
					\begin{itemize}
						\item L'utente può definire la direzione del parametro (UC);
						\item L'utente può definire il nome del parametro (UC);
						\item L'utente può definire il tipo del parametro (UC);
						\item L'utente può definire il valore di default del parametro (UC).
					\end{itemize}
				\item \textbf{Postcondizione}: Il sistema ha aggiunto il parametro alla lista.
			\end{itemize}
		\subsection{Caso d'uso UC : Definire la direzione del parametro}
			\begin{itemize}
				\item \textbf{Attori}: Utente;
				\item \textbf{Descrizione}: L'utente può definire la direzione di un parametro
				della lista parametri dell'operazione;
				\item \textbf{Precondizione}: Il sistema è in attesa che l'utente imposti
				la direzione del parametro;
				\item \textbf{Postcondizione}: Il sistema ha impostato la direzione al
				parametro della lista.
			\end{itemize}
		\subsection{Caso d'uso UC : Definire il nome del parametro}
			\begin{itemize}
				\item \textbf{Attori}: Utente;
				\item \textbf{Descrizione}: L'utente può definire il nome di un parametro
				della lista parametri dell'operazione;
				\item \textbf{Precondizione}: Il sistema è in attesa che l'utente inserisca
				il nome del parametro;
				\item \textbf{Postcondizione}: Il sistema ha impostato il nome al parametro
				della lista.
			\end{itemize}
		\subsection{Caso d'uso UC : Definire il tipo del parametro}
			\begin{itemize}
				\item \textbf{Attori}: Utente;
				\item \textbf{Descrizione}: L'utente può definire il tipo di un parametro
				della lista parametri dell'operazione;
				\item \textbf{Precondizione}: Il sistema è in attesa che l'utente inserisca
				il tipo del parametro;
				\item \textbf{Postcondizione}: Il sistema ha impostato il tipo al parametro
				della lista.
			\end{itemize}
		\subsection{Caso d'uso UC : Definire il valore di default del parametro}
			\begin{itemize}
				\item \textbf{Attori}: Utente;
				\item \textbf{Descrizione}: L'utente può definire il valore di default di un
				parametro della lista parametri dell'operazione;
				\item \textbf{Precondizione}: Il sistema è in attesa che l'utente imposti
				il valore di default del parametro;
				\item \textbf{Postcondizione}: Il sistema ha impostato il valore di default al
				parametro della lista.
			\end{itemize}
		\subsection{Caso d'uso UC : Modificare un parametro}
			\begin{itemize}
				\item \textbf{Attori}: Utente;
				\item \textbf{Descrizione}: L'utente può modificare un parametro della lista
				parametri dell'operazione;
				\item \textbf{Precondizione}: Il sistema è pronto per modificare il parametro;
				\item \textbf{Scenario principale degli eventi}:
					\begin{itemize}
						\item L'utente può definire la direzione del parametro (UC Sopra);
						\item L'utente può definire il nome del parametro (UC Sopra);
						\item L'utente può definire il tipo del parametro (UC Sopra);
						\item L'utente può definire il valore di default del parametro (UC Sopra).
					\end{itemize}
				\item \textbf{Postcondizione}: Il sistema ha modificato il parametro della
				lista.
			\end{itemize}
		\subsection{Caso d'uso UC : Rimuovere un parametro}
			\begin{itemize}
				\item \textbf{Attori}: Utente;
				\item \textbf{Descrizione}: L'utente può rimuovere un parametro della lista
				parametri dell'operazione;
				\item \textbf{Precondizione}: L'utente ha selezionato il parametro da
				rimuovere;
				\item \textbf{Postcondizione}: Il sistema ha rimosso il parametro dalla lista.
			\end{itemize}
		\subsection{Caso d'uso UC : Definire il tipo di ritorno dell'operazione}
			\begin{itemize}
				\item \textbf{Attori}: Utente;
				\item \textbf{Descrizione}: L'utente può inserire il tipo di ritorno
				dell'operazione;
				\item \textbf{Precondizione}: Il sistema è in attesa che l'utente inserisca
				il tipo di ritorno;
				\item \textbf{Postcondizione}: Il sistema ha impostato il tipo di ritorno
				all'operazione.
			\end{itemize}
		\subsection{Caso d'uso UC : Definire proprietà aggiuntive dell'operazione}
			\begin{itemize}
				\item \textbf{Attori}: Utente;
				\item \textbf{Descrizione}: L'utente può impostare proprietà aggiuntive
				all'operazione;
				\item \textbf{Precondizione}: Il sistema è pronto per ricevere proprietà
				aggiuntive;
				\item \textbf{Postcondizione}: Il sistema ha impostato le proprietà
				aggiuntive, scritte dall'utente, all'operazione.
			\end{itemize}
		\subsection{Caso d'uso UC : Modificare un'operazione}
			\begin{itemize}
				\item \textbf{Attori}: Utente;
				\item \textbf{Descrizione}: L'utente ha scelto di modificare un'operazione
				dell'interfaccia. L'utente deve selezionare l'operazione;
				\item \textbf{Precondizione}: L'utente desidera modificare un'operazione
				dell'interfaccia selezionata dall'editor del diagramma delle classi del
				sistema. È stata selezionata l'operazione da modificare;
				\item \textbf{Scenario principale degli eventi}:
					\begin{itemize}
						\item L'utente può impostare la visibilità (UC Sopra);
						\item L'utente può definire il nome dell'operazione (UC Sopra);
						\item L'utente può definire la lista parametri dell'operazione (UC Sopra);
						\item L'utente può definire il tipo di ritorno dell'operazione (UC Sopra);
						\item L'utente può definire proprietà aggiuntive dell'operazione (UC Sopra).
					\end{itemize}
				\item \textbf{Scenari alternativi}: Viene annullata la modifica, il sistema
				rimane nello stato precedente al tentativo di modifica;
				\item \textbf{Postcondizione}: Nell'editor del diagramma delle classi del
				sistema è visualizzato il diagramma dove è stata modificata l'operazione.
			\end{itemize}
		\subsection{Caso d'uso UC : Rimuovere un'operazione}
			\begin{itemize}
				\item \textbf{Attori}: Utente;
				\item \textbf{Descrizione}: L'utente può rimuovere un'operazione
				dell'interfaccia. L'utente deve selezionare l'operazione;
				\item \textbf{Precondizione}: L'utente desidera rimuovere un'operazione
				dell'interfaccia selezionata dall'editor del diagramma delle classi del
				sistema. È stata selezionata l'operazione da rimuovere;
				\item \textbf{Postcondizione}: Nell'editor del diagramma delle classi del
				sistema è visualizzato il diagramma dove è stata rimossa l'operazione.
			\end{itemize}
		\subsection{Caso d'uso UC : Rimuovere un'interfaccia}
			\begin{itemize}
				\item \textbf{Attori}: Utente;
				\item \textbf{Descrizione}: L'utente può rimuovere un'interfaccia nell'editor
				del diagramma delle classi. L'utente deve selezionare l'interfaccia;
				\item \textbf{Precondizione}: L'utente desidera rimuovere un'interfaccia
				selezionata dall'editor del diagramma delle classi del sistema;
				\item \textbf{Postcondizione}: Nell'editor del diagramma delle classi del
				sistema è visualizzato il diagramma dove è stata rimossa l'interfaccia.
			\end{itemize}
		\subsection{Caso d'uso UC : Definire la realizzazione di un'interfaccia}
			\begin{itemize}
				\item \textbf{Attori}: Utente;
				\item \textbf{Descrizione}: L'utente sceglie di definire un'associazione di
				realizzazione tra una classe ed un'interfaccia;
				\item \textbf{Precondizione}: Il sistema è pronto a creare l'associazione di
				realizzazione che l'utente desidera definire;
				\item \textbf{Scenario principale degli eventi}:
					\begin{itemize}
						\item L'utente può selezionare un'interfaccia (UC);
						\item L'utente può selezionare una classe (UC).
					\end{itemize}
				\item \textbf{Scenari alternativi}: La definizione della realizzazione di
				un'interfaccia viene annullata, il sistema rimane nello stato precedente al
				tentativo di modifica;
				\item \textbf{Postcondizione}: Nell'editor del diagramma delle classi del
				sistema è visualizzato il diagramma dove è stata definita la realizzazione.
			\end{itemize}
		\subsection{Caso d'uso UC : Selezionare un'interfaccia}
			\begin{itemize}
				\item \textbf{Attori}: Utente;
				\item \textbf{Descrizione}: L'utente seleziona l'interfaccia di interesse;
				\item \textbf{Precondizione}: Il sistema mostra il diagramma delle classi;
				\item \textbf{Postcondizione}: Il sistema evidenzia l'interfaccia selezionata
				dall'utente.
			\end{itemize}
		\subsection{Caso d'uso UC : Selezionare una classe}
			\begin{itemize}
				\item \textbf{Attori}: Utente;
				\item \textbf{Descrizione}: L'utente seleziona la classe di interesse;
				\item \textbf{Precondizione}: Il sistema mostra il diagramma delle classi;
				\item \textbf{Postcondizione}: Il sistema evidenzia la classe selezionata
				dall'utente.
			\end{itemize}
		\subsection{Caso d'uso UC : Rimuovere la realizzazione di un'interfaccia}
			\begin{itemize}
				\item \textbf{Attori}: Utente;
				\item \textbf{Descrizione}: L'utente sceglie di rimuovere un'associazione di
				realizzazione tra una classe ed un'interfaccia;
				\item \textbf{Precondizione}: Il sistema è pronto a rimuovere l'associazione
				di realizzazione che l'utente desidera;
				\item \textbf{Postcondizione}: Nell'editor del diagramma delle classi del
				sistema è visualizzato il diagramma dove è stata rimossa la realizzazione.
			\end{itemize}
			
%%%%%%%%%%%%%%%%%%%% LAYER DI VISUALIZZAZIONE %%%%%%%%%%%%%%%%%%%
		\subsection{Caso d'uso UC : Cambiare layer di visualizzazione}
			\begin{itemize}
				\item \textbf{Attori}: Utente;
				\item \textbf{Descrizione}: L'utente sceglie di cambiare il layer degli
				oggetti visualizzabili;
				\item \textbf{Precondizione}: Nell'editor del diagramma delle classi, sono
				presenti almeno due layer, l'utente desidera cambiare il layer di
				visualizzazione;
				\item \textbf{Postcondizione}: Nell'editor del diagramma delle classi del
				sistema sono visualizzati gli oggetti del diagramma appartenenti al layer
				selezionato.
			\end{itemize}
			
%%%%%%%%%%%%%%%%%%% SEZIONE PACKAGE %%%%%%%%%%%%%%%%%%%
		\subsection{Caso d'uso UC : Editare il diagramma dei package}
			\begin{itemize}
				\item \textbf{Attori}: Utente;
				\item \textbf{Descrizione}: L'utente sceglie di usare l'editor per il
				diagramma dei package;
				\item \textbf{Precondizione}: Il sistema è pronto all'utilizzo dell'editor per
				il diagramma dei package, l'utente desidera utilizzarlo;
				\item \textbf{Scenario principale degli eventi}:
					\begin{itemize}
						\item L'utente può creare un package (UC);
						\item L'utente può modificare un package (UC);
						\item L'utente può rimuovere un package (UC).
					\end{itemize}
				\item \textbf{Postcondizione}: L'utente genera un diagramma dei package
				attraverso l'editor.
			\end{itemize}
		\subsection{Caso d'uso UC : Creare un package}
			\begin{itemize}
				\item \textbf{Attori}: Utente;
				\item \textbf{Descrizione}: L'utente sceglie di creare un package;
				\item \textbf{Precondizione}: Il sistema è pronto alla creazione di package,
				l'utente desidera creare un package;
				\item \textbf{Postcondizione}: Nell'editor del diagramma dei package del
				sistema è visualizzato il diagramma dove è stato aggiunto il package creato.
			\end{itemize}
		\subsection{Caso d'uso UC : Modificare un package}
			\begin{itemize}
				\item \textbf{Attori}: Utente;
				\item \textbf{Descrizione}: L'utente sceglie di modificare un package
				all'interno dell'editor del diagramma dei package;
				\item \textbf{Precondizione}: Nell'editor del diagramma dei package del
				sistema è stato selezionato un package che l'utente desidera modificare;
				\item \textbf{Scenario principale degli eventi}:
					\begin{itemize}
						\item L'utente può definire il nome del package (UC);
						\item L'utente può impostare la visibilità del package (UC);
						\item L'utente può definire una dipendenza tra package (UC);
						\item L'utente può rimuovere una dipendenza tra package (UC);
						\item L'utente può innestare una classe nel package (UC);
						\item L'utente può rimuovere una classe dal package (UC);
						\item L'utente può innestare un'interfaccia nel package (UC);
						\item L'utente può rimuovere un'interfaccia dal package (UC);
						\item L'utente può innestare un package nel package (UC);
						\item L'utente può rimuovere un package dal package (UC);
					\end{itemize}
				\item \textbf{Scenari alternativi}: Viene annullata la modifica, il sistema
				rimane nello stato precedente al tentativo di modifica;
				\item \textbf{Postcondizione}: Nell'editor del diagramma dei package del
				sistema è visualizzato il diagramma dove sono state apportate le modifiche
				al package.
			\end{itemize}
		\subsection{Caso d'uso UC : Definire il nome del package}
			\begin{itemize}
				\item \textbf{Attori}: Utente;
				\item \textbf{Descrizione}: L'utente può dare un nome ad un package;
				\item \textbf{Precondizione}: Il sistema è in attesa che l'utente inserisca
				una stringa per rinominare il package;
				\item \textbf{Postcondizione}: Nell'editor del diagramma dei package del
				sistema è visualizzato il diagramma dove è stato cambiato il nome
				al package.
			\end{itemize}
		\subsection{Caso d'uso UC : Impostare la visibilità del package}
			\begin{itemize}
				\item \textbf{Attori}: Utente;
				\item \textbf{Descrizione}: L'utente può impostare la visibilità
				del package;
				\item \textbf{Precondizione}: Il sistema è in attesa che l'utente selezioni il
				tipo di visualizzazione da impostare al package;
				\item \textbf{Postcondizione}: Il sistema ha impostato la visibilità
				del package.
			\end{itemize}
		\subsection{Caso d'uso UC : Definire dipendenza tra package}
			\begin{itemize}
				\item \textbf{Attori}: Utente;
				\item \textbf{Descrizione}: L'utente sceglie di definire una dipendenza tra
				due package;
				\item \textbf{Precondizione}: Il sistema è pronto a creare la dipendenza che
				l'utente desidera definire;
				\item \textbf{Scenario principale degli eventi}:
					\begin{itemize}
						\item L'utente può selezionare un package (il package con dipendenza
						uscente) (UC);
						\item L'utente può selezionare un package (il package con dipendenza
						entrante) (UC).
					\end{itemize}
				\item \textbf{Scenari alternativi}: La definizione della dipendenza viene
				annullata, il sistema rimane nello stato precedente al tentativo di modifica;
				\item \textbf{Postcondizione}: Nell'editor del diagramma dei package del
				sistema è visualizzato il diagramma dove è stata definita la dipendenza.
			\end{itemize}
		\subsection{Caso d'uso UC : Selezionare un package}
			\begin{itemize}
				\item \textbf{Attori}: Utente;
				\item \textbf{Descrizione}: L'utente seleziona il package di interesse;
				\item \textbf{Precondizione}: Il sistema mostra il diagramma dei package;
				\item \textbf{Postcondizione}: Il sistema evidenzia il package selezionato
				dall'utente.
			\end{itemize}
		\subsection{Caso d'uso UC : Rimuovere dipendenza tra package}
			\begin{itemize}
				\item \textbf{Attori}: Utente;
				\item \textbf{Descrizione}: L'utente sceglie di rimuovere una dipendenza tra
				due package;
				\item \textbf{Precondizione}: Il sistema è pronto a rimuovere la dipendenza
				che l'utente desidera;
				\item \textbf{Postcondizione}: Nell'editor del diagramma dei package del
				sistema è visualizzato il diagramma dove è stata rimossa la dipendenza.
			\end{itemize}
		\subsection{Caso d'uso UC : Innestare una classe nel package}
			\begin{itemize}
				\item \textbf{Attori}: Utente;
				\item \textbf{Descrizione}: L'utente sceglie di innestare una classe
				all'interno di un package;
				\item \textbf{Precondizione}: Il sistema è pronto per effettuare l'operazione;
				\item \textbf{Scenario principale degli eventi}:
					\begin{itemize}
						\item L'utente può selezionare una classe (UC Sopra);
						\item L'utente può selezionare un package (UC Sopra);
					\end{itemize}
				\item \textbf{Scenari alternativi}: L'innesto viene annullato, il sistema
				rimane nello stato precedente al tentativo di modifica;
				\item \textbf{Postcondizione}: Nell'editor del diagramma dei package del
				sistema è visualizzato il diagramma dove è stato effettuato l'innesto.
			\end{itemize}
		\subsection{Caso d'uso UC : Rimuovere una classe dal package}
			\begin{itemize}
				\item \textbf{Attori}: Utente;
				\item \textbf{Descrizione}: L'utente può rimuovere una classe da un package
				nell'editor del diagramma dei package. L'utente deve selezionare
				la classe;
				\item \textbf{Precondizione}: L'utente desidera rimuovere una classe
				selezionata da un package nell'editor del diagramma dei package del sistema;
				\item \textbf{Postcondizione}: Nell'editor del diagramma dei package del
				sistema è visualizzato il diagramma dove è stata rimossa la classe.
			\end{itemize}
			
\end{document}
