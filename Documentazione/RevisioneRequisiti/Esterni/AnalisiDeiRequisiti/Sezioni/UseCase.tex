\documentclass[../AnalisiDeiRequisiti.tex]{subfiles}
\begin{document}
	\section{Use Case}
	\subsection{Caso d'uso UC1: Gestire un progetto}
	\begin{itemize}
		\item \textbf{Attori}: Utente
		\item \textbf{Descrizione}: L'utente vuole gestire il progetto su cui lavorare;
		\item \textbf{Precondizione}: Il programma si è avviato correttamente ed è pronto a ricevere un input dall'utente;
		\item \textbf{Flusso principale degli eventi}: \begin{itemize}
			\item L'utente può creare un nuovo progetto (UC1.1);
			\item L'utente può caricare un progetto (UC1.2);
			\item L'utente può chiudere il progetto attuale (UC1.3);
			\item L'utente può salvare il progetto attuale (UC1.4);
			\item L'utente può annullare l'ultimo comando eseguito (UC1.5);
			\item L'utente può ripristinare l'ultimo comando annullato (UC1.6);
			\item L'utente può leggere il codice prodotto (UC1.7);
			\item L'utente può esportare il codice prodotto (UC1.8);
		\end{itemize}
		\item \textbf{Postcondizione}: È stata completata l'operazione sul progetto desiderata.
	\end{itemize}
	\subsection{Caso d'uso UC1.1: Creare un nuovo progetto}
	\begin{itemize}
		\item \textbf{Attori}: Utente
		\item \textbf{Descrizione}: L'utente vuole creare un nuovo progetto;
		\item \textbf{Precondizione}: Il programma si è avviato correttamente ed è pronto a ricevere un input dall'utente;
		\item \textbf{Flusso principale degli eventi}: L'utente crea un nuovo progetto;
		\item \textbf{Postcondizione}: È stato creato un nuovo progetto che è pronto ad essere modificato.
	\end{itemize}
	\subsection{Caso d'uso UC1.2: Caricare un progetto}
	\begin{itemize}
		\item \textbf{Attori}: Utente
		\item \textbf{Descrizione}: L'utente vuole caricare un progetto precedentemente creato;
		\item \textbf{Precondizione}: Il programma si è avviato correttamente ed è pronto a ricevere un input dall'utente;
		\item \textbf{Flusso principale degli eventi}: L'utente carica un progetto precedentemente creato;
		\item \textbf{Postcondizione}: È stato caricato un nuovo progetto che è pronto ad essere modificato.
	\end{itemize}
	\subsection{Caso d'uso UC1.3: Chiudere il progetto attuale}
	\begin{itemize}
		\item \textbf{Attori}: Utente
		\item \textbf{Descrizione}: L'utente vuole chiudere il progetto correntemente aperto;
		\item \textbf{Precondizione}: Il programma è in attesa di un comando dall'utente e ha un progetto aperto;
		\item \textbf{Flusso principale degli eventi}: L'utente chiude il progetto correntemente aperto;
		\item \textbf{Postcondizione}: Il progetto aperto in precedenza è stato chiuso.
	\end{itemize}
	\subsection{Caso d'uso UC1.4: Salvare il progetto attuale}
	\begin{itemize}
		\item \textbf{Attori}: Utente
		\item \textbf{Descrizione}: L'utente vuole salvare il lavoro fatto fino a quel momento;
		\item \textbf{Precondizione}: Nelle schermate degli editor messi a disposizione del programma sono stati disegnati i diagrammi che rappresentano il codice desiderato;
		\item \textbf{Flusso principale degli eventi}: L'utente salva il lavoro fatto fino a quel momento;
		\item \textbf{Postcondizione}: In una cartella a scelta dell'utente il programma ha generato un file contenente tutte le informazioni necessarie per ripristinarne lo stato attuale.
	\end{itemize}
	\subsection{Caso d'uso UC1.5: Annullare l'ultimo comando eseguito}
	\begin{itemize}
		\item \textbf{Attori}: Utente
		\item \textbf{Descrizione}: L'utente vuole annullare l'effetto dell'ultimo comando eseguito nell'editor del diagramma correntemente in uso;
		\item \textbf{Precondizione}: L'utente sta utilizzando l'editor di un diagramma tra quelli disponibili, ha eseguito almeno un comando e il sistema lo ha memorizzato;
		\item \textbf{Flusso principale degli eventi}: L'utente annulla l'effetto dell'ultimo comando eseguito nell'editor del diagramma correntemente in uso;
		\item \textbf{Postcondizione}: Il sistema ha ripristinato lo stato in cui si trovava il diagramma, correntemente in uso, prima che venisse eseguito il comando che è stato annullato; Il sistema ha memorizzato tale comando.
	\end{itemize}
	\subsection{Caso d'uso UC1.6: Ripristinare l'ultimo comando annullato}
	\begin{itemize}
		\item \textbf{Attori}: Utente
		\item \textbf{Descrizione}: L'utente vuole ripristinare l'effetto dell'ultimo comando precedentemente annullato nell'editor del diagramma correntemente in uso;
		\item \textbf{Precondizione}: Il programma è in esecuzione con un progetto aperto ed è appena stato annullato un comando;
		\item \textbf{Flusso principale degli eventi}: L'utente ripristina l'effetto dell'ultimo comando precedentemente annullato nell'editor del diagramma correntemente in uso;
		\item \textbf{Postcondizione}: Il programma è tornato nello stato precedente all'annullamento.
	\end{itemize}
	\subsection{Caso d'uso UC1.7: Leggere il codice prodotto}
	\begin{itemize}
		\item \textbf{Attori}: Utente
		\item \textbf{Descrizione}: L'utente vuole leggere il codice;
		\item \textbf{Precondizione}: Il sistema è pronto a mostrare il codice prodotto e in attesa di un comando da parte dell'utente;
		\item \textbf{Flusso principale degli eventi}: L'utente legge il codice;
		\item \textbf{Postcondizione}: Nella schermata del visualizzatore del codice è mostrato il codice prodotto.
	\end{itemize}
	\subsection{Caso d'uso UC1.8: Esportare il codice prodotto}
	\begin{itemize}
		\item \textbf{Attori}: Utente
		\item \textbf{Descrizione}: L'utente vuole esportare il codice generato nei file sorgente appropriati per il linguaggio corrispondente;
		\item \textbf{Precondizione}: Nelle schermate degli editor messi a disposizione del programma sono stati disegnati i diagrammi che rappresentano il codice desiderato;
		\item \textbf{Flusso principale degli eventi}: L'utente esporta il codice generato nei file sorgente appropriati per il linguaggio corrispondente;
		\item \textbf{Postcondizione}: In una cartella a scelta dell'utente il programma ha generato tutti i file sorgenti voluti, organizzati secondo quanto specificato dall'utente tramite i diagrammi. Questi file contengono codice corretto e compilabile. Qualora il programma non avesse potuto tradurre efficacemente una parte del diagramma dell'utente, il programma ha comunicato un avvertimento all'utente e commentato opportunamente il codice nel sorgente;
	\end{itemize}
	\subsection{Caso d'uso UC2: Editare il diagramma delle classi}
	\begin{itemize}
		\item \textbf{Attori}: Utente
		\item \textbf{Descrizione}: L'utente ha avviato correttamente il programma e ha aperto un progetto. Ora l'utente può editare il diagramma delle classi;
		\item \textbf{Precondizione}: L'utente ha avviato correttamente il programma e ha aperto un progetto;
		\item \textbf{Flusso principale degli eventi}: \begin{itemize}
			\item L'utente può creare una classe (UC2.1);
			\item L'utente può modificare una classe (UC2.2);
			\item L'utente può eliminare una classe (UC2.3);
			\item L'utente può definire una relazione (UC2.4);
			\item L'utente può modificare una relazione (UC2.5);
			\item L'utente può eliminare una relazione (UC2.6);
			\item L'utente può creare una interfaccia (UC2.7);
			\item L'utente può modificare una interfaccia (UC2.8);
			\item L'utente può eliminare una interfaccia (UC2.9);
			\item L'utente può creare un commento (UC2.10);
			\item L'utente può collegare un commento (UC2.11);
			\item L'utente può modificare un commento (UC2.12);
			\item L'utente può eliminare un commento (UC2.13);
			\item L'utente può aprire l'editor del diagramma dei package (UC2.14);
			\item L'utente può riposizionare un elemento (UC2.15);
		\end{itemize}
		\item \textbf{Scenari alternativi}: Viene annullata la modifica, il sistema rimane nello stato precedente al tentativo di modifica;
		\item \textbf{Postcondizione}: Il sistema apporta le modifiche desiderate al diagramma delle classi.
	\end{itemize}
	\subsection{Caso d'uso UC2.1: Creare una classe}
	\begin{itemize}
		\item \textbf{Attori}: Utente
		\item \textbf{Descrizione}: L'utente può aggiungere una nuova classe vuota al diagramma delle classi;
		\item \textbf{Precondizione}: Il programma è in esecuzione con un progetto aperto nel diagramma delle classi;
		\item \textbf{Flusso principale degli eventi}: L'utente aggiunge una nuova classe vuota al diagramma delle classi;
		\item \textbf{Postcondizione}: Viene aggiunta una nuova classe al diagramma delle classi.
	\end{itemize}
	\subsection{Caso d'uso UC2.2: Modificare una classe}
	\begin{itemize}
		\item \textbf{Attori}: Utente
		\item \textbf{Descrizione}: L'utente vuole apportare modifiche minori ad una classe;
		\item \textbf{Precondizione}: L'utente ha avviato il programma, sta visualizzando il diagramma delle classi e ha selezionato la classe che vuole modificare;
		\item \textbf{Flusso principale degli eventi}: \begin{itemize}
			\item L'utente apporta le modifiche minori desiderate;
			\item L'utente può aprire la schermata di modifica di classe corrispondente (UC2.2.1);
		\end{itemize}
		\item \textbf{Postcondizione}: Le modifiche vengono applicate alla classe nel diagramma delle classi.
	\end{itemize}
	\subsection{Caso d'uso UC2.2.1: Aprire la schermata di modifica di classe corrispondente}
	\begin{itemize}
		\item \textbf{Attori}: Utente
		\item \textbf{Descrizione}: L'utente vuole modificare nel dettaglio una classe;
		\item \textbf{Precondizione}: L'utente ha avviato il programma, ha aperto il diagramma delle classi e ha selezionato la classe che vuole modificare;
		\item \textbf{Flusso principale degli eventi}: L'utente modifica nel dettaglio una classe;
		\item \textbf{Postcondizione}: Il sistema visualizza la schermata di modifica di classe corrispondente alla classe selezionata dall'utente.
	\end{itemize}
	\subsection{Caso d'uso UC2.3: Eliminare una classe}
	\begin{itemize}
		\item \textbf{Attori}: Utente
		\item \textbf{Descrizione}: L'utente vuole eliminare una classe;
		\item \textbf{Precondizione}: Esiste una classe che l'utente desidera eliminare;
		\item \textbf{Flusso principale degli eventi}: L'utente elimina una classe;
		\item \textbf{Postcondizione}: La classe non viene più visualizzata nell'editor del diagramma delle classi.
	\end{itemize}
	\subsection{Caso d'uso UC2.4: Definire una relazione}
	\begin{itemize}
		\item \textbf{Attori}: Utente
		\item \textbf{Descrizione}: L'utente vuole definire una relazione tra due elementi.
		\item \textbf{Precondizione}: Sono presenti due elementi e l'utente desidera che presentino una relazione l'una dall'altra.
		\item \textbf{Flusso principale degli eventi}: \begin{itemize}
			\item L'utente vuole definire la dipendenza tra due elementi (UC2.4.1);
			\item L'utente vuole definire l'associazione tra due elementi (UC2.4.2);
			\item L'utente vuole definire l'ereditarietà  tra due elementi (UC2.4.3);
			\item L'utente vuole definire l'aggregazione tra due elementi (UC2.4.4);
			\item L'utente vuole definire la composizione tra due elementi (UC2.4.5);
			\item L'utente vuole definire il raffinamento di una classe parametrica (UC2.4.6);
			\item L'utente vuole definire la realizzazione di un'interfaccia (UC2.4.7).
		\end{itemize}
		\item \textbf{Postcondizione}: I due elementi vengono messi in relazione.
	\end{itemize}
		\subsection{Caso d'uso UC2.4.1: Definire la dipendenza tra due elementi}
		\begin{itemize}
			\item \textbf{Attori}: Utente
			\item \textbf{Descrizione}: L'utente vuole definire la dipendenza tra due elementi;
			\item \textbf{Precondizione}: Sono presenti due elementi e l'utente vuole evidenziarne la dipendenza;
			\item \textbf{Flusso principale degli eventi}: L'utente definisce la dipendenza tra due elementi;
			\item \textbf{Postcondizione}: La dipendenza tra le due elementi è stata definita.
		\end{itemize}
		\subsection{Caso d'uso UC2.4.2: Definire associazione tra due elementi}
		\begin{itemize}
			\item \textbf{Attori}: Utente
			\item \textbf{Descrizione}: L'utente vuole definire un'associazione tra due elementi;
			\item \textbf{Precondizione}: Sono presenti due elementi e l'utente vuole evidenziarne l'associazione;
			\item \textbf{Flusso principale degli eventi}: L'utente definisce un'associazione tra due elementi;
			\item \textbf{Postcondizione}: L'associazione tra le due elementi è stata definita.
		\end{itemize}
		\subsection{Caso d'uso UC2.4.3: Definire ereditarietà  tra due elementi}
		\begin{itemize}
			\item \textbf{Attori}: Utente
			\item \textbf{Descrizione}: L'utente vuole definire un vincolo di ereditarietà  tra due elementi;
			\item \textbf{Precondizione}: Sono presenti due elementi e l'utente vuole evidenziarne il vincolo di ereditarietà ;
			\item \textbf{Flusso principale degli eventi}: L'utente definisce un vincolo di ereditarietà  tra due elementi;
			\item \textbf{Postcondizione}: L'ereditarietà  tra le due elementi è stata definita.
		\end{itemize}
		\subsection{Caso d'uso UC2.4.4: Definire aggregazione tra due elementi}
		\begin{itemize}
			\item \textbf{Attori}: Utente
			\item \textbf{Descrizione}: L'utente vuole definire un vincolo di aggregazione tra due elementi;
			\item \textbf{Precondizione}: Sono presenti due elementi e l'utente vuole evidenziarne il vincolo di aggregazione;
			\item \textbf{Flusso principale degli eventi}: L'utente definisce un vincolo di aggregazione tra due elementi;
			\item \textbf{Postcondizione}: L'aggregazione tra le due elementi è stata definita.
		\end{itemize}
		\subsection{Caso d'uso UC2.4.5: Definire la composizione tra due elementi}
		\begin{itemize}
			\item \textbf{Attori}: Utente
			\item \textbf{Descrizione}: L'utente vuole definire una composizione tra due elementi;
			\item \textbf{Precondizione}: Sono presenti due elementi e l'utente vuole evidenziarne la composizione;
			\item \textbf{Flusso principale degli eventi}: L'utente definisce una composizione tra due elementi;
			\item \textbf{Postcondizione}: La relazione di composizione tra i due elementi è stata definita.
		\end{itemize}
		\subsection{Caso d'uso UC2.4.6: Definire il raffinamento di una classe parametrica}
		\begin{itemize}
			\item \textbf{Attori}: Utente
			\item \textbf{Descrizione}: L'utente vuole definire il raffinamento di una classe parametrica;
			\item \textbf{Precondizione}: L'utente si trova nella schermata dell'editor del diagramma delle classi ha selezionato la classe parametrica che desidera raffinare;
			\item \textbf{Flusso principale degli eventi}: L'utente definisce il raffinamento di una classe parametrica;
			\item \textbf{Postcondizione}: La classe parametrica viene raffinata.
		\end{itemize}
		\subsection{Caso d'uso UC2.4.7: Definire la realizzazione di un'interfaccia}
		\begin{itemize}
			\item \textbf{Attori}: Utente
			\item \textbf{Descrizione}: L'utente vuole inserire la relazione di realizzazione tra  un'interfaccia e una classe all'interno del diagramma delle classi;
			\item \textbf{Precondizione}: L'utente sta visualizzando il diagramma delle classi e sono presenti un'interfaccia e una classe ;
			\item \textbf{Flusso principale degli eventi}: L'utente inserisce la relazione di realizzazione tra  un'interfaccia e una classe all'interno del diagramma delle classi;
			\item \textbf{Postcondizione}: Nell'editor del diagramma delle classi del sistema è visualizzato il diagramma dove è stata definita la realizzazione.
		\end{itemize}
		\subsection{Caso d'uso UC2.5: Modificare una relazione}
		\begin{itemize}
			\item \textbf{Attori}: Utente
			\item \textbf{Descrizione}: L'utente vuole modificare una relazione tra due elementi;
			\item \textbf{Precondizione}: È presente una relazione che l'utente vuole modificare;
			\item \textbf{Flusso principale degli eventi}: \begin{itemize}
				\item L'utente vuole modificare la dipendenza tra elementi (UC2.5.1).
				\item L'utente vuole modificare l'associazione tra due elementi (UC2.5.2).
				\item L'utente vuole modificare l'ereditarietà  tra due elementi (UC2.5.3).
				\item L'utente vuole modificare l'aggregazione tra due elementi (UC2.5.4).
				\item L'utente vuole modificare la composizione tra due elementi (UC2.5.5).
				\item L'utente vuole modificare il raffinamento di una classe parametrica (UC2.5.6).
				\item L'utente vuole modificare la realizzazione di un'interfaccia (UC2.5.7).
			\end{itemize}
			\item \textbf{Postcondizione}: .
		\end{itemize}
		\subsection{Caso d'uso UC2.5.1: Modificare la dipendenza tra due elementi}
		\begin{itemize}
			\item \textbf{Attori}: Utente
			\item \textbf{Descrizione}: L'utente vuole modificare la dipendenza tra due elementi;
			\item \textbf{Precondizione}: Sono presenti due elementi che hanno una relazione di dipendenza e l'utente vuole modificare questa relazione;
			\item \textbf{Flusso principale degli eventi}: L'utente modifica la dipendenza tra due elementi;
			\item \textbf{Postcondizione}: La dipendenza tra le due elementi è stata modificata.
		\end{itemize}
		\subsection{Caso d'uso UC2.5.2: Modificare l'associazione tra due elementi}
		\begin{itemize}
			\item \textbf{Attori}: Utente
			\item \textbf{Descrizione}: L'utente vuole modificare un'associazione tra due elementi;
			\item \textbf{Precondizione}: Sono presenti due elementi che hanno una relazione di associazione e l'utente vuole modificarne la relazione;
			\item \textbf{Flusso principale degli eventi}: L'utente modifica un'associazione tra due elementi;
			\item \textbf{Postcondizione}: L'associazione tra le due elementi è stata modificata.
		\end{itemize}
		\subsection{Caso d'uso UC2.5.3: Modificare l'ereditarietà  tra due elementi}
		\begin{itemize}
			\item \textbf{Attori}: Utente
			\item \textbf{Descrizione}: L'utente vuole modificare una relazione di ereditarietà  tra due elementi;
			\item \textbf{Precondizione}: Sono presenti due elementi che hanno una relazione di ereditarietà  e l'utente vuole modificare questa relazione;
			\item \textbf{Flusso principale degli eventi}: L'utente modifica una relazione di ereditarietà  tra due elementi;
			\item \textbf{Postcondizione}: L'ereditarietà  tra le due elementi è stata modificata.
		\end{itemize}
		\subsection{Caso d'uso UC2.5.4: Modificare l'aggregazione tra due elementi}
		\begin{itemize}
			\item \textbf{Attori}: Utente
			\item \textbf{Descrizione}: L'utente vuole modificare un vincolo di aggregazione tra due elementi;
			\item \textbf{Precondizione}: Sono presenti due elementi che hanno un vinscolo di aggregazione e l'utente vuole modificare la relazione;
			\item \textbf{Flusso principale degli eventi}: L'utente modifica un vincolo di aggregazione tra due elementi;
			\item \textbf{Postcondizione}: L'aggregazione tra le due elementi è stata definita.
		\end{itemize}
		\subsection{Caso d'uso UC2.5.5: Modificare la composizione tra due elementi}
		\begin{itemize}
			\item \textbf{Attori}: Utente
			\item \textbf{Descrizione}: L'utente vuole modificare una composizione tra due elementi;
			\item \textbf{Precondizione}: Sono presenti due elementi con relazione di composizione e l'utente vuole modificarne la relazione;
			\item \textbf{Flusso principale degli eventi}: L'utente modifica una composizione tra due elementi;
			\item \textbf{Postcondizione}: La relazione di composizione tra i due elementi è stata definita.	
		\end{itemize}
		\subsection{Caso d'uso UC2.5.6: Modificare il raffinamento di una classe parametrica}
		\begin{itemize}
			\item \textbf{Attori}: Utente
			\item \textbf{Descrizione}: L'utente vuole modificare il raffinamento di una classe parametrica;
			\item \textbf{Precondizione}: L'utente si trova nella schermata dell'editor del diagramma delle classi ha selezionato il raffinamento di una classe parametrica che desidera modificare;
			\item \textbf{Flusso principale degli eventi}: L'utente modifica il raffinamento di una classe parametrica;
			\item \textbf{Postcondizione}: La relazione di raffinamento viene modificata.
		\end{itemize}
		\subsection{Caso d'uso UC2.5.7: Modificare la realizzazione di un'interfaccia}
		\begin{itemize}
			\item \textbf{Attori}: Utente
			\item \textbf{Descrizione}: L'utente vuole modificare la relazione di realizzazione tra un'interfaccia e una classe all'interno del diagramma delle classi;
			\item \textbf{Precondizione}: L'utente sta visualizzando il diagramma delle classi e sono presenti un'interfaccia e una classe che la realizza;
			\item \textbf{Flusso principale degli eventi}: L'utente modifica la relazione di realizzazione tra un'interfaccia e una classe all'interno del diagramma delle classi;
			\item \textbf{Postcondizione}: Nell'editor del diagramma delle classi del sistema è visualizzato il diagramma dove è stata modificatala realizzazione.
		\end{itemize}
		\subsection{Caso d'uso UC2.6: Eliminare una relazione}
		\begin{itemize}
			\item \textbf{Attori}: Utente
			\item \textbf{Descrizione}: L'utente vuole eliminare una relazione;
			\item \textbf{Precondizione}: Esiste una relazione che l'utente desidera eliminare;
			\item \textbf{Flusso principale degli eventi}: \begin{itemize}
				\item L'utente vuole eliminare la dipendenza tra due elementi (UC2.6.1);
				\item L'utente vuole eliminare l'associazione tra due elementi (UC2.6.2);
				\item L'utente vuole eliminare l'ereditarietà  tra due elementi (UC2.6.3);
				\item L'utente vuole eliminare l'aggregazione tra due elementi(UC2.6.4);
				\item L'utente vuole eliminare la composizione tra due elementi (UC2.6.5);
				\item L'utente vuole eliminare il raffinamenti di una classe parametrica (UC2.6.6);
				\item L'utente vuole eliminare la realizzazione di un'interfaccia (UC2.6.7).
			\end{itemize}
			\item \textbf{Postcondizione}: La relazione viene eliminata.
		\end{itemize}
		\subsection{Caso d'uso UC2.6.1: Eliminare la dipendenza tra due elementi }
		\begin{itemize}
			\item \textbf{Attori}: Utente
			\item \textbf{Descrizione}: L'utente vuole eliminare una dipendenza tra due elementi;
			\item \textbf{Precondizione}: Esiste una dipendenza tra classi che l'utente desidera eliminare;
			\item \textbf{Flusso principale degli eventi}: L'utente elimina una dipendenza tra due elementi;
			\item \textbf{Postcondizione}: La dipendenza tra classi viene eliminata.
		\end{itemize}
		\subsection{Caso d'uso UC2.6.2: Eliminare l'associazione tra due elementi}
		\begin{itemize}
			\item \textbf{Attori}: Utente
			\item \textbf{Descrizione}: L'utente vuole eliminare un'associazione tra due elementi;
			\item \textbf{Precondizione}: Sono presenti due elementi che hanno una relazione di associazione che l'utente vuole eliminare la relazione;
			\item \textbf{Flusso principale degli eventi}: L'utente elimina un'associazione tra due elementi;
			\item \textbf{Postcondizione}: L'associazione tra le due elementi è stata eliminata.
		\end{itemize}
		\subsection{Caso d'uso UC2.6.3: Eliminare l'ereditarietà  tra due elementi}
		\begin{itemize}
			\item \textbf{Attori}: Utente
			\item \textbf{Descrizione}: L'utente vuole eliminare l'ereditarietà  tra due elementi;
			\item \textbf{Precondizione}: Sono presenti due elementi che hanno una relazione di ereditarietà  che l'utente vuole eliminare;
			\item \textbf{Flusso principale degli eventi}: L'utente elimina l'ereditarietà  tra due elementi;
			\item \textbf{Postcondizione}: L'ereditarietà  tra le due elementi è stata eliminata.
		\end{itemize}
		\subsection{Caso d'uso UC2.6.4: Eliminare l'aggregazione tra due elementi}
		\begin{itemize}
			\item \textbf{Attori}: Utente
			\item \textbf{Descrizione}: L'utente vuole eliminare un vincolo di aggregazione tra due elementi;
			\item \textbf{Precondizione}: Sono presenti due elementi che hanno un vinscolo di aggregazione che l'utente vuole eliminare;
			\item \textbf{Flusso principale degli eventi}: L'utente elimina un vincolo di aggregazione tra due elementi;
			\item \textbf{Postcondizione}: L'aggregazione tra le due elementi è stata eliminata.
		\end{itemize}
		\subsection{Caso d'uso UC2.6.5: Eliminare la composizione tra due elementi}
		\begin{itemize}
			\item \textbf{Attori}: Utente
			\item \textbf{Descrizione}: L'utente vuole eliminare una composizione tra due elementi;
			\item \textbf{Precondizione}: Sono presenti due elementi con relazione di composizione che l'utente vuole eliminare;
			\item \textbf{Flusso principale degli eventi}: L'utente elimina una composizione tra due elementi;
			\item \textbf{Postcondizione}: La relazione di composizione tra i due elementi è stata eliminata.
		\end{itemize}
		\subsection{Caso d'uso UC2.6.6: Eliminare il raffinamento di una classe parametrica}
		\begin{itemize}
			\item \textbf{Attori}: Utente
			\item \textbf{Descrizione}: L'utente vuole eliminare il raffinamento di una classe parametrica;
			\item \textbf{Precondizione}: L'utente si trova nella schermata dell'editor del diagramma delle classi e ha selezionato il raffinamento di una classe parametrica che desidera eliminare;
			\item \textbf{Flusso principale degli eventi}: L'utente elimina il raffinamento di una classe parametrica;
			\item \textbf{Postcondizione}: La relazione di raffinamento viene eliminata.
		\end{itemize}
		\subsection{Caso d'uso UC2.6.7: Eliminare la realizzazione di un'interfaccia}
		\begin{itemize}
			\item \textbf{Attori}: Utente
			\item \textbf{Descrizione}: L'utente vuole eliminare la relazione di realizzazione tra un'interfaccia e una classe all'interno del diagramma delle classi;
			\item \textbf{Precondizione}: L'utente sta visualizzando il diagramma delle classi e sono presenti un'interfaccia e una classe che la realizza;
			\item \textbf{Flusso principale degli eventi}: L'utente elimina la relazione di realizzazione tra un'interfaccia e una classe all'interno del diagramma delle classi;
			\item \textbf{Postcondizione}: Nell'editor del diagramma delle classi del sistema è visualizzato il diagramma dove è stata eliminata realizzazione.
		\end{itemize}
		\subsection{Caso d'uso UC2.7: Creare una interfaccia}
		\begin{itemize}
			\item \textbf{Attori}: Utente
			\item \textbf{Descrizione}: L'utente vuole creare un'interfaccia;
			\item \textbf{Precondizione}: Il sistema è pronto alla creazione di un'interfaccia, l'utente desidera creare un'interfaccia;
			\item \textbf{Flusso principale degli eventi}: L'utente crea un'interfaccia;
			\item \textbf{Postcondizione}: Nell'editor del diagramma delle classi l'interfaccia è correttamente visualizzato il diagramma nel quale è stata creata l'interfaccia.
		\end{itemize}
		\subsection{Caso d'uso UC2.8: Modificare una interfaccia}
		\begin{itemize}
			\item \textbf{Attori}: Utente
			\item \textbf{Descrizione}: L'utente sceglie di modificare un'interfaccia all'interno dell'editor del diagramma delle classi;
			\item \textbf{Precondizione}: Nell'editor del diagramma delle classi è stata selezionata l'interfaccia che l'utente desidera modificare;
			\item \textbf{Flusso principale degli eventi}: \begin{itemize}
				\item L'utente può apportare modifiche minori all'interfaccia;
				\item L'utente può aprire la schermata di modifica interfaccia (UC2.8.1);
			\end{itemize}
			\item \textbf{Scenari alternativi}: Viene annullata la modifica, l'interfaccia rimane nello stato precedente al tentativo di modifica;
			\item \textbf{Postcondizione}: Nell'editor del diagramma delle classi è visualizzato il diagramma dove sono state apportate le modifiche all'interfaccia.
		\end{itemize}
		\subsection{Caso d'uso UC2.8.1: Aprire la schermata di modifica di interfaccia corrispondente}
		\begin{itemize}
			\item \textbf{Attori}: Utente
			\item \textbf{Descrizione}: L'utente vuole modificare nel dettaglio un'interfaccia;
			\item \textbf{Precondizione}: L'utente ha avviato il programma, ha aperto il diagramma delle classi e ha selezionato l'interfaccia che vuole modificare;
			\item \textbf{Flusso principale degli eventi}: L'utente modifica nel dettaglio un'interfaccia;
			\item \textbf{Postcondizione}: Il sistema visualizza la schermata di modifica di interfaccia corrispondente all'interfaccia selezionata dall'utente.
		\end{itemize}
		\subsection{Caso d'uso UC2.9: Eliminare una interfaccia}
		\begin{itemize}
			\item \textbf{Attori}: Utente
			\item \textbf{Descrizione}: L'utente vuole eliminare un'interfaccia dal diagramma delle classi;
			\item \textbf{Precondizione}: L'utente ha selezionato l'interfaccia che vuole rimuovere;
			\item \textbf{Flusso principale degli eventi}: L'utente elimina un'interfaccia dal diagramma delle classi;
			\item \textbf{Postcondizione}: Nell'editor del diagramma delle classi è visualizzato il diagramma dove è stata rimossa l'interfaccia.
		\end{itemize}
		\subsection{Caso d'uso UC2.10: Creare un commento}
		\begin{itemize}
			\item \textbf{Attori}: Utente
			\item \textbf{Descrizione}: L'utente vuole creare commento all'interno del diagramma delle classi;
			\item \textbf{Precondizione}: L'utente ha avviato il programma aperto nel diagramma delle classi;
			\item \textbf{Flusso principale degli eventi}: L'utente crea un commento all'interno del diagramma delle classi;
			\item \textbf{Postcondizione}: Il commento viene aggiunto al diagramma delle classi;
		\end{itemize}
		\subsection{Caso d'uso UC2.11: Collegare un commento}
		\begin{itemize}
			\item \textbf{Attori}: Utente
			\item \textbf{Descrizione}: L'utente vuole collegare un commento a un altro elemento;
			\item \textbf{Precondizione}: L'utente ha avviato il programma e ha il diagramma delle classi aperto;
			\item \textbf{Flusso principale degli eventi}: L'utente collega un commento a un altro elemento;
			\item \textbf{Postcondizione}: Il commento viene collegato all'elemento nel diagramma delle classi.
		\end{itemize}
		\subsection{Caso d'uso UC2.12: Modificare un commento}
		\begin{itemize}
			\item \textbf{Attori}: Utente
			\item \textbf{Descrizione}: L'utente vuole modificare un commento nel diagramma delle classi;
			\item \textbf{Precondizione}: L'utente ha selezioneto il commento che cuole modificare all'interno del diagramma delle classi;
			\item \textbf{Flusso principale degli eventi}: L'utente modifica un commento nel diagramma delle classi;
			\item \textbf{Postcondizione}: Il commento all'interno del diagramma delle classi viene modificato.
		\end{itemize}
		\subsection{Caso d'uso UC2.13: Eliminare un commento}
		\begin{itemize}
			\item \textbf{Attori}: Utente
			\item \textbf{Descrizione}: L'utente vuole eliminare un commento dal diagramma delle classi;
			\item \textbf{Precondizione}: L'utente ha selezionato il commento che vuole eliminare;
			\item \textbf{Flusso principale degli eventi}: L'utente elimina un commento dal diagramma delle classi;
			\item \textbf{Postcondizione}: Il commento viene eliminato dal diagramma delle classi.
		\end{itemize}
		\subsection{Caso d'uso UC2.14: Aprire l'editor del diagramma dei package}
		\begin{itemize}
			\item \textbf{Attori}: Utente
			\item \textbf{Descrizione}: L'utente vuole aprire l'editor del diagramma dei package;
			\item \textbf{Precondizione}: L'utente si trova nella schermata dell'editor del diagramma delle classi e il sistema è pronto a ricevere un comando dall'utente;
			\item \textbf{Flusso principale degli eventi}: L'utente apre l'editor del diagramma dei package;
			\item \textbf{Postcondizione}: Il sistema visualizza la schermata dell'editor del diagramma dei package;
		\end{itemize}
		\subsection{Caso d'uso UC2.15: Riposizionare un elemento}
		\begin{itemize}
			\item \textbf{Attori}: Utente
			\item \textbf{Descrizione}: L'utente vuole cambiare la posizione di un elemento all'interno del diagramma;
			\item \textbf{Precondizione}: L'utente si trova nella schermata dell'editor del diagramma delle classi e il sistema è pronto a ricevere un comando dall'utente;
			\item \textbf{Flusso principale degli eventi}: L'utente riposiziona l'elemento;
			\item \textbf{Postcondizione}: Il sistema visualizza il diagramma con l'elemento riposizionato correttamente;
		\end{itemize}
		\subsection{Caso d'uso UC3: Modificare una classe mediante la schermata di modifica di classe}
		\begin{itemize}
			\item \textbf{Attori}: Utente
			\item \textbf{Descrizione}: L'utente vuole modificare una classe;
			\item \textbf{Precondizione}: L'utente ha selezionato una classe all'interno del diagramma delle classi che vuole modificare;
			\item \textbf{Flusso principale degli eventi}: \begin{itemize}
				\item L'utente può aggiungere un attributo (UC3.1);
				\item L'utente può modificare un attributo (UC3.2);
				\item L'utente può eliminare un attributo (UC3.3);
				\item L'utente può aggiungere un'operazione (UC3.4);
				\item L'utente può modificare un'operazione (UC3.5);
				\item L'utente può rimuovere un'operazione (UC3.6);
				\item L'utente può commentare una classe (UC3.7);
				\item L'utente può marchiare una classe (UC3.8);
				\item L'utente può passare al diagramma delle classi (UC3.9);
			\end{itemize}
			\item \textbf{Scenari alternativi}: Viene annullata la modifica, la classe rimane nello stato precedente al tentativo di modifica;
			\item \textbf{Postcondizione}: Le modifiche decise dall'utente vengono applicate alla classe.
		\end{itemize}
		\subsection{Caso d'uso UC3.1: Aggiungere un attributo}
		\begin{itemize}
			\item \textbf{Attori}: Utente
			\item \textbf{Descrizione}: L'utente vuole aggiungere un attributo ad una classe;
			\item \textbf{Precondizione}: L'utente ha selezionato una classe all'interno del diagramma delle classi alla quale vuole aggiungere un attributo;
			\item \textbf{Flusso principale degli eventi}: L'utente aggiunge un attributo alla classe;
			\item \textbf{Scenari alternativi}: Viene annullata la modifica, la classe rimane nello stato precedente al tentativo di aggiunta;
			\item \textbf{Postcondizione}: L'attributo viene aggiunto alla classe con i parametri decisi dall'utente.
		\end{itemize}
		\subsection{Caso d'uso UC3.2: Modificare un attributo}
		\begin{itemize}
			\item \textbf{Attori}: Utente
			\item \textbf{Descrizione}: L'utente vuole modificare un attributo di una classe;
			\item \textbf{Precondizione}: L'utente ha selezionato l'attributo che vuole modificare all'interno di una classe;
			\item \textbf{Flusso principale degli eventi}: L'utente modifica l'attributo della classe;
			\item \textbf{Scenari alternativi}: Viene annullata la modifica, il campo dati rimane nello stato precedente al tentativo di modifica;
			\item \textbf{Postcondizione}: Le modifiche decise dall'utente vengono applicate all'attributo all'interno della classe.
		\end{itemize}
		\subsection{Caso d'uso UC3.3: Eliminare un attributo}
		\begin{itemize}
			\item \textbf{Attori}: Utente
			\item \textbf{Descrizione}: L'utente vuole eliminare un attributo all'interno di una classe;
			\item \textbf{Precondizione}: L'utente ha selezionato l'attributo che vuole eliminare;
			\item \textbf{Flusso principale degli eventi}: L'utente elimina un attributo;
			\item \textbf{Scenari alternativi}: Viene annullata la modifica, la classe rimane nello stato precedente al tentativo di eliminazione;
			\item \textbf{Postcondizione}: L'attributo viene eliminato dalla classe dopo eventuali avvisi nel caso ci siano dipendenze da controllare.
		\end{itemize}
		\subsection{Caso d'uso UC3.4: Aggiungere un'operazione}
		\begin{itemize}
			\item \textbf{Attori}: Utente
			\item \textbf{Descrizione}: L'utente vuole aggiungere un'operazione ad una classe;
			\item \textbf{Precondizione}: L'utente ha selezionato una classe all'interno del diagramma delle classi alla quale vuole aggiungere un'operazione;
			\item \textbf{Flusso principale degli eventi}: L'utente aggiunge un'operazione ad una classe;
			\item \textbf{Scenari alternativi}: Viene annullata la modifica, la classe rimane nello stato precedente al tentativo di aggiunta;
			\item \textbf{Postcondizione}: L'operazione viene aggiunta alla classe con i parametri decisi dall'utente.
		\end{itemize}
		\subsection{Caso d'uso UC3.5: Modificare un'operazione}
		\begin{itemize}
			\item \textbf{Attori}: Utente
			\item \textbf{Descrizione}: L'utente vuole modificare un' operazione all'interno di una classe;
			\item \textbf{Precondizione}: L'utente ha selezionato l'operazione che vuole modificare all'interno di una classe;
			\item \textbf{Flusso principale degli eventi}: \begin{itemize}
				\item L'utente può aggiungere un parametro (UC3.5.1);
				\item L'utente può modificare un parametro (UC3.5.2);
				\item L'utente può eliminare un parametro (UC3.5.3);
				\item L'utente può apportare modifiche minori;
			\end{itemize}
			\item \textbf{Scenari alternativi}: Viene annullata la modifica, l'operazione rimane nello stato precedente al tentativo di modifica;
			\item \textbf{Postcondizione}: Le modifiche decise dall'utente vengono applicate all'operazione all'interno della classe.
		\end{itemize}
		\subsection{Caso d'uso UC3.5.1: Aggiungere un parametro}
		\begin{itemize}
			\item \textbf{Attori}: Utente
			\item \textbf{Descrizione}: L'utente vuole aggiungere un parametro alla lista parametri dell'operazione;
			\item \textbf{Precondizione}: Nella schermata di modifica delle classi è stata selezionata l'operazione da modificare e il sistema è pronto a ricevere un comando dall'utente;
			\item \textbf{Flusso principale degli eventi}: L'utente aggiunge un parametro alla lista parametri dell'operazione;
			\item \textbf{Postcondizione}: Nella schermata di modifica delle classi è visualizzato il parametro che è stato aggiunto alla lista parametri dell'operazione.
		\end{itemize}
		\subsection{Caso d'uso UC3.5.2: Modificare un parametro}
		\begin{itemize}
			\item \textbf{Attori}: Utente
			\item \textbf{Descrizione}: L'utente vuole modificare un parametro della lista parametri dell'operazione;
			\item \textbf{Precondizione}: Nella schermata di modifica delle classi è stata selezionata l'operazione da modificare e il sistema è pronto a ricevere un comando dall'utente;
			\item \textbf{Flusso principale degli eventi}: \begin{itemize} \item L'utente può definire la direzione del parametro (UC); \item L'utente può rinominare il parametro (UC); \item L'utente può definire il tipo del parametro (UC); \item L'utente può definire il valore di default del parametro (UC). \end{itemize}
				\item \textbf{Postcondizione}: Nella schermata di modifica delle classi è visualizzato il parametro che è stato modificato nella lista parametri dell'operazione.
			\end{itemize}
			\subsection{Caso d'uso UC3.5.3: Rimuovere un parametro}
			\begin{itemize}
				\item \textbf{Attori}: Utente
				\item \textbf{Descrizione}: L'utente vuole rimuovere un parametro;
				\item \textbf{Precondizione}: Nella schermata di modifica delle classi è stata selezionata l'operazione da modificare e il sistema è pronto a ricevere un comando dall'utente;
				\item \textbf{Flusso principale degli eventi}: L'utente rimuove un parametro;
				\item \textbf{Postcondizione}: Nella schermata di modifica delle classi è visualizzato il parametro che è stato rimosso dalla lista parametri dell'operazione.
			\end{itemize}
			\subsection{Caso d'uso UC3.6: Rimuovere un'operazione}
			\begin{itemize}
				\item \textbf{Attori}: Utente
				\item \textbf{Descrizione}: L'utente vuole rimuovere un'operazione all'interno di una classe;
				\item \textbf{Precondizione}: L'utente ha selezionato l'operazione che vuole rimuovere;
				\item \textbf{Flusso principale degli eventi}: L'utente rimuove un'operazione all'interno di una classe;
				\item \textbf{Scenari alternativi}: Viene annullata la modifica, la classe rimane nello stato precedente al tentativo di rimozione;
				\item \textbf{Postcondizione}: L'operazione viene rimossa dalla classe dopo eventuali avvisi nel caso ci siano dipendenze da controllare.
			\end{itemize}
			\subsection{Caso d'uso UC3.7: Commentare una classe}
			\begin{itemize}
				\item \textbf{Attori}: Utente
				\item \textbf{Descrizione}: L'utente vuole commentare una classe;
				\item \textbf{Precondizione}: L'utente ha selezionato la classe che desidera commentare;
				\item \textbf{Flusso principale degli eventi}: L'utente commenta una classe;
				\item \textbf{Scenari alternativi}: Viene annullata la modifica, la classe rimane nello stato precedente al tentativo di modifica;
				\item \textbf{Postcondizione}: Il commento relativo alla classe viene impostato.
			\end{itemize}
			\subsection{Caso d'uso UC3.8: Marchiare una classe}
			\begin{itemize}
				\item \textbf{Attori}: Utente
				\item \textbf{Descrizione}: L'utente vuole marchiare una classe con un attributo;
				\item \textbf{Precondizione}: L'utente ha selezionato la classe che desidera marchiare;
				\item \textbf{Flusso principale degli eventi}: L'utente marchia una classe con un attributo;
				\item \textbf{Scenari alternativi}: Viene annullata la modifica, la classe rimane nello stato precedente al tentativo di modifica;
				\item \textbf{Postcondizione}: La classe è stata marchiata con un attributo.
			\end{itemize}
			\subsection{Caso d'uso UC3.9: Passare dalla modifica di una classe al diagramma delle classi}
			\begin{itemize}
				\item \textbf{Attori}: Utente
				\item \textbf{Descrizione}: L'utente vuole spostarsi nella schermata del diagramma delle classi;
				\item \textbf{Precondizione}: L'utente si trova nella schermata di modifica delle classi;
				\item \textbf{Flusso principale degli eventi}: L'utente si sposta nella schermata del diagramma delle classi;
				\item \textbf{Postcondizione}: L'utente si trova nella schermata del diagramma delle classi.
			\end{itemize}
			\subsection{Caso d'uso UC4: Modificare un'interfaccia mediante la schermata di modifica di interfaccia}
			\begin{itemize}
				\item \textbf{Attori}: Utente
				\item \textbf{Descrizione}: L'utente vuole modificare un'interfaccia;
				\item \textbf{Precondizione}: L'utente si trova nell'editor delle classi ed il sistema è pronto per ricevere un comando;
				\item \textbf{Flusso principale degli eventi}: \begin{itemize}
					\item L'utente può aggiungere un'operazione (UC4.1);
					\item L'utente può modificare un'operazione (UC4.2);
					\item L'utente può rimuovere un'operazione (UC4.3);
					\item L'utente può commentare l'interfaccia (UC4.4);
					\item L'utente può marchiare l'interfaccia (UC4.5);
					\item L'utente può passare dalla modifica dell'interfaccia al diagramma delle classi (UC4.6);
				\end{itemize}
				\item \textbf{Scenari alternativi}: Viene annullata la modifica, l'interfaccia rimane nello stato precedente al tentativo di modifica;
				\item \textbf{Postcondizione}: Le modifiche decise dall'utente vengono applicate all'interfaccia;
			\end{itemize}
			\subsection{Caso d'uso UC4.1: Aggiungere un'operazione}
			\begin{itemize}
				\item \textbf{Attori}: Utente
				\item \textbf{Descrizione}: L'utente vuole aggiungere un'operazione all'interfaccia;
				\item \textbf{Precondizione}: Nella schermata di modifica di interfaccia il sistema è pronto a ricevere un comando dall'utente;
				\item \textbf{Flusso principale degli eventi}: L'utente aggiunge un'operazione all'interfaccia;
				\item \textbf{Postcondizione}: Nella schermata di modifica di interfaccia è visualizzata la classe a cui è stata aggiunta l'operazione desiderata;
			\end{itemize}
			\subsection{Caso d'uso UC4.2: Modificare un'operazione}
			\begin{itemize}
				\item \textbf{Attori}: Utente
				\item \textbf{Descrizione}: L'utente vuole modificare un'operazione già  inserita in un'interfaccia;
				\item \textbf{Precondizione}: Nella schermata di modifica di interfaccia è stata selezionata l'operazione da modificare e il sistema è pronto a ricevere un comando dall'utente;
				\item \textbf{Flusso principale degli eventi}: \begin{itemize}
					\item L'utente può apportare modifiche minori;
					\item L'utente può aprire il diagramma delle attività  corrispondente (UC4.2.1);
					\item L'utente può aggiungere un parametro (UC4.2.2);
					\item L'utente può modificare un parametro (UC4.2.3);
					\item L'utente può eliminare un parametro (UC4.2.4);
				\end{itemize}
				\item \textbf{Scenari alternativi}: Viene annullata la modifica, il sistema rimane nello stato precedente al tentativo di modifica;
				\item \textbf{Postcondizione}: Nella schermata di modifica di interfaccia è visualizzato il diagramma dove è stata modificata l'operazione.
			\end{itemize}
			\subsection{Caso d'uso UC4.2.1: Aprire il diagramma delle attività  corrispondente}
			\begin{itemize}
				\item \textbf{Attori}: Utente
				\item \textbf{Descrizione}: L'utente vuole aprire il diagramma delle attività  associato all'operazione che vuole modificare;
				\item \textbf{Precondizione}: Nella schermata di modifica di interfaccia è stata selezionata l'operazione da modificare e il sistema è pronto a ricevere un comando dall'utente;
				\item \textbf{Flusso principale degli eventi}: L'utente apre il diagramma delle attività  associato all'operazione che vuole modificare;
				\item \textbf{Postcondizione}: Il sistema visualizza il diagramma delle attività  corrispondente all'operazione che l'utente vuole modificare;
			\end{itemize}
			\subsection{Caso d'uso UC4.2.2: Aggiungere un parametro}
			\begin{itemize}
				\item \textbf{Attori}: Utente
				\item \textbf{Descrizione}: L'utente vuole aggiungere un parametro alla lista parametri dell'operazione;
				\item \textbf{Precondizione}: Nella schermata di modifica di interfaccia è stata selezionata l'operazione da modificare e il sistema è pronto a ricevere un comando dall'utente;
				\item \textbf{Flusso principale degli eventi}: L'utente aggiunge un parametro alla lista parametri dell'operazione;
				\item \textbf{Postcondizione}: Nella schermata di modifica di interfaccia è visualizzato il diagramma dove è stato aggiunto un parametro alla lista parametri dell'operazione.
			\end{itemize}
			\subsection{Caso d'uso UC4.2.3: Modificare un parametro}
			\begin{itemize}
				\item \textbf{Attori}: Utente
				\item \textbf{Descrizione}: L'utente vuole modificare un parametro della lista parametri dell'operazione;
				\item \textbf{Precondizione}: Nella schermata di modifica di interfaccia è stata selezionata l'operazione da modificare e il sistema è pronto a ricevere un comando dall'utente;
				\item \textbf{Flusso principale degli eventi}: L'utente modifica il parametro;
				\item \textbf{Scenari alternativi}: Viene annullata la modifica, il parametro rimane nello stato precedente al tentativo di modifica;
				\item \textbf{Postcondizione}: Nella schermata di modifica di interfaccia è visualizzato il diagramma dove è stato modificato un parametro nella lista parametri dell'operazione.
			\end{itemize}
			\subsection{Caso d'uso UC4.2.4: Eliminare un parametro}
			\begin{itemize}
				\item \textbf{Attori}: Utente
				\item \textbf{Descrizione}: L'utente vuole eliminare un parametro;
				\item \textbf{Precondizione}: Nella schermata di modifica di interfaccia è stata selezionato il parametro da rimuovere e il sistema è pronto a ricevere un comando dall'utente;
				\item \textbf{Flusso principale degli eventi}: L'utente elimina un parametro;
				\item \textbf{Postcondizione}: Nella schermata di modifica di interfaccia è visualizzato il diagramma dove è stato rimosso il parametro dalla lista parametri dell'operazione.	
			\end{itemize}
			\subsection{Caso d'uso UC4.3: Rimuovere un'operazione}
			\begin{itemize}
				\item \textbf{Attori}: Utente
				\item \textbf{Descrizione}: L'utente vuole rimuovere un'operazione;
				\item \textbf{Precondizione}: Nella schermata di modifica di interfaccia è stata selezionata l'operazione da rimuovere e il sistema è pronto a ricevere un comando dall'utente;
				\item \textbf{Flusso principale degli eventi}: L'utente rimuove l'operazione;
				\item \textbf{Postcondizione}: Nella schermata di modifica di interfaccia è visualizzato il diagramma dove è stata rimossa l'operazione.
			\end{itemize}
			\subsection{Caso d'uso UC4.4: Commentare l'interfaccia}
			\begin{itemize}
				\item \textbf{Attori}: Utente
				\item \textbf{Descrizione}: L'utente vuole commentare l'interfaccia;
				\item \textbf{Precondizione}: Nella schermata di modifica di interfaccia è stata selezionata l'interfaccia da commentare e il sistema è pronto a ricevere un comando dall'utente;
				\item \textbf{Flusso principale degli eventi}: L'utente commenta l'interfaccia;
				\item \textbf{Postcondizione}: Nella schermata di modifica di interfaccia è visualizzato il diagramma dove è stata commentata l'interfaccia.
			\end{itemize}
			\subsection{Caso d'uso UC4.5: Marchiare l'interfaccia}
			\begin{itemize}
				\item \textbf{Attori}: Utente
				\item \textbf{Descrizione}: L'utente vuole marchiare l'interfaccia;
				\item \textbf{Precondizione}: Nella schermata di modifica di interfaccia è stata selezionata l'interfaccia da marchiare e il sistema è pronto a ricevere un comando dall'utente;
				\item \textbf{Flusso principale degli eventi}: L'utente marchia l'interfaccia;
				\item \textbf{Postcondizione}: Nella schermata di modifica di interfaccia è visualizzato il diagramma dove è stata marchiata l'interfaccia.
			\end{itemize}
			\subsection{Caso d'uso UC4.6: Passare dalla modifica di interfaccia al diagramma delle classi}
			\begin{itemize}
				\item \textbf{Attori}: Utente
				\item \textbf{Descrizione}: L'utente vuole tornare al diagramma delle classi dalla schermata di modifica di interfaccia;
				\item \textbf{Precondizione}: Il sistema visualizza la schermata di modifica di interfaccia;
				\item \textbf{Flusso principale degli eventi}: L'utente torna al diagramma delle classi dalla schermata di modifica di interfaccia;
				\item \textbf{Postcondizione}: Il sistema visualizza l'editor del diagramma delle classi
			\end{itemize}
			\subsection{Caso d'uso UC5: Editare il diagramma dei package}
			\begin{itemize}
				\item \textbf{Attori}: Utente
				\item \textbf{Descrizione}: L'utente vuole editare il diagramma dei package;
				\item \textbf{Precondizione}: Nella schermata dell'editor del diagramma dei package il sistema è pronto a ricevere un comando dall'utente;
				\item \textbf{Flusso principale degli eventi}: \begin{itemize}
					\item L'utente può creare un package (UC5.1);
					\item L'utente può modificare un package (UC5.2);
					\item L'utente può rimuovere un package (UC5.3);
					\item L'utente può definire una dipendenza (UC5.4);
					\item L'utente può rimuovere una dipendenza (UC5.5);
					\item L'utente può tornare all'editor del diagramma delle classi (UC5.6);
					\item L'utente può riposizionare un elemento (UC5.7);
				\end{itemize}
				\item \textbf{Scenari alternativi}: Viene annullata la modifica, il diagramma rimane nello stato precedente al tentativo di modifica;
				\item \textbf{Postcondizione}: L'utente ha editato diagramma dei package come voluto e il sistema è pronto a ricevere un nuovo comando.
			\end{itemize}
			\subsection{Caso d'uso UC5.1: Creare un package}
			\begin{itemize}
				\item \textbf{Attori}: Utente
				\item \textbf{Descrizione}: L'utente vuole creare un package;
				\item \textbf{Precondizione}: Nella schermata dell'editor del diagramma dei package il sistema è pronto a ricevere un comando dall'utente;
				\item \textbf{Flusso principale degli eventi}: L'utente crea un package;
				\item \textbf{Postcondizione}: Nella schermata dell'editor del diagramma dei package è visualizzato il diagramma a cui è stato aggiunto un nuovo package.
			\end{itemize}
			\subsection{Caso d'uso UC5.2: Modificare un package}
			\begin{itemize}
				\item \textbf{Attori}: Utente
				\item \textbf{Descrizione}: L'utente vuole modificare un package;
				\item \textbf{Precondizione}: Nell'editor del diagramma dei package è stato selezionato un package che l'utente desidera modificare;
				\item \textbf{Flusso principale degli eventi}: \begin{itemize}
					\item L'utente modifica il package;
					\item L'utente può innestare un elemento nel package (UC5.2.1);
					\item L'utente può rimuovere un elemento dal package (UC5.2.2);
				\end{itemize}
				\item \textbf{Scenari alternativi}: Viene annullata la modifica, il sistema	rimane nello stato precedente al tentativo di modifica;
				\item \textbf{Postcondizione}: Nell'editor del diagramma dei package è visualizzato il diagramma dove sono state apportate le modifiche al package e il sistema è pronto a ricevere un nuovo comando.
			\end{itemize}
			\subsection{Caso d'uso UC5.2.1: Innestare un elemento nel package}
			\begin{itemize}
				\item \textbf{Attori}: Utente
				\item \textbf{Descrizione}: L'utente vuole innestare un elemento all'interno di un package;
				\item \textbf{Precondizione}: Nell'editor del diagramma dei package il sistema è pronto ad effettuare l'innesto;
				\item \textbf{Flusso principale degli eventi}: L'utente innesta un elemento all'interno del package;
				\item \textbf{Postcondizione}: Nell'editor del diagramma dei package è visualizzato il diagramma dove è stato effettuato l'innesto e il sistema è pronto a ricevere un nuovo comando.
			\end{itemize}
			\subsection{Caso d'uso UC5.2.2: Rimuovere un elemento dal package}
			\begin{itemize}
				\item \textbf{Attori}: Utente
				\item \textbf{Descrizione}: L'utente vuole rimuovere un elemento da un package;
				\item \textbf{Precondizione}: Nell'editor del diagramma dei package è selezionato l'elemento che l'utente desidera rimuovere;
				\item \textbf{Flusso principale degli eventi}: L'utente rimuove un elemento da un package;
				\item \textbf{Postcondizione}: Nell'editor del diagramma dei package è visualizzato il diagramma dove è stato rimosso l'elemento e il sistema è pronto a ricevere un nuovo comando.
			\end{itemize}
			\subsection{Caso d'uso UC5.3: Eliminare un package}
			\begin{itemize}
				\item \textbf{Attori}: Utente
				\item \textbf{Descrizione}: L'utente vuole eliminare un package;
				\item \textbf{Precondizione}: Nell'editor del diagramma dei package è selezionato il package che l'utente desidera rimuovere;
				\item \textbf{Flusso principale degli eventi}: L'utente elimina un package;
				\item \textbf{Postcondizione}: Nell'editor del diagramma dei package è visualizzato il diagramma dove è stato eliminato il package e il sistema è pronto a ricevere un nuovo comando.
			\end{itemize}
			\subsection{Caso d'uso UC5.4: Definire una dipendenza tra package}
			\begin{itemize}
				\item \textbf{Attori}: Utente
				\item \textbf{Descrizione}: L'utente vuole definire una dipendenza tra due package;
				\item \textbf{Precondizione}: Nell'editor del diagramma dei package il sistema è pronto a creare la dipendenza che l'utente desidera definire;
				\item \textbf{Flusso principale degli eventi}: L'utente definisce una dipendenza tra due package;
				\item \textbf{Postcondizione}: Nell'editor del diagramma dei package del sistema è visualizzato il diagramma dove è stata definita la dipendenza e il sistema è pronto a ricevere un nuovo comando.
			\end{itemize}
			\subsection{Caso d'uso UC5.5: Rimuovere una dipendenza tra package}
			\begin{itemize}
				\item \textbf{Attori}: Utente
				\item \textbf{Descrizione}: L'utente vuole rimuovere una dipendenza tra due package;
				\item \textbf{Precondizione}: Nell'editor del diagramma dei package il sistema è pronto a eliminare la dipendenza che l'utente desidera rimuovere;
				\item \textbf{Flusso principale degli eventi}: L'utente rimuove una dipendenza tra due package;
				\item \textbf{Postcondizione}: Nell'editor del diagramma dei package è visualizzato il diagramma dove è stata rimossa la dipendenza e il sistema è pronto a ricevere un nuovo comando.
			\end{itemize}
			\subsection{Caso d'uso UC5.6: Passare dal diagramma dei package al diagramma delle classi}
			\begin{itemize}
				\item \textbf{Attori}: Utente
				\item \textbf{Descrizione}: L'utente vuole tornare alla schermata dell'editor del diagramma delle classi di un particolare package;
				\item \textbf{Precondizione}: Il sistema visualizza l'editor del diagramma dei package;
				\item \textbf{Flusso principale degli eventi}: L'utente torna alla schermata dell'editor del diagramma delle classi di un particolare package;
				\item \textbf{Postcondizione}: Il sistema visualizza nell'editor del diagramma delle classi il package desiderato.
			\end{itemize}
			\subsection{Caso d'uso UC5.7: Riposizionare un elemento}
			\begin{itemize}
				\item \textbf{Attori}: Utente
				\item \textbf{Descrizione}: L'utente vuole cambiare la posizione di un elemento all'interno del diagramma;
				\item \textbf{Precondizione}: L'utente si trova nella schermata dell'editor del diagramma dei package e il sistema è pronto a ricevere un comando dall'utente;
				\item \textbf{Flusso principale degli eventi}: L'utente riposiziona l'elemento;
				\item \textbf{Postcondizione}: Il sistema visualizza il diagramma con l'elemento riposizionato correttamente;
			\end{itemize}
			\subsection{Caso d'uso UC6: Editare il diagramma delle attività }
			\begin{itemize}
				\item \textbf{Attori}: Utente
				\item \textbf{Descrizione}: L'utente vuole editare il diagramma delle attività ;
				\item \textbf{Precondizione}: L'utente si trova nella schermata dell'editor del diagramma delle attività  e il sistema è pronto a ricevere un comando dall'utente;
				\item \textbf{Flusso principale degli eventi}: \begin{itemize}
					\item L'utente può creare un'attività  (UC6.1);
					\item L'utente può modificare un'attità  (UC6.2);
					\item L'utente può eliminare un'attività  (UC6.3);
					\item L'utente può aggiungere un elemento di decisione (UC6.4);
					\item L'utente può modificare un elemento di decisione (UC6.5);
					\item L'utente può eliminare un elemento di decisione (UC6.6);
					\item L'utente può aggiungere una regione di espansione (UC6.7);
					\item L'utente può modificare una regione di espansione (UC6.8);
					\item L'utente può eliminare una regione di espansione (UC6.9);
					\item L'utente può aggiungere una trasformazione tra pin (UC6.10);
					\item L'utente può modificare trasformazione tra pin (UC6.11);
					\item L'utente può eliminare trasformazione tra pin (UC6.12);
					\item L'utente può aggiungere evento temporale (UC6.13);
					\item L'utente può modificare evento temporale (UC6.14);
					\item L'utente può eliminare evento temporale (UC6.15);
					\item L'utente può tornare al diagramma delle classi (UC6.16);
					\item L'utente può riposizionare un elemento (UC6.17).
				\end{itemize}
					\item \textbf{Scenari alternativi}: Viene annullata la modifica, l'interfaccia rimane nello stato precedente al tentativo di modifica;
					\item \textbf{Postcondizione}: Nella schermata dell'editor del diagramma delle attività  è mostrato il diagramma come è stato editato dall'utente e il sistema è pronto a ricevere un nuovo comando.
				\end{itemize}
				\subsection{Caso d'uso UC6.1: Creare un'attività }
				\begin{itemize}
					\item \textbf{Attori}: Utente
					\item \textbf{Descrizione}: L'utente vuole creare un'attività ;
					\item \textbf{Precondizione}: L'utente si trova nella schermata dell'editor del diagramma delle attività  e il sistema è pronto a ricevere un comando dall'utente;
					\item \textbf{Flusso principale degli eventi}: L'utente crea una nuova attività  che viene aggiunta nel diagramma delle attività .
					\item \textbf{Postcondizione}: Nella schermata dell'editor del diagramma delle attività  è visualizzato il diagramma a cui è stata aggiunta la nuova attività .
				\end{itemize}
				\subsection{Caso d'uso UC6.2: Modificare un'attività }
				\begin{itemize}
					\item \textbf{Attori}: Utente
					\item \textbf{Descrizione}: L'utente vuole modificare un'attività ;
					\item \textbf{Precondizione}: Nella schermata dell'editor del diagramma delle attività  è stata selezionata l'attività  da modificare e il sistema è pronto a ricevere un comando dall'utente;
					\item \textbf{Flusso principale degli eventi}: \begin{itemize}
						\item L'utente può apportare modifiche minori;
						\item L'utente può aggiungere un pin all'attività  (UC6.2.1);
						\item L'utente può modificare un pin dell'attività  (UC6.2.2);
						\item L'utente può rimuovere un pin dell'attività  (UC6.2.3);
						\item L'utente può aprire l'editor del bubble flowchart (UC6.2.4);
					\end{itemize}
					\item \textbf{Scenari alternativi}: Viene annullata la modifica, il sistema rimane nello stato precedente al tentativo di modifica;
					\item \textbf{Postcondizione}: Nella schermata dell'editor del diagramma delle attività  è visualizzato il diagramma in cui è stata modificata l'attività .
				\end{itemize}
				\subsection{Caso d'uso UC6.2.1: Aprire l'editor del bubble flowchart}
				\begin{itemize}
					\item \textbf{Attori}: Utente
					\item \textbf{Descrizione}: L'utente vuole aprire l'editor del bubble flowchart; 
					\item \textbf{Precondizione}: Nella schermata dell'editor del diagramma delle attività  è stata selezionata l'attività  di cui si vuole editare il bubble flowchart e il sistema è pronto a ricevere un comando dall'utente;
					\item \textbf{Flusso principale degli eventi}: L'utente apre l'editor del bubble flowchart;
					\item \textbf{Postcondizione}: Il sistema visualizza la schermata dell'editor del bubble flowchart con aperto il diagramma della attività  corrispondente. Il sistema è pronto a ricevere un nuovo comando dall'utente.
				\end{itemize}
				\subsection{Caso d'uso UC6.2.2: Aggiungere un pin all'attività }
				\begin{itemize}
					\item \textbf{Attori}: Utente
					\item \textbf{Descrizione}: L'utente vuole aggiungere un pin all'attività ;
					\item \textbf{Precondizione}: Nella schermata dell'editor del diagramma delle attività  è stata selezionata l'attività  a cui aggiungere un pin e il sistema è pronto a ricevere un comando dall'utente;
					\item \textbf{Flusso principale degli eventi}: L'utente aggiunge un pin all'attività ;
					\item \textbf{Postcondizione}: Nella schermata dell'editor del diagramma delle attività  è visualizzato il diagramma in cui è stato aggiunto il pin all'attività .
				\end{itemize}
				\subsection{Caso d'uso UC6.2.3: Modificare un pin dell'attività }
				\begin{itemize}
					\item \textbf{Attori}: Utente
					\item \textbf{Descrizione}: L'utente vuole modificare un pin dell'attività ;
					\item \textbf{Precondizione}: Nella schermata dell'editor del diagramma delle attività  è stato selezionato il pin da modificare e il sistema è pronto a ricevere un comando dall'utente;
					\item \textbf{Flusso principale degli eventi}: \begin{itemize}
					\item L'utente può definire la direzione del parametro corrispondente;
					\item L'utente può rinominare il parametro corrispondente;
					\item L'utente può definire il tipo del parametro corrispondente;
					\item L'utente può definire il valore di default del parametro corrispondente.
				\end{itemize}
					\item \textbf{Scenari alternativi}: Viene annullata la modifica, il sistema rimane nello stato precedente al tentativo di modifica;
					\item \textbf{Postcondizione}: Nella schermata dell'editor del diagramma delle attività  è visualizzato il diagramma in cui è stato modificato il pin dell'attività .
				\end{itemize}
				\subsection{Caso d'uso UC6.2.4: Rimuovere un pin dell'attività }
				\begin{itemize}
					\item \textbf{Attori}: Utente
					\item \textbf{Descrizione}: L'utente vuole rimuovere un pin dell'attività ;
					\item \textbf{Precondizione}: Nella schermata dell'editor del diagramma delle attività  è stato selezionato il pin da rimuovere e il sistema è pronto a ricevere un comando dall'utente;
					\item \textbf{Flusso principale degli eventi}: L'utente rimuove il pin selezionato dal diagramma delle attività ;
					\item \textbf{Postcondizione}: Nella schermata dell'editor del diagramma delle attività  è visualizzato il diagramma in cui è stato rimosso il pin all'attività .
				\end{itemize}
				\subsection{Caso d'uso UC6.3: Eliminare un'attività }
				\begin{itemize}
					\item \textbf{Attori}: Utente
					\item \textbf{Descrizione}: L'utente vuole eliminare un'attività ;
					\item \textbf{Precondizione}: Nella schermata dell'editor del diagramma delle attività  è stata selezionata l'attività  da eliminare e il sistema è pronto a ricevere un comando dall'utente;
					\item \textbf{Flusso principale degli eventi}: L'utente elimina un'attività ;
					\item \textbf{Postcondizione}: Nella schermata dell'editor del diagramma delle attività  è visualizzato il diagramma da cui è stata rimossa l'attività .
				\end{itemize}
				\subsection{Caso d'uso UC6.4: Aggiungere un elemento di decisione}
				\begin{itemize}
					\item \textbf{Attori}: Utente
					\item \textbf{Descrizione}: L'utente vuole aggiungere un elemento di decisione;
					\item \textbf{Precondizione}: L'utente si trova nella schermata dell'editor del diagramma delle attività  e il sistema è pronto a ricevere un comando dall'utente;
					\item \textbf{Flusso principale degli eventi}: L'utente aggiunge un elemento di decisione, il quale viene visualizzato nel diagramma delle attività ;
					\item \textbf{Postcondizione}: Nella schermata dell'editor del diagramma delle attività  è visualizzato il diagramma a cui è stato aggiunto l'elemento di decisione.
				\end{itemize}
				\subsection{Caso d'uso UC6.5: Modificare un elemento di decisione}
				\begin{itemize}
					\item \textbf{Attori}: Utente
					\item \textbf{Descrizione}: L'utente vuole modificare un elemento di decisione;
					\item \textbf{Precondizione}: Nella schermata dell'editor del diagramma delle attività  è stato selezionato l'elemento di decisione da modificare e il sistema è pronto a ricevere un comando dall'utente;
					\item \textbf{Flusso principale degli eventi}: L'utente modifica l'elemento di decisione selezionato nel diagramma delle attività ;
					\item \textbf{Postcondizione}: Nella schermata dell'editor del diagramma delle attività  è visualizzato il diagramma in cui è stato modificato l'elemento di decisione.
				\end{itemize}
				\subsection{Caso d'uso UC6.6: Eliminare un elemento di decisione}
				\begin{itemize}
					\item \textbf{Attori}: Utente
					\item \textbf{Descrizione}: L'utente vuole eliminare un elemento di decisione;
					\item \textbf{Precondizione}: Nella schermata dell'editor del diagramma delle attività  è stato selezionato l'elemento di decisione da eliminare e il sistema è pronto a ricevere un comando dall'utente;
					\item \textbf{Flusso principale degli eventi}: L'utente elimina un elemento di decisione;
					\item \textbf{Postcondizione}: Nella schermata dell'editor del diagramma delle attività  è visualizzato il diagramma da cui è stato rimosso l'elemento di decisione.
				\end{itemize}
				\subsection{Caso d'uso UC6.7: Aggiungere una regione di espansione}
				\begin{itemize}
					\item \textbf{Attori}: Utente
					\item \textbf{Descrizione}: L'utente vuole aggiungere una regione di espansione;
					\item \textbf{Precondizione}: L'utente si trova nella schermata dell'editor del diagramma delle attività  e il sistema è pronto a ricevere un comando dall'utente;
					\item \textbf{Flusso principale degli eventi}: L'utente aggiunge una regione di espansione;
					\item \textbf{Postcondizione}: Nella schermata dell'editor del diagramma delle attività  è visualizzato il diagramma a cui è stata aggiunta la regione di espansione;
				\end{itemize}
				\subsection{Caso d'uso UC6.8: Modificare una regione di espansione}
				\begin{itemize}
					\item \textbf{Attori}: Editor UML
					\item \textbf{Descrizione}: L'utente vuole modificare una regione di espansione;
					\item \textbf{Precondizione}: Nella schermata dell'editor del diagramma delle attività  è stata selezionata la regione di espansione da modificare e il sistema è pronto a ricevere un comando dall'utente;
					\item \textbf{Flusso principale degli eventi}: L'utente modifica la regione di espansione;
					\item \textbf{Scenari alternativi}: Viene annullata la modifica, il sistema rimane nello stato precedente al tentativo di modifica;
					\item \textbf{Postcondizione}: Nella schermata dell'editor del diagramma delle attività  è visualizzato il diagramma in cui è stata modificata la regione di espansione.
				\end{itemize}
				\subsection{Caso d'uso UC6.9: Eliminare una regione di espansione}
				\begin{itemize}
					\item \textbf{Attori}: Utente
					\item \textbf{Descrizione}: L'utente vuole eliminare una regione di espansione;
					\item \textbf{Precondizione}: Nella schermata dell'editor del diagramma delle attività  è stata selezionata la regione di espansione da eliminare e il sistema è pronto a ricevere un comando dall'utente;
					\item \textbf{Flusso principale degli eventi}: L'utente elimina una regione di espansione;
					\item \textbf{Postcondizione}: Nella schermata dell'editor del diagramma delle attività  è visualizzato il diagramma da cui è stata rimossa la regione di espansione.
				\end{itemize}
				\subsection{Caso d'uso UC6.10: Aggiungere una trasformazione tra pin}
				\begin{itemize}
					\item \textbf{Attori}: Utente
					\item \textbf{Descrizione}: L'utente vuole aggiungere una trasformazione tra pin;
					\item \textbf{Precondizione}: Nella schermata dell'editor del diagramma delle attività  è stato selezionato il pin da cui far partire la trasformazione e il sistema è pronto a ricevere un comando dall'utente;
					\item \textbf{Flusso principale degli eventi}: L'utente aggiunge una trasformazione tra pin;
					\item \textbf{Postcondizione}: Nella schermata dell'editor del diagramma delle attività  è visualizzato il diagramma a cui è stata aggiunta la trasformazione tra pin.
				\end{itemize}
				\subsection{Caso d'uso UC6.11: Modificare una trasformazione tra pin}
				\begin{itemize}
					\item \textbf{Attori}: Utente
					\item \textbf{Descrizione}: L'utente vuole modificare una trasformazione tra pin;
					\item \textbf{Precondizione}: Nella schermata dell'editor del diagramma delle attività  è stata selezionata la trasformazione da modificare e il sistema è pronto a ricevere un comando dall'utente;
					\item \textbf{Flusso principale degli eventi}: L'utente modifica una trasformazione tra pin;
					\item \textbf{Scenari alternativi}: Viene annullata la modifica, il sistema rimane nello stato precedente al tentativo di modifica;
					\item \textbf{Postcondizione}: Nella schermata dell'editor del diagramma delle attività  è visualizzato il diagramma in cui è stata modificata la trasformazione tra pin.
				\end{itemize}
				\subsection{Caso d'uso UC6.12: Eliminare una trasformazione tra pin}
				\begin{itemize}
					\item \textbf{Attori}: Utente
					\item \textbf{Descrizione}: L'utente vuole eliminare una trasformazione tra pin;
					\item \textbf{Precondizione}: Nella schermata dell'editor del diagramma delle attività  è stata selezionata la trasformazione da eliminare e il sistema è pronto a ricevere un comando dall'utente;
					\item \textbf{Flusso principale degli eventi}: L'utente elimina una trasformazione tra pin;
					\item \textbf{Postcondizione}: Nella schermata dell'editor del diagramma delle attività  è visualizzato il diagramma da cui è stata rimossa la trasformazione tra pin.
				\end{itemize}
				\subsection{Caso d'uso UC6.13: Aggiungere un evento temporale}
				\begin{itemize}
					\item \textbf{Attori}: Utente
					\item \textbf{Descrizione}: L'utente vuole aggiungere un evento temporale;
					\item \textbf{Precondizione}: L'utente si trova nella schermata dell'editor del diagramma delle attività  e il sistema è pronto a ricevere un comando dall'utente;
					\item \textbf{Flusso principale degli eventi}: L'utente aggiunge un evento temporale;
					\item \textbf{Postcondizione}: Nella schermata dell'editor del diagramma delle attività  è visualizzato il diagramma a cui è stato aggiunto l'evento temporale.
				\end{itemize}
				\subsection{Caso d'uso UC6.14: Modificare un evento temporale}
				\begin{itemize}
					\item \textbf{Attori}: Utente
					\item \textbf{Descrizione}: L'utente vuole modificare un evento temporale;
					\item \textbf{Precondizione}: Nella schermata dell'editor del diagramma delle attività  è stato selezionato l'evento temporale da modificare e il sistema è pronto a ricevere un comando dall'utente;
					\item \textbf{Flusso principale degli eventi}: \begin{itemize}
						\item L'utente può rinominare l'evento temporale;
						\item L'utente può definire la durata dell'evento temporale;
					\end{itemize}
					\item \textbf{Scenari alternativi}: Viene annullata la modifica, il sistema rimane nello stato precedente al tentativo di modifica;
					\item \textbf{Postcondizione}: Nella schermata dell'editor del diagramma delle attività  è visualizzato il diagramma in cui è stato modificato l'evento temporale.
				\end{itemize}
				\subsection{Caso d'uso UC6.15: Eliminare un evento temporale}
				\begin{itemize}
					\item \textbf{Attori}: Utente
					\item \textbf{Descrizione}: L'utente vuole eliminare un evento temporale;
					\item \textbf{Precondizione}: Nella schermata dell'editor del diagramma delle attività  è stato selezionato l'evento temporale da eliminare e il sistema è pronto a ricevere un comando dall'utente;
					\item \textbf{Flusso principale degli eventi}: L'utente elimina un evento temporale;
					\item \textbf{Postcondizione}: Nella schermata dell'editor del diagramma delle attività  è visualizzato il diagramma da cui è stato rimosso l'evento temporale.
				\end{itemize}
				\subsection{Caso d'uso UC6.16: Passare dal diagramma delle attività  al diagramma delle classi}
				\begin{itemize}
					\item \textbf{Attori}: Utente
					\item \textbf{Descrizione}: L'utente vuole passare dal diagramma delle attività  al diagramma delle classi;
					\item \textbf{Precondizione}: L'utente si trova nella schermata dell'editor del diagramma delle attività  e il sistema è pronto a ricevere un comando dall'utente;
					\item \textbf{Flusso principale degli eventi}: L'utente passa dal diagramma delle attività  al diagramma delle classi;
					\item \textbf{Postcondizione}: Il sistema visualizza la schermazta dell'editor del diagramma delle classi ed è pronto a ricevere un nuovo comando dall'utente.
				\end{itemize}
				\subsection{Caso d'uso UC6.17: Riposizionare un elemento}
				\begin{itemize}
					\item \textbf{Attori}: Utente
					\item \textbf{Descrizione}: L'utente vuole cambiare la posizione di un elemento all'interno del diagramma;
					\item \textbf{Precondizione}: L'utente si trova nella schermata dell'editor del diagramma delle attività  e il sistema è pronto a ricevere un comando dall'utente;
					\item \textbf{Flusso principale degli eventi}: L'utente riposiziona l'elemento all'interno del diagramma;
					\item \textbf{Postcondizione}: Il sistema visualizza il diagramma con l'elemento riposizionato correttamente;
				\end{itemize}
				\subsection{Caso d'uso UC7: Editare il bubble flowchart}
				\begin{itemize}
					\item \textbf{Attori}: Utente
					\item \textbf{Descrizione}: L'utente vuole editare un bubble flowchart;
					\item \textbf{Precondizione}: Nella schermata dell'editor del bubble flowchart il sistema è pronto a ricevere un comando dall'utente;
					\item \textbf{Flusso principale degli eventi}: \begin{itemize}
						\item L'utente può aggiungere una bubble (UC7.1);
						\item L'utente può modificare una bubble (UC7.2);
						\item L'utente può eliminare una bubble (UC7.3);
						\item L'utente può aggiungere un elemento di decisione (UC7.4);
						\item L'utente può modificare un elemento di decisione (UC7.5);
						\item L'utente può eliminare un elemento di decisione (UC7.6);
						\item L'utente può passare all'editor del diagramma delle attività  (UC7.7);
						\item L'utente può riposizionare un elemento (UC7.8);
					\end{itemize}
					\item \textbf{Postcondizione}: L'utente ha editato il bubble flowchart come voluto e il sistema è pronto a ricevere un nuovo comando.
				\end{itemize}
				\subsection{Caso d'uso UC7.1: Aggiungere una bubble}
				\begin{itemize}
					\item \textbf{Attori}: Utente
					\item \textbf{Descrizione}: L'utente vuole aggiungere una bubble di un tipo desiderato al bubble flowchart;
					\item \textbf{Precondizione}: Nella schermata dell'editor del bubble flowchart il sistema è pronto per l'aggiunta di una bubble;
					\item \textbf{Flusso principale degli eventi}: L'utente aggiunge una bubble di un tipo desiderato al bubble flowchart;
					\item \textbf{Postcondizione}: Nella schermata dell'editor del bubble flowchart è visualizzato il diagramma a cui è stata aggiunta una bubble vuota del tipo voluto.
				\end{itemize}
				\subsection{Caso d'uso UC7.2: Modificare una bubble}
				\begin{itemize}
					\item \textbf{Attori}: Utente
					\item \textbf{Descrizione}: L'utente vuole modificare i parametri di una bubble. Questi saranno specifici per ciascuna bubble, i relativi casi d'uso saranno quindi approfonditi nelle successive fasi di progettazione;
					\item \textbf{Precondizione}: Nella schermata dell'editor del bubble flowchart è stata selezionata una bubble;
					\item \textbf{Flusso principale degli eventi}: L'utente modifica i parametri di una bubble;
					\item \textbf{Scenari alternativi}: Viene annullata la modifica, il sistema	rimane nello stato precedente al tentativo di modifica;
					\item \textbf{Postcondizione}: Nella schermata dell'editor del bubble flowchart è visualizzato il diagramma in cui sono stati opportunamente modificati i parametri della bubble.
				\end{itemize}
				\subsection{Caso d'uso UC7.3: Eliminare una bubble}
				\begin{itemize}
					\item \textbf{Attori}: Utente
					\item \textbf{Descrizione}: L'utente vuole eliminare una bubble;
					\item \textbf{Precondizione}: Nella schermata dell'editor del bubble flowchart è stata selezionata una bubble;
					\item \textbf{Flusso principale degli eventi}: L'utente elimina una bubble;
					\item \textbf{Postcondizione}: Nella schermata dell'editor del bubble flowchart è visualizzato il diagramma in cui è stata eliminata la bubble.
				\end{itemize}
				\subsection{Caso d'uso UC7.4: Aggiungere un elemento di decisione}
				\begin{itemize}
					\item \textbf{Attori}: Utente
					\item \textbf{Descrizione}: L'utente vuole aggiungere un elemento di decisione al bubble flowchart;
					\item \textbf{Precondizione}: Nella schermata dell'editor del bubble flowchart il sistema è pronto per l'aggiunta di un elemento di decisione;
					\item \textbf{Flusso principale degli eventi}: L'utente aggiunge un elemento di decisione al bubble flowchart;
					\item \textbf{Postcondizione}: Nella schermata dell'editor del bubble flowchart è visualizzato il diagramma a cui è stato aggiunto un elemento di decisione vuoto.
				\end{itemize}
				\subsection{Caso d'uso UC7.5: Modificare un elemento di decisione}
				\begin{itemize}
					\item \textbf{Attori}: Utente
					\item \textbf{Descrizione}: L'utente vuole modificare i parametri di un elemento di decisione;
					\item \textbf{Precondizione}: Nella schermata dell'editor del bubble flowchart è stato selezionato un elemento di decisione;
					\item \textbf{Flusso principale degli eventi}: L'utente modifica i parametri di un elemento di decisione;
					\item \textbf{Postcondizione}: Nella schermata dell'editor del bubble flowchart è visualizzato il diagramma in cui sono stati opportunamente modificati i parametri dell'elemento di decisione.
				\end{itemize}
				\subsection{Caso d'uso UC7.6: Eliminare un elemento di decisione}
				\begin{itemize}
					\item \textbf{Attori}: Utente
					\item \textbf{Descrizione}: L'utente vuole eliminare un elemento di decisione;
					\item \textbf{Precondizione}: Nella schermata dell'editor del bubble flowchart è stato selezionato un elemento di decisione;
					\item \textbf{Flusso principale degli eventi}: L'utente elimina un elemento di decisione;
					\item \textbf{Postcondizione}: Nella schermata dell'editor del bubble flowchart è visualizzato il diagramma in cui è stato eliminato l'elemento di decisione.
				\end{itemize}
				\subsection{Caso d'uso UC7.7: Passare dal bubble flowchart al diagramma delle attività }
				\begin{itemize}
					\item \textbf{Attori}: Utente
					\item \textbf{Descrizione}: L'utente vuole tornare alla schermata dell'editor del diagramma delle attività ;
					\item \textbf{Precondizione}: Il sistema visualizza l'editor del bubble flowchart;
					\item \textbf{Flusso principale degli eventi}: L'utente torna alla schermata dell'editor del diagramma delle attività ;
					\item \textbf{Postcondizione}: Il sistema visualizza l'editor del diagramma delle attività .
				\end{itemize}
				\subsection{Caso d'uso UC7.8: Riposizionare un elemento}
				\begin{itemize}
					\item \textbf{Attori}: Utente
					\item \textbf{Descrizione}: L'utente vuole cambiare la posizione di un elemento all'interno del diagramma;
					\item \textbf{Precondizione}: L'utente si trova nella schermata dell'editor del bubble flowchart e il sistema è pronto a ricevere un comando dall'utente;
					\item \textbf{Flusso principale degli eventi}: L'utente riposiziona l'elemento;
					\item \textbf{Postcondizione}: Il sistema visualizza il diagramma con l'elemento riposizionato correttamente;
				\end{itemize}
\end{document}