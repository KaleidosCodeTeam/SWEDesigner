\documentclass[a4paper]{report}
\usepackage[english, italian]{babel}
\usepackage[T1]{fontenc}
\usepackage[utf8]{inputenc}
\usepackage{url}
\usepackage{graphicx}
\graphicspath{{../Figures/}}
\usepackage[hidelinks]{hyperref}
\usepackage{booktabs}
\usepackage{tabularx}
\usepackage{pifont}
\usepackage[table]{xcolor}	
\usepackage{float}



\begin{document}
\chapter{Casi d'uso }
\section{UC1: Editare il diagramma delle classi}
\begin{itemize}
	\item \textbf{Attori: }Utente;
	\item \textbf{Scopo e descrizione: }L'utente ha avviato correttamente il programma e ha aperto un progetto.Ora l'utente può editare il diagramma delle classi ;
	
	!!!!!!!!!!! da definire la descrizione dell'interfaccia !!!!!!!!!!!!!!!
	
	\item \textbf{Flusso principale degli eventi: } ;
	\begin{enumerate}
		\item L'utente può creare una nuova classe (UC1.1);
		\item L'utente può modificare una classe (UC1.2);
		\item L'utente può eliminare una classe (UC1.3);
		\item L'utente può definire una nuova relazione (UC1.4);
		\item L'utente può modificare una relazione (UC1.5);
		\item L'utente può eliminare una relazione (UC1.6);
		\item L'utente può creare una nuova interfaccia (UC1.7);
		\item L'utente può modificare una interfaccia (UC1.8);
		\item L'utente può eliminare una interfaccia (UC1.9);
		\item L'utente può cambiare layer di visualizzazione (UC1.10);
	\end{enumerate}
	\item \textbf{Pre condizione: }L'utente ha avviato correttamente il programma e ha aperto un progetto;
	\item \textbf{Post condizione: }Il sistema apporta le modifiche desiderate al diagramma delle classi.
\end{itemize}

\subsection{UC1.2: Creare una nuova classe}
\begin{itemize}
	\item \textbf{Attori:} Utente;
	\item \textbf{Scopo e descrizione:}L'utente può aggiungere una nuova classe vuota al diagramma delle classi;
	\item \textbf{Pre condizione:}Il programma è in esecuzione con un progetto apreto nel diagramma delle classi;
	\item \textbf{Post condizione:}Viene aggiunta una nuova classe al diagramma delle classi.
\end{itemize}

\subsection{UC1.3: Modificare una classe}
\begin{itemize}
	\item \textbf{Attori:} Utente;
	\item \textbf{Scopo e descrizione:}L'utente vuole modificare uno dei vari attributi della classe selezionata ;
	\item \textbf{Pre condizione:}L'utente ha selezionato una classe da modificare all'interno di un progetto;
	\item \textbf{Post condizione:}Le varie modifiche decise dall'utente verranno applicate alla classe nel diagramma delle classi .
\end{itemize}

\subsubsection{UC1.4: Impostare il nome della classe}
\begin{itemize}
	\item \textbf{Attori:} Utente;
	\item \textbf{Scopo e descrizione: }L'utente vuole impostare il nome di una classe  !!!!culoculoculo  appena creata culoculoculo!!!!!  ;
	\item \textbf{Pre condizione: }L'utente ha creato una classe all'interno del diagramma delle classi;
	\item \textbf{Post condizione: }Il nuovo nome deciso dall'utente viene impostato come nome della classe all'interno del diagramma delle classi.
\end{itemize}

\subsubsection{UC1.5: Aggiungere un campo dati alla classe}
\begin{itemize}
	\item \textbf{Attori:} Utente;
	\item \textbf{Scopo e descrizione: }L'utente vuole aggiungere un nuovo campo dati a una classe;
	\item \textbf{Pre condizione: }L'utente ha selezionato una classe all'interno del diagramma delle classi alla quale vuole aggiungere un campo dati;
	\item \textbf{Post condizione: }Il campo dati viene aggiunto alla classe con i parametri decisi dall'utente.
\end{itemize}

\subsubsection{UC1.6: Eliminare un campo dati alla classe}
\begin{itemize}
	\item \textbf{Attori:} Utente;
	\item \textbf{Scopo e descrizione: }L'utente vuole eliminare un campo dati all'interno di una classe;
	\item \textbf{Pre condizione: }L'utente ha selezionato il campo dati che vuole eliminare;
	\item \textbf{Post condizione: }Il campo dati viene rimosso dalla classe dopo eventuali avvisi nel caso ci siano dipendenze da controllare .
\end{itemize}

\subsubsection{UC1.7: Modificare un campo dati alla classe}
\begin{itemize}
	\item \textbf{Attori:} Utente;
	\item \textbf{Scopo e descrizione: }L'utente vuole modificare un campo dati all'interno di una classe del diagramma delle classi;
	\item \textbf{Pre condizione: }L'utente ha selezionato il campo dati che vuole modificare all'interno di una classe;
	\item \textbf{Post condizione: }Le modifiche decise dall'utente vengono applicate al campo dati all'interno della classe.
\end{itemize}

\subsubsection{UC1.8: Aggiungere un' operazione alla classe}
\begin{itemize}
	\item \textbf{Attori:} Utente;
	\item \textbf{Scopo e descrizione: }L'utente vuole aggiungere un' operazione a una classe;
	\item \textbf{Pre condizione: }L'utente ha selezionato una classe all'interno del diagramma delle classi alla quale vuole aggiungere un'operazione;
	\item \textbf{Post condizione: }L'operazione viene aggiunta alla classe con i parametri decisi dall'utente..
\end{itemize}

\subsubsection{UC1.9: Eliminare un' operazione  alla classe}
\begin{itemize}
	\item \textbf{Attori:} Utente;
	\item \textbf{Scopo e descrizione: }L'utente vuole eliminare un'operazione all'interno di una classe;
	\item \textbf{Pre condizione: }L'utente ha selezionato L'operazione che vuole eliminare;
	\item \textbf{Post condizione: }L'operazione viene rimossa dalla classe dopo eventuali avvisi nel caso ci siano dipendenze da controllare .
\end{itemize}

\subsubsection{UC1.10: Modificare un' operazione alla classe}
\begin{itemize}
	\item \textbf{Attori:} Utente;
	\item \textbf{Scopo e descrizione: }L'utente vuole modificare un' operazione all'interno di una classe del diagramma delle classi;
	\item \textbf{Pre condizione: }L'utente ha selezionato l'operazione che vuole modificare all'interno di una classe;
	\item \textbf{Post condizione: }Le modifiche decise dall'utente vengono applicate all'operazione all'interno della classe.
\end{itemize}

\end{document}

