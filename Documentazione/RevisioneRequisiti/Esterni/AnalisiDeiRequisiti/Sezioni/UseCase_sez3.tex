
\paragraph{Caso d'uso UC: Innestare una classe interna}
\begin{itemize}
	\item\textbf{Attori}: Utente.
	\item\textbf{Descrizione}: L'utente vuole innestare una classe all'interno di un'altra classe.
	\item\textbf{Precondizione}: Sono presenti due classi distinte e non innestate l'una nell'altra.
	\item\textbf{Postcondizione}: Nell'editor del diagramma delle classi le due classi sono visualizzate una innestata dentro l'altra.	
\end{itemize}

\paragraph{Caso d'uso UC: Impostare l'importanza della classe}
\begin{itemize}
	\item\textbf{Attori}: Utente
	\item\textbf{Descrizione}: L'utente vuole impostare l'importanza di una classe.
	\item\textbf{Precondizione}: Esiste una classe e l'utente vuole modificare l'importanza.
	\item\textbf{Postcondizione}:L'importanza è settata sul valore che l'utente desidera.
\end{itemize}

\paragraph{Caso d'uso UC: Eliminare una classe}
\begin{itemize}
	\item\textbf{Attori}: Utente.
	\item\textbf{Descrizione}: L'utente vuole eliminare una classe.
	\item\textbf{Precondizione}: Esiste una classe che l'utente desidera eliminare.
	\item\textbf{Postcondizione}: La classe non è più visualizzata nell'editor del diagramma delle classi.
\end{itemize}

\paragraph{Caso d'uso UC: Definire una relazione}
\begin{itemize}
	\item\textbf{Attori}: Utente.
	\item\textbf{Descrizione}: L'utente vuole definire una relazione.
	\item\textbf{Precondizione}: Sono presenti due classi e l'utente desiderano che presentino una relazione l'una dall'altra.
	\item\textbf{Scenario principale}: 
		\begin{itemize}
			\item L'utente vuole definire la dipendenza tra due classi (UC).
			\item L'utente vuole definire l'associazione tra due classi (UC).
			\item L'utente vuole definire l'ereditarietà tra due classi (UC).
			\item L'utente vuole definire l'aggregazione tra due classi (UC).
			\item L'utente vuole definire la composizione tra due classi (UC).
		\end{itemize}
	\item\textbf{Postcondizione}: Le due classi presentano una relazione.
	
\end{itemize}

\paragraph{Caso d'uso UC: Modificare una relazione}
\begin{itemize}
	\item\textbf{Attori}: Utente.
	\item\textbf{Descrizione}: L'utente vuole modificare una relazione tra due classi.
	\item\textbf{Precondizione}: È presente una relazione che l'utente vuole modificare.
	\item\textbf{Scenario principale}: 
		\begin{itemize}
			\item L'utente vuole modificare la dipendenza tra due classi (UC).
			\item L'utente vuole modificare l'associazione tra due classi (UC).
			\item L'utente vuole modificare l'ereditarietà tra due classi (UC).
			\item L'utente vuole modificare l'aggregazione tra due classi (UC).
			\item L'utente vuole modificare la composizione tra due classi (UC).
		\end{itemize}
	\item\textbf{Postcondizione}: La relazione viene modificata.
\end{itemize}

\paragraph{Caso d'uso UC: Eliminare una relazione}
\begin{itemize}
	\item\textbf{Attori}: Utente.
	\item\textbf{Descrizione}: L'utente vuole eliminare una relazione.
	\item\textbf{Precondizione}: Esiste una relazione che l'utente desidera eliminare.
	\item\textbf{Scenario principale}: 
		\begin{itemize}
			\item L'utente vuole eliminare la dipendenza tra due classi (UC).
			\item L'utente vuole eliminare l'associazione tra due classi (UC).
			\item L'utente vuole eliminare l'ereditarietà tra due classi (UC).
			\item L'utente vuole eliminare l'aggregazione tra due classi (UC).
			\item L'utente vuole eliminare la composizione tra due classi (UC).
		\end{itemize}
	\item\textbf{Postcondizione}: La relazione viene eliminata.
\end{itemize}

\paragraph{Caso d'uso UC: Definire dipendenza tra classi}
\begin{itemize}
	\item\textbf{Attori}: Utente.
	\item\textbf{Descrizione}: L'utente vuole definire la dipendenza tra due classi.
	\item\textbf{Precondizione}: Sono presenti due classi e l'utente vuole evidenziarne la dipendenza.
	\item\textbf{Postcondizione}: La dipendenza tra le due classi è stata definita.
\end{itemize}

\paragraph{Caso d'uso 5: Modificare dipendenza tra classi}
\begin{itemize}
	\item\textbf{Attori}: Utente.
	\item\textbf{Descrizione}: L'utente vuole definire la dipendenza tra due classi.
	\item\textbf{Precondizione}: Sono presenti due classi che presentano una dipendenza l'una dall'altra.
	\item\textbf{Postcondizione}: La dipendenza tra le due classi è stata modificata nel modo che l'utente desidera.
\end{itemize}

\paragraph{Caso d'uso UC: Eliminare dipendenza tra classi}
\begin{itemize}
	\item\textbf{Attori}: Utente.
	\item\textbf{Descrizione}: L'utente vuole eliminare una dipendenza tra classi.
	\item\textbf{Precondizione}: Esiste una dipendenza tra classi che l'utente desidera eliminare.
	\item\textbf{Postcondizione}: La dipendenza tra classi viene eliminata.
\end{itemize}

\paragraph{Caso d'uso UC: Definire associazione tra classi}
\begin{itemize}
	\item\textbf{Attori}: Utente.
	\item\textbf{Descrizione}: L'utente vuole definire un'associazione tra due classi.
	\item\textbf{Precondizione}: Sono presenti due classi e l'utente vuole evidenziarne l'associazione.
	\item\textbf{Postcondizione}: L'associazione tra le due classi è stata definita.
\end{itemize}

\paragraph{Caso d'uso UC: Modificare associazione tra classi}
\begin{itemize}
	\item\textbf{Attori}: Utente.
	\item\textbf{Descrizione}: L'utente vuole eliminare un'associazione tra classi.
	\item\textbf{Precondizione}: Sono presenti due classi e l'utente vuole modificarne l'associazione.
	\item\textbf{Postcondizione}: L'associazione tra le due classi è stata modificata.
\end{itemize}

\paragraph{Caso d'uso UC: Eliminare associazione tra classi}
\begin{itemize}
	\item\textbf{Attori}: Utente.
	\item\textbf{Descrizione}: L'utente vuole eliminare un'associazione tra classi.
	\item\textbf{Precondizione}: Esiste un'associazione tra classi che l'utente desidera eliminare.
	\item\textbf{Postcondizione}: L'associazione tra classi viene eliminata.
\end{itemize}

\paragraph{Caso d'uso UC: Definire ereditarietà tra classi}
\begin{itemize}
	\item\textbf{Attori}: Utente.
	\item\textbf{Descrizione}: L'utente vuole definire un vincolo di ereditarietà tra due classi.
	\item\textbf{Precondizione}: Sono presenti due classi e l'utente vuole evidenziarne il vincolo di ereditarietà.
	\item\textbf{Postcondizione}: L'ereditarietà tra le due classi è stata definita.
\end{itemize}

\paragraph{Caso d'uso UC: Modificare ereditarietà tra classi}
\begin{itemize}
	\item\textbf{Attori}: Utente.
	\item\textbf{Descrizione}: L'utente vuole modificare un vincolo di ereditarietà tra due classi.
	\item\textbf{Precondizione}: Sono presenti due classi e l'utente vuole modificarne il vincolo di ereditarietà.
	\item\textbf{Postcondizione}: L'ereditarietà tra le due classi è stata modificata.
\end{itemize}

\paragraph{Caso d'uso UC: Eliminare ereditarietà tra classi}
\begin{itemize}
	\item\textbf{Attori}: Utente.
	\item\textbf{Descrizione}: L'utente vuole eliminare un'ereditarietà tra classi
	\item\textbf{Precondizione}: Esiste un'ereditarietà tra classi che l'utente desidera eliminare.
	\item\textbf{Postcondizione}: L'ereditarietà tra classi viene eliminata.
\end{itemize}

\paragraph{Caso d'uso UC: Definire aggregazione tra classi}
\begin{itemize}
	\item\textbf{Attori}: Utente.
	\item\textbf{Descrizione}: L'utente vuole definire una relazione di aggregazione tra due classi.
	\item\textbf{Precondizione}: Sono presenti due classi e l'utente vuole evidenziarne l'aggregazione.
	\item\textbf{Postcondizione}: La relazione di aggregazione tra le due classi è stata definita.
\end{itemize}

\paragraph{Caso d'uso UC: Modificare aggregazione tra classi}
\begin{itemize}
	\item\textbf{Attori}: Utente.
	\item\textbf{Descrizione}: L'utente vuole modificare una relazione di aggregazione tra due classi.
	\item\textbf{Precondizione}: Sono presenti due classi e l'utente vuole modificarne l'aggregazione.
	\item\textbf{Postcondizione}: La relazione di aggregazione tra le due classi è stata modificata.
\end{itemize}

\paragraph{Caso d'uso UC: Eliminare aggregazione tra classi}
\begin{itemize}
	\item\textbf{Attori}: Utente.
	\item\textbf{Descrizione}: L'utente vuole eliminare una relazione di aggregazione tra classi.
	\item\textbf{Precondizione}: Esiste una relazine di aggregazione tra classi che l'utente desidera eliminare.
	\item\textbf{Postcondizione}: La relazione di aggregazione tra classi viene eliminata.
\end{itemize}

\paragraph{Caso d'uso UC: Definire composizione tra classi}
\begin{itemize}
	\item\textbf{Attori}: Utente.
	\item\textbf{Descrizione}: L'utente vuole definire una composizione tra due classi.
	\item\textbf{Precondizione}: Sono presenti due classi e l'utente vuole evidenziarne la composizione.
	\item\textbf{Postcondizione}: La relazione di composizione tra le due classi è stata definita.
\end{itemize}

\paragraph{Caso d'uso UC:  Modificare composizione tra classi}
\begin{itemize}
	\item\textbf{Attori}: Utente.
	\item\textbf{Descrizione}: L'utente vuole modificare una relazione di composizione tra classi.
	\item\textbf{Precondizione}: Esiste una relazine di composizione tra classi che l'utente desidera modificare.
	\item\textbf{Postcondizione}: La relazione di composizione tra classi viene modificata.
\end{itemize}

\paragraph{Caso d'uso UC: Eliminare composizione tra classi}
\begin{itemize}
	\item\textbf{Attori}: Utente.
	\item\textbf{Descrizione}: L'utente vuole eliminare una relazione di composizione tra classi.
	\item\textbf{Precondizione}: Esiste una relazine di composizione tra classi che l'utente desidera eliminare.
	\item\textbf{Postcondizione}: La relazione di composizione tra classi viene eliminata.
\end{itemize}

\paragraph{Caso d'uso UC: Creare un'interfaccia}
\begin{itemize}
	\item\textbf{Attori}: Utente.
	\item\textbf{Descrizione}: L'utente vuole creare un'interfaccia.
	\item\textbf{Precondizione}: Il sistema è pronto alla creazione di un'interfaccia, l'utente desidera creare un'interfaccia.
	\item\textbf{Postcondizione}: Nell'editor del diagramma delle classi l'interfaccia è correttamente visualizzato il diagramma nel quale è stata creata l'interfaccia.
\end{itemize}
