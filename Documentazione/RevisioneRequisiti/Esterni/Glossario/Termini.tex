\newglossaryentry{UML} {
	name=UML,
	description={Acronimo di Unified Modeling Language (linguaggio di modellizzazione
	unificato), è un linguaggio di modellizzazione e specifica basato sul paradigma
	orientato agli oggetti}
}
\newglossaryentry{Java} {
	name=Java,
	description={Linguaggio di programmazione ad alto livello orientato agli oggetti}
}
\newglossaryentry{Javascript} {
	name=Javascript,
	description={Linguaggio di scripting orientato agli oggetti e agli eventi,
	comunemente utilizzato nella programmazione web lato client}
}
\newglossaryentry{Android} {
	name=Android,
	description={Sistema operativo per dispositivi mobili sviluppato da Google Inc. e basato su kernel Linux}
}
\newglossaryentry{HTML} {
	name=HTML,
	description={Acronimo di HyperText Markup Language (linguaggio a marcatori per
	ipertesti), è un linguaggio di markup usato principalmente per creare la
	struttura di documenti ipertestuali}
}
\newglossaryentry{CSS} {
	name=CSS,
	description={Acronimo di Cascading Style Sheets (fogli di stile a cascata), è un linguaggio usato per
	definire la formattazione di documenti HTML}
}
\newglossaryentry{ISO} {
	name=ISO,
	description={Abbreviazione di International Organization fo Standardization (organizzazione internazionale per la
	normalizzazione), è la più importante organizzazione a livello mondiale per la definizione di norme tecniche}
}
\newglossaryentry{UTF-8} {
	name=UTF-8,
	description={Acronimo di Unicode Transformation Format 8 bit, è una codifica di caratteri Unicode in sequenze di
	lunghezza variabile di byte}
}
\newglossaryentry{Mailing list} {
	name=Mailing list,
	description={Lista di distribuzione o diffusione, è un servizio/strumento offribile da una rete di computer verso
	vari utenti e costituito da un sistema organizzato per la partecipazione di più persone ad una discussione asincrona
	o per la distribuzione di informazioni utili agli interessati/iscritti attraverso l'invio di e-mail ad una lista di
	indirizzi di posta elettronica di utenti iscritti}
}
\newglossaryentry{Git} {
	name=Git,
	description={È un software di controllo versione distribuito utilizzabile da interfaccia a riga di comando, creato da Linus Torvalds nel 2005}
}
\newglossaryentry{Repository} {
	name=Repository,
	description={Il repository è un archivio in cui sono racchiusi dati ed informazioni in formato digitale, valorizzati e archiviati sulla base di metadati che ne permettano la rapida individuazione.
	Nel caso specifico di repository Git, si tratta dell'archivio contenente tutte le versioni del codice caricato sul server}
}
\newglossaryentry{Google Drive} {
	name=Google Drive,
	description={Google Drive è un servizio, in ambiente cloud computing, di memorizzazione e sincronizzazione online introdotto da Google il 24 aprile 2012. 
		Il servizio può essere usato via Web, caricando e visualizzando i file tramite il web browser, oppure tramite l'applicazione installata su computer, 
		che sincronizza automaticamente una cartella locale del file system con quella condivisa. Su Google Drive sono presenti anche i documenti creati con Google Documenti}
}
\newglossaryentry{Diagrammi di Gantt} {
	name=Diagrammi di Gantt,
	description={Il diagramma di Gantt è uno strumento di supporto alla gestione dei progetti che permette la rappresentazione grafica di un calendario di attività, 
		utile al fine di pianificare, coordinare e tracciare specifiche attività in un progetto dando una chiara illustrazione dello stato d'avanzamento del progetto rappresentato}
}
\newglossaryentry{Diagrammi di PERT} {
	name=Diagrammi di PERT,
	description={Il diagramma reticolare di PERT (Program Evaluation and Review Technique) descrive la sequenza cronologica delle azioni pianificate per il completamento di un progetto nel suo complesso.
		Esso rappresenta graficamente il piano d'azione.}
}
\newglossaryentry{Work Breakdown Structure} {
	name=Work Breakdown Structure,
	description={Con l'espressione inglese work breakdown structure (WBS), detta anche struttura di scomposizione del lavoro (traduzione letterale) o struttura analitica di progetto, 
		si intende l'elenco di tutte le attività di un progetto organizzate attraverso un albero gerarchico}
}
\newglossaryentry{Schedule Variance} {
	name=Schedule Variance,
	description={È una metrica di progetto standard. Indica se si è in linea, in anticipo o in ritardo rispetto alla schedulazione delle attività di progetto pianificate}
}
\newglossaryentry{Budget Variance} {
	name=Budget Variance,
	description={È una metrica di progetto standard. Indica se alla data corrente si è speso di più o di meno rispetto a quanto previsto alla data corrente}
}
\newglossaryentry{XML} {
	name=XML,
	description={È un metalinguaggio per la definizione di linguaggi di markup, ovvero un linguaggio marcatore basato su un meccanismo sintattico che consente di definire e controllare il significato degli elementi contenuti in un documento o in un testo.}
}
\newglossaryentry{} {
	name= ,
	description={ }
}