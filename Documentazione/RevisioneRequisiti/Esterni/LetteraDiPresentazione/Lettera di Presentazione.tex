% Document-Author: KaleidosCode
% Document-Date: 01/04/2017 aggiornare la data
% Document-Description: Lettera di presentazione

\documentclass[a4paper,12pt]{article}
\usepackage{../../../Templates/kaleidos}
\usepackage{titletoc}
\newcommand\VRule[1][\arrayrulewidth]{\vrule width #1}

\author{KaleidosCode}
\date{01/04/2017}	% aggiornare la data


\pagestyle{empty}
\intestazioni{yht}

\begin{document}
	\begin{titlepage}
		%\centering
		%\logo
		\includegraphics[scale=0.2]{../../../Immagini/KaleidosCodeLogo.png}
		\hrule
		\vspace{1.2cm}
		\flushright 1 aprile 2017\\
		\vspace{0.4cm}
		Alla gentile attenzione del Committente:\\
		\vardanega\\
		\cardin\\
		Università degli Studi di Padova\\
		Dipartimento di Matematica\\
		Via Trieste, 63\\
		35121, Padova (PD)\\
		\vspace{1.2cm}
		\flushleft Il \responsabilediprogetto, Sovilla Matteo - \kaleidoscode\\
		\vspace{0.4cm}
		Oggetto: \textbf{presentazione proposta per capitolato d'appalto C6}\\
		\vspace{1cm}
		Egregio \vardanega,\\
		\vspace{0.4cm}
		\par Con la presente, il gruppo \kaleidoscode\ intende comunicarLe ufficialmente l'impegno alla realizzazione
		del prodotto da Lei commissionato:
		\begin{center}
			\textbf{SWEDesigner: editor di diagrammi UML con generazione di codice} 
		\end{center}
		proposto da \proponente.
		\par L'offerta è corredata dai seguenti documenti, allegati alla presente lettera:
		\begin{itemize}
			\item \analisideirequisitiv\ (Esterni/AnalisiDeiRequisiti\char`_v1.0.0.pdf);
			\item \glossariov\ (Esterni/Glossario\char`_v1.0.0.pdf);
			\item \normediprogettov\ (Interni/NormeDiProgetto\char`_v1.0.0.pdf);
			\item \pianodiprogettov\ (Esterni/PianoDiProgetto\char`_v1.0.0.pdf);
			\item \pianodiqualificav\ (Esterni/PianoDiQualifica\char`_v1.0.0.pdf);
			\item \studiodifattibilitav\ (Interni/StudioDiFattibilita\char`_v1.0.0.pdf);
			\item Verbale\char`_17-02-2017 (Esterni/Verbale\char`_17-02-2017.pdf).
			\item Verbale\char`_23-02-2017 (Esterni/Verbale\char`_23-02-2017.pdf).
		\end{itemize}	
		Il gruppo \kaleidoscode\ ha stimato di consegnare il prodotto richiesto entro la fine del secondo semestre
		dell'anno accademico 2016–2017 con un preventivo di costo pari a \hbox{\euro\ 12 393,00}.\\
		I dettagli di analisi del prodotto, di pianificazione e di qualità sono trattati in maniera approfondita
		nei documenti allegati.
		\vspace{0.5cm}
		\par Di seguito viene presentato l'organigramma del team:
		\vspace{0.4cm}
		\begin{table}[H]
			\center
			\begin{tabular}{!{\VRule[1.4pt]}l!{\VRule}c!{\VRule}l!{\VRule[1.4pt]}}
				\noalign{\hrule height 1.4pt}
				\textbf{Nome} & \textbf{Matricola} & \textbf{Posta elettronica} \\ \hline
				Bonato Enrico & 1096071 & enrico.bonato.5@studenti.unipd.it \\ \hline
				Bonolo Marco & 1102360 & marco.bonolo@studenti.unipd.it \\ \hline
				Pace Giulio & 1102974 & giulio.pace@studenti.unipd.it \\ \hline
				Pezzuto Francesco & 1116523 & francesco.pezzuto@studenti.unipd.it \\ \hline
				Sanna Giovanni & 1029744 & giovannibruno.sanna@studenti.unipd.it \\ \hline
				Sovilla Matteo & 1124500 & matteo.sovilla@studenti.unipd.it \\
				\noalign{\hrule height 1.4pt}
			\end{tabular}
			\caption{Organigramma del gruppo\label{tab:table_label}}
		\end{table}
		\vspace{1.4cm}
		\par Rimango a Sua completa disposizione per ogni ulteriore chiarimento e la ringrazio per l'attenzione.\\
		\vspace{1cm}
		Distinti saluti,
		\flushright Sovilla Matteo\\
		\vspace{0.4cm}
		\includegraphics[scale=0.5]{../../../Immagini/Firme/MatteoSovilla.png}
		
	\end{titlepage}
\end{document}
