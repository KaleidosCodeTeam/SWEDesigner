% Document-Author: KaleidosCode
% Document-Date: 10/03/2017 aggiornare la data
% Document-Description: Glossario dei termini per SWEDesigner

\documentclass[a4paper,12pt]{article}
\usepackage{../../../templates/kaleidos}
\usepackage{titletoc}

\author{KaleidosCode}
\date{10/03/2017}	% aggiornare la data

 
 \begin{document}
 	\begin{titlepage}
 		\centering
 		\logo
 		\vspace{1.2cm}
 		\flushright Alla gentile concessione del Committente: \\
 		\vardanega \\
 		\cardin\\
 		Università degli Studi di Padova \\
 		Dipartimento di matematica\\
 		via Trieste 63\\
 		35121, Padova(PD)\\
 		\vspace{0.5cm}
 		1 aprile 2017\\
 		\vspace{1.4cm}
 		\flushleft
 		Oggetto: \textbf{presentazione della proposta per il capitolato d’appalto C6}
 		\vspace{0.4cm}
		\flushleft	\responsabilediprogetto\\
		KaleidosCode 
		
		\vspace{0.6cm}
		Egregio \vardanega ,\\
		\vspace{0.4cm}
		Con la presente, il gruppo KaleidosCode intende comunicarLe ufficialmente l’impegno alla realizzazione del prodotto da Lei commissionato, denominato:
		
		%%non ricordo il comando%%
		
		L’offerta è corredata dai seguenti documenti, allegati alla presente lettera:
		\begin{itemize}
			\item \analisideirequisitiv\  (EsterniAnalisiDeiRequisiti\char`_v1.1.0.pdf);
			\item \glossariov\  (Esterni/Glossario\char`_v1.2.0.pdf);
			\item \normediprogettov\  (Interni/NormeDiProgetto\char`_v1.2.0.pdf);
			\item \pianodiprogettov\  (Esterni/PianoDiProgetto\char`_v1.2.0.pdf);
			\item \pianodiqualificav\  (Esterni/PianoDiQualifica\char`_v1.2.0.pdf);
			\item \studiodifattibilitav\  (Interni/StudioDiFattibilita\char`_v1.2.0.pdf);
			\item VerbaleEsterno17/03/2017  (Esterni/VerbaleEsterno\char`_27-02-2017\char`_v1.2.0.pdf).
			\item VerbaleEsterno23/03/2017  (Esterni/VerbaleEsterno\char`_23-02-2017\char`_v1.2.0.pdf).
\end{itemize}
		
		Il gruppo KaleidosCode ha stimato di consegnare il prodotto richiesto entro la fine del secondo trimestre dell’anno accademico 2016–2017 con un preventivo di costo pari a 12.500 e.\\
		I dettagli di analisi del prodotto, di pianificazione e di qualità sono trattati in maniera approfondita nei documenti allegati.\\
		\vspace{0.5cm}
		Di seguito viene presentato l’organigramma del team:
		\vspace{0.4cm}
		\begin{table}[H]
			\center
			\begin{tabularx}{\textwidth}{|X|X|X|}
				\noalign{\hrule height 1.5pt}
				\textbf{Nome} & \textbf{Matricola} & \textbf{Posta elettronica}     \\
				\hline
				Bonato Marco  &  &\\
				\hline
				Bonolo Enrico  &   &\\
				\hline
				Pace Giulio  &   &\\
				\hline
				Pezzuto Francesco  &   &\\
				\hline
				Sanna Giovanni &   &\\
				\hline
				Sovilla Matteo &   &\\
				\noalign{\hrule height 1.5pt}
			\end{tabularx}
			\caption{Organigramma.  \label{tab:table_label}}
		\end{table}
		
		
		Rimango a Sua completa disposizione per ogni ulteriore chiarimento.
		La ringrazio anticipatamente della Sua attenzione.\\
		\vspace{1cm}
		Cordiali Saluti,
		\flushright Matteo Sovilla
		
		
 		
 		
 		
 		
 		
 		
 		
 		
 		
 		
 		
	\end{titlepage}
\end{document}