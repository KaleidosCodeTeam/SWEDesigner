\documentclass[../PianoDiProgetto.tex]{subfiles}
\begin{document}
	\section{Pianificazione}
	
		\subsection{Analisi}
		\textbf{Periodo} : Da 27/02/2017 a 25/03/2017. \\
		Questa fase comincia con la formazione del gruppo di lavoro e prosegue fino al completamento della prima stesura dei documenti necessari alla Revisione dei Requisiti.
		\begin{itemize}
			\item \textbf{Norme di progetto}: vengono stilate le \normediprogetto\ in base alle direttive emanate 
			dall'\amministratore\. Questa attività è anticipata rispetto alle altre poiché ne regola direttamente lo svolgimento;
			\item \textbf{Piano di qualifica}: viene stilato il  \pianodiqualifica ;
			\item \textbf{Studio di fattibilità}: vengono valutati singolarmente tutti i capitolati e viene redatto uno \studiodifattibilita\. Al termine dello studio si sceglie il capitolato da sviluppare come progetto didattico;
			\item \textbf{Analisi dei requisiti}: viene redatta la prima versione dell'\analisideirequisiti\ approfondendo l'analisi di base svolta nell'ambito dello \studiodifattibilita ;
			\item \textbf{Piano di progetto}: viene redatto il \pianodiprogetto\ sulla base delle scadenze e del modello di sviluppo adottato. Una prima fase di pianificazione viene svolta dal \responsabilediprogetto\ in contemporanea alla fase di individuazione degli strumenti e alla stesura delle norme di progetto. Questo eviterà periodi di stallo iniziale, dal momento che questa attività regola tutte le altre;
			\item \textbf{Glossario}: contestualmente alla redazione degli altri documenti viene compilato un glossario che contenga la spiegazione dei termini considerati di non immediata comprensione.
		\end{itemize}
		% DIAGRAMMA DI GANTT DELLE ATTIVITÀ
		\begin{figure}[H]
			\centering
			\includegraphics[scale=0.55]{Figures/Gantt_Analisi.jpg}
			\caption{Analisi: Diagramma di Gantt}
		\end{figure}
			
			
			
		\subsection{Analisi di dettaglio}
		\textbf{Periodo} : Da 26/03/2017 a 03/04/2017. \\
		Questa fase comincia con la fine della fase di analisi e prosegue fino alla scadenza della consegna della Revisione dei Requisiti.
		\begin{itemize}
			\item \textbf{Analisi di dettaglio}: si approfondisce quanto svolto in fase di analisi e si migliora in particolare il documento \analisideirequisiti ;
			\item \textbf{Incremento e Verifica}: se necessario vengono aggiornati e verificati i documenti redatti in precedenza.
		\end{itemize}
		% DIAGRAMMA DI GANTT DELLE ATTIVITÀ
		\begin{figure}[H]
			\centering
			\includegraphics[scale=0.55]{Figures/Gantt_AnalisiDettaglio.jpg}
			\caption{Analisi di dettaglio: Diagramma di Gantt}
		\end{figure}
	
	
	
		\subsection{Progettazione architetturale}
		\textbf{Periodo} : Da 04/04/2017 a 27/04/2017. \\
		Questa fase comincia al termine della fase di Analisi di dettaglio e termina con un incontro di presentazione ufficiale con il proponente.
		\begin{itemize}
			\item \textbf{Specifica Tecnica}: viene redatta la \specificatecnica\ dove sono esposte le scelte progettuali di alto livello del prodotto finale. Sono qui descritti i design pattern utilizzati, l'architettura generale del prodotto e il tracciamento dei requisiti;
			\item \textbf{Incremento e Verifica}: se necessario vengono aggiornati e verificati i documenti redatti in precedenza.
		\end{itemize}
		% DIAGRAMMA DI GANTT DELLE ATTIVITÀ
		\begin{figure}[H]
			\centering
			\includegraphics[scale=0.55]{Figures/Gantt_ProgettazioneArchitetturale}
			\caption{Progettazione architetturale: Diagramma di Gantt}
		\end{figure}
		
		
		
		\subsection{Progettazione di dettaglio e Codifica}
		\textbf{Periodo} : Da 28/04/2017 a 20/06/2017. \\
		Questa macro-fase comincia al termine della fase di Progettazione architetturale e prosegue fino alla scadenza della consegna della Revisione di Qualifica.
		È a sua volta divisa in 3 grandi iterazioni che riguardano Progettazione di dettaglio e Codifica rispettivamente dei requisiti obbligatori, desiderabili e opzionali.
		\begin{itemize}
			\item \textbf{Definizione di prodotto}: viene redatta la \definizionediprodotto\ dove vengono definite approfonditamente la struttura e le relazioni dei vari componenti del prodotto in accordo con quanto descritto nella \specificatecnica ;
			\item \textbf{Codifica}: inizia in questa fase lo sviluppo del codice del prodotto, seguendo la struttura stabilita dalla \definizionediprodotto ;
			\item \textbf{Manuale Utente e Manuale Amministratore}: contestualmente alla progettazione di dettaglio del prodotto si redigono i manuali contenenti le linee guida per l'utilizzo del prodotto; 
			\item \textbf{Incremento e Verifica}: se necessario vengono aggiornati e verificati i documenti redatti in precedenza.
		\end{itemize}
		% DIAGRAMMA DI GANTT DELLE ATTIVITÀ
		\begin{figure}[H]
			\centering
			\includegraphics[scale=0.55]{Figures/Gantt_DettaglioObbligatori}
			\caption{Progettazione di dettaglio e Codifica: Diagramma di Gantt}
		\end{figure}
		\begin{figure}[H]
			\centering
			\includegraphics[scale=0.7]{Figures/Gantt_DettaglioOpz}
			\caption{Progettazione di dettaglio e Codifica: Diagramma di Gantt}
		\end{figure}
	
	
		
		\subsection{Validazione}
		\textbf{Periodo} : Da 21/06/2017 a 06/07/2017. \\
		Questa fase comincia alla fine della fase di Progettazione di dettaglio e Codifica e prosegue fino alla scadenza della consegna della Revisione di Accettazione.
		\begin{itemize}
			\item \textbf{Validazione}: si controlla che il prodotto soddisfi i requisiti specificati nel documento di \analisideirequisiti ;
			\item \textbf{Collaudo}: il prodotto viene testato in ogni funzionalità richiesta dal capitolato;
			\item \textbf{Incremento e Verifica}: se necessario vengono aggiornati e verificati i documenti redatti in precedenza;
			\item \textbf{Consegna}: il prodotto e i documenti prodotti vengono consegnati al committente durante la Revisione di Accettazione.
		\end{itemize}
		% DIAGRAMMA DI GANTT DELLE ATTIVITÀ
		\begin{figure}[H]
			\centering
			\includegraphics[scale=0.55]{Figures/Gantt_Validazione.jpg}
			\caption{Validazione: Diagramma di Gantt}
		\end{figure}
			
\end{document}
