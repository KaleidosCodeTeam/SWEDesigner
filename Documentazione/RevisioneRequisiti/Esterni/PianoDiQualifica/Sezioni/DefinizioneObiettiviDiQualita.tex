\documentclass[../PianoDiQualifica.tex]{subfiles}
\begin{document}

\section{Definizione obiettivi di qualità}
	Prendendo come riferimento lo standard [ISO/IEC 9126] e lo standard [ISO/IEC 12207] il team si impegna a garantire che SWEDesigner abbia le seguenti qualità:
	\subsection{Funzionalità}
		Si garantisce che il sistema prodotto abbia tutte le funzionalità che il documento \analisideirequisitiv\ indica. L'implementazione di ogni requisito deve essere quanto più completa ed economica.
		\begin{itemize}
			\item \textbf{Misura}: l'unità di misura utilizzata sarà la quantità di requisiti mappati in componenti del sistema create e funzionanti; %se ci pensate bene questa frase ha senso
			\item \textbf{Metrica}: la sufficienza è raggiunta quando vengono soddisfatti tutti i requisiti obbligatori;
			\item \textbf{Strumenti}: il sistema deve superare tutti i test previsti dai dalla documentazione prodotta e consegnata in sede di Revisione dei Requisiti.
			%l'ho scritto io, è giusto?		
		\end{itemize}
	\subsection{Affidabilità}
		Il sistema deve essere quanto più possibile robusto. Nel caso di eventuali errori deve essere di facile ripristino.
		\begin{itemize}
			\item \textbf{Misura}: l'unità di misura utilizzata sarà la quantità di esecuzioni che vanno a buon fine;
			\item \textbf{Metrica}: visto che non è possibile valutare a monte tutte le possibili casistiche di utilizzo le esecuzioni dovranno il più possibile coprire la possibile gamma di possibilità. Per questo motivo è impossibile stabilire oggettivamente una esatta soglia che corrisponda alla sufficienza;
			\item \textbf{Strumenti}: ancora da definire.
		\end{itemize}
	\subsection{Usabilità}
		Il sistema deve risultare per quanto possibile intuitivo e di facile utilizzo. Deve coniugare una facilità di apprendimento e utilizzo con il soddisfacimento di tutte le necessità dell'utente.
		\begin{itemize}
			\item \textbf{Misura}: poichè non esiste una metrica oggettiva che riguarda questo ambito l'unità di misura utilizzata sarà una valutazione soggettiva dell'usabilità;
			\item \textbf{Metrica}: non esistendo una metrica oggettiva è impossibile determinare con certezza quale sia la sufficienza. In ogni caso i membri del gruppo si impegneranno a garantire un'usabilità più alta possibile;
			\item \textbf{Strumenti}: si vedano le \normediprogettov\ .
		\end{itemize}
	\subsection{Efficienza}
		Il sistema deve ridurre al minimo l'utilizzo delle risorse impiegate e deve fornire le funzionalità richieste nel minor tempo possibile.
		\begin{itemize}
			\item \textbf{Misura}:
			\item \textbf{Metrica}:
			\item \textbf{Strumenti}: 	
		\end{itemize}
	\subsection{Manutenibilità}
		Il sistema deve essere più possibile estensibile e comprensibile.
		\begin{itemize}
			\item \textbf{Misura}:l'unità di misura utilizzata sarà quella descritta nella sezione "Metriche per il codice" %segnare bene con un riferimento al capitolo una volta che viene  scritto
			\item \textbf{Metrica}: il prodotto deve avere la sufficienza in tutte le metriche descritte nella sezione "Metriche per il codice".
			\item \textbf{Strumenti}: si vedano le \normediprogettov\ .	
		\end{itemize}
	\subsection{Portabilità}
		Il sistema deve essere più portabile possibile. Il \gl{front end} dovrà funzionare correttamente su più browser possibile. Inoltre dovrà essere supportato da più sistemi operativi possibili.
		\begin{itemize}
			\item \textbf{Misura}: il front end deve rispettare gli standard \gl{W3C};
			\item \textbf{Metrica}: Il software dovrà avere le caratteristiche di portabilità descritte. Per questo motivo sarà necessario raggiungere la sufficienza in tutte le metriche descritte nella sezione "Metriche per il codice";
			\item \textbf{Strumenti}: si vedano le \normediprogettov\ .	
		\end{itemize}
	\subsection{Altre qualità}
		Saranno importanti per la qualità del progetto anche i seguenti aspetti:
		\begin{itemize}
				\item \textbf{incapsulamento}: un buon livello di incapsulamento è preferibile in quanto aumenta la riusabilità e la manutenibilità del codice. A questo scopo saranno quindi utilizzate interfacce dove possibile
				\item \textbf{coesione}: le funzionalità che concorrono a uno stesso obiettivo devono risiedere nello stesso componente in modo da favorire semplicità e manutenibilità. In questo modo viene inoltre ridotto l'indice di dipendenza.
		\end{itemize}
\end{document}