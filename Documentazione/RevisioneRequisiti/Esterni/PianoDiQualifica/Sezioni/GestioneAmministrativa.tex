\documentclass[../PianoDiQualifica.tex]{subfiles}
\begin{document}

\section{Gestione amministrativa}
	\subsection{Definizione di un errore}
		Si verifica un errore quando si presenta una delle seguenti condizioni:
		\begin{itemize}
			\item errore ortografico prodotto all'interno dei documenti
			\item violazione dei vincoli imposti dalle norme tipografiche nei documenti
			\item violazione dei vincoli imposti dalle norme tipografiche nel codice
			\item violazione degli indici delle misure e delle metriche prefissate
			\item violazione dei vincoli di prodotto
		\end{itemize}
	\subsection{Comunicazione degli errori}
		La comunicazione degli errori viene svolta dai verificatori attraverso l'apposita sezione su \gl{Asana}.
	\subsection{Risoluzione degli errori}
		Ogni qualvolta vengono riscontrati degli errori, il team si riunisce per decidere come intervenire per risolvere  gli errori. La risoluzione può quindi essere assegnata ad un solo membro del gruppo o ad un sottoinsieme ristretto di esso.
\end{document}