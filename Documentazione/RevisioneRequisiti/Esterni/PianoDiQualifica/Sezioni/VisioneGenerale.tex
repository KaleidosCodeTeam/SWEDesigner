\documentclass[../PianoDiQualifica.tex]{subfiles}
\begin{document}
	\section{Visione generale della strategia}
	Per garantire la qualità dei prodotti realizzati durante lo sviluppo del progetto, è indispensabile definire e perseguire strategie che assicurino la qualità dei processi adottati, nonchè il loro continuo miglioramento; inoltre, è necessario definire metriche e pianificare attività che valutino in modo preciso la qualità dei prodotti ottenuti e dei processi adottati. A tal scopo, verranno adottate le seguenti strategie:
	\begin{itemize}
	\item Definizione accurata di norme che regolamentano e standardizzano i processi coinvolti nel progetto, in termini di:
		\begin{itemize}
		\item Processi di fornitura;
		\item Processi di sviluppo;
		\item Processi di supporto;
		\item Processi organizzativi;
		\end{itemize}
	\item Descrizione dettagliata delle strategie di pianificazione adottate per sviluppo del progetto, in termini di:
		\begin{itemize}
		\item Modello di sviluppo adottato;
		\item Analisi dei rischi che si possono incontrare;
		\item Pianificazione delle attività e dei tempi;
		\item Stima preventiva delle risorse che saranno impiegate;
		\item Assegnazione delle risorse, al fine di portare a termine le attività pianificate nei tempi previsti;
		\item Consultivo, durante lo sviluppo del progetto, delle risorse impiegate;
		\end{itemize}
	\item Ad ogni processo coinvolto nello sviluppo del progetto verrà applicato il principio \gl{PDCA}, affiancato dal modello \gl{CMM}. Essi permettono il controllo, la valutazione e il miglioramento continuo dei processi, nonchè la determinazione del livello di maturità dell'organizzazione nel gestire tali processi.
	\end{itemize}
	
		\subsection{Organizzazione}
		La gestione della strategia di verifica si basa sull'attuazione delle relative attività descritte nelle \normediprogetto\ . Tali attività vengono eseguite per ogni processo attuato, allo scopo di verifica della qualità del processo stesso e dell'eventuale prodotto ottenuto, facendo riferimento anche alle metriche definite nel presente documento [..].
		Ogni documento prevede un diario delle modifiche che permette di concentrare l'attività di verifica solo nelle parti modificate dopo l'ultima verifica eseguita.
		Data la diversa natura dei prodotti ottenuti dalle fasi del progetto, si necessita, per ognuna di esse, una diversa procedura di verifica:
		\begin{itemize}
		\item \textbf{Analisi}: I metodi di verifica utilizzati per questa fase sono descritti nelle \normediprogetto\ ;
		\item \textbf{Analisi di dettaglio}: in tale fase verranno verificati i processi che porteranno all'incremento dei prodotti realizzati nella fase precedente; verrà inoltre garantito che tutti i requisiti possano essere rintracciabili. I metodi utilizzati per la verifica di questa fase saranno descritti nelle \normediprogetto\ e incrementati nelle fasi successive.
		\item \textbf{Progettazione e codifica}: in tale fase verranno verificati i processi che porteranno all'incremento dei prodotti realizzati nella fase precedente. L' attività di verifica per questa fase prevede l'esecuzione di test pianificati, come in [???].  I metodi utilizzati per la verifica di questa fase saranno descritti nelle \normediprogetto\ e incrementati nelle fasi successive.
		\end{itemize} 
		
		\subsection{Scadenze temporali}
		Dato l'obiettivo di rispettare le scadenze fissate nel \pianodiprogetto\, è indispensabile pianificare l'attività di verifica della documentazione e del codice prodotto, in modo che risulti sistematica e organizzata; grazie all'applicazione di tale strategia, l'individuazione e la correzione degli errori averrà il prima possibile, impedendo la loro rapida diffusione e mitigando la possibilità che gli stessi si ripresentino in futuro; diminuendo così il rischio di ritardi. Tale pianificazione è documentata nel \pianodiprogetto\, il quale contiene, nella sottosezione 1.4, anche le scadenze temporali che il gruppo \kaleidoscode\ si impegna a rispettare.
		\subsection{Responsabilità}
		I ruoli responsabili delle attività di verifica sono il \responsabilediprogetto\ e il \verificatore. I loro compiti e responsabilità, descritti nelle \normediprogetto, permettono alle attività di verifica di essere efficienti e sistematiche.


\end{document}