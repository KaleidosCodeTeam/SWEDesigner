\documentclass[../StudioDiFattibilita.tex]{subfiles}
\begin{document}
	\section{Capitolato C3}
		\subsection{Descrizione}
			Il \gl{progetto} prevede la realizzazione di un applicativo web che disegni e descriva gli scenari di danno, con particolare focus sulle catastrofi naturali, che posso colpire un'azienda.
		\subsection{Dominio applicativo}
			L'applicazione è rivolta a ogni azienda, in quanto chiunque è potenzialmente interessato a conoscere i possibili rischi legati alla loro attività.
		\subsection{Dominio tecnologico} %posso usare proponente cosi??
			Il \gl{proponente} non richiede uno \gl{stack} tecnologico particolare. Suggerisce però le seguenti teconologie:
			\begin{itemize}
				\item \textbf{\gl{Amazon Web Services}} per l'archiviazione dati;
				\item \textbf{\gl{Asana}} per la gestione dei processi;
				\item \textbf{\gl{Bootstrap}} e \textbf{\gl{JavaScript}} per la realizzazione dell'applicazione;
				\item \textbf{\gl{Slack}} per la comunicazione.
			\end{itemize}
		\subsection{Valutazione}
			\subsubsection{Aspetti positivi}
				Gli aspetti ritenuti positivi di questo \gl{progetto} sono:
				\begin{itemize}
					\item lo \gl{stack} tecnologico consigliato dal \gl{proponente} sembra ragionevolmente semplice da utilizzare;
					\item il \gl{progetto} nel complesso riguarda una tematica interessante per alcuni membri del gruppo.
				\end{itemize}					
			\subsubsection{Fattori di rischio}
			I fattori di rischio che sono stati individuati sono:
			\begin{itemize}
				\item difficoltà di contatto con il \gl{proponente};
			\end{itemize}
			\subsection{Conclusioni}
			Il progetto sembra interessante per lo \gl{stack} tecnologico consigliato e per la tematica affrontata. Il fatto che il \gl{proponente} si trova all'estero, però, è un fattore di rischio molto alto in quanto il gruppo teme che la comunicazione possa essere difficile e frammentaria. Per questo motivo il \gl{progetto} è stato scartato.
\end{document}