\documentclass[../StudioDiFattibilita.tex]{subfiles}
\begin{document}
	\section{Capitolato scelto}
		\subsection{Descrizione}
		Si è scelto di presentare una proposta d'appalto per il capitolato C6 - SWEDesigner.
		Si tratta di un editor di diagrammi UML che generi il relativo codice Java e Javascript automaticamente.
		In particolare si richiede di realizzare, tra i diagrammi previsti dall'UML, il diagramma delle classi e il diagramma delle attività 
		e di ricavare da questi rispettivamente lo scheletro delle classi e il corpo dei metodi nei due linguaggi di programmazione
		indicati.
		\subsection{Studio del dominio}
			\subsubsection{Dominio applicativo}
			Il capitolato si colloca nel dominio degli strumenti per la realizzazione di nuovo software.
			I diagrammi UML si sono dimostrati finora deboli nell'accoppiamento tra il codice prodotto in fase di implementazione e 
			i diagrammi disegnati durante la sua progettazione. È richiesto al fornitore di esplorare iterazioni ed eventuali 
			estensioni che avvicinino le due fasi, in maniera da rendere possibile almeno all'interno di un dominio predeterminato
			la creazione di codice funzionante ottenuto dai soli diagrammi.
			\subsubsection{Dominio tecnologico}
			Citando il testo del capitolato, il sistema dovrà essere realizzato con tecnologie Web.
			In particolare si richiede che la parte server venga realizzata in Java con server Tomcat o Javascaript con server
			Node.Js. La parte client dovrà essere eseguibile in un browser HTML5 ed utilizzare fogli stile CSS per l’aspetto 
			estetico e Javascript per la parte attiva.
		\subsection{Valutazione}
			\subsubsection{Aspetti positivi}
			Il capitolato è stato valutato positivamente per questi motivi principali:
			\begin{itemize}
				\item Vi è un alto interesse nell'affrontare un progetto che preveda una fase di ricerca e permetta di
				confrontarsi con un problema aperto;
				\item Si ritiene che lavorare all'interno del dominio tecnologico sopra riportato sia altamente formativo per
				l'ampia diffusione di cui godono le tecnologie richieste all'interno del mondo del lavoro;
				\item La diffusione delle tecnologie richieste porta ad un'ampia disponibilità di documentazione a riguardo, oltre
				che ad una notevole quantità di software open source che apre buone possibilità per la riusabilità del codice.
			\end{itemize}
			\subsubsection{Fattori di rischio}
			\begin{itemize}
				\item Inesperienza sulle tecnologie adottate: il gruppo non ha mai lavorato con la base di software richiesta, e
				ciò richiederà un’importante investimento di ore sulla formazione personale;
				\item Conoscenza dei diagrammi UML poco approfondita: sebbene siano stati affrontati durante il corso di
				ingegneria del software, è la prima volta che i componenti del gruppo si ritrovano ad usare questo tipo di
				diagrammi. Sarà necessario colmare in fretta le possibili lacune trattandosi di un tema centrale nello sviluppo
				del progetto;
				\item Ambito sperimentale: la conversione da diagramma UML a codice non è sempre possibile. Ciò introduce
				un fattore di rischio considerevole in quanto sarà compito del gruppo individuare un dominio opportuno e
				gestire eventuali situazioni particolari non previste a priori.
			\end{itemize}
		\subsection{Analisi di mercato}
		Si riporta dal capitolato:
		“L’ innovazione oggi è la costante di qualunque settore di attività lavorativa. Motore dell’innovazione è il software, che
		permette di inserire elementi di agilità ed intelligenza in ogni attività umana, dalla fornitura di servizi alle realizzazioni
		meccaniche. La costante richiesta di nuovo software si scontra con la cronica mancanza di esperti e con la bassa qualità del
		software prodotto; per aumentare la qualità e la velocità di produzione occorre rendere questa attività un processo
		industriale ingegnerizzato allontanandosi dall’ artigianalità che ancora a volte lo caratterizza.”
		ùLo svolgimento del progetto porterà allo sviluppo di un prodotto che si propone di colmare questa lacuna rendendo più
		agevole la progettazione di software di qualità.
		\subsection{Conclusioni}
		Per il grande interesse suscitato, la voglia di mettersi in gioco per cercare una soluzione ad un problema attuale e l’importante
		valore formativo del progetto, il gruppo ha deciso di presentare una proposta d’appalto per il capitolato C6.
\end{document}